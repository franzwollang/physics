% !TeX program = pdflatex
\documentclass[11pt]{article}
\usepackage[a4paper,margin=1in]{geometry}
\usepackage{amsmath,amssymb,amsfonts}
\usepackage{graphicx}
\usepackage{bm}
\usepackage{mathtools}
\usepackage{microtype}
\usepackage{enumitem}
\usepackage{hyperref}
\usepackage{xr-hyper}
\usepackage{fancyhdr}

% External references (optional, for future SI)
\externaldocument{SI}

\hypersetup{colorlinks=true,linkcolor=blue,citecolor=blue,urlcolor=blue}

% Footer marking the draft status
\pagestyle{fancy}
\fancyhf{}
\fancyfoot[C]{\small Draft — version posted to Zenodo on 2025-09-12}
\renewcommand{\headrulewidth}{0pt}
\renewcommand{\footrulewidth}{0pt}

% ---------- Macros (minimal, aligned with house style) ----------
\newcommand{\cs}{c_s}
\newcommand{\Heff}{H_{\mathrm{eff}}}
\newcommand{\Nband}{\mathcal N}
\newcommand{\dd}{\mathrm d}
\newcommand{\rcl}{R_{\mathrm{cl}}}
\newcommand{\vflat}{v_{\mathrm{flat}}}
\newcommand{\kmin}{k_{\min}}
\newcommand{\kmax}{k_{\max}}

% Shortcuts
\newcommand{\grad}{\nabla}

\title{Scale\,–\,Downshifting as a Unified Origin for Dark Matter and Dark Energy (\textit{Draft})}
\author{Franz Wollang\\ \small Independent Researcher}
\date{\small Dated: 2025-09-12}

\begin{document}
\maketitle

% Prominent draft disclaimer box
\begin{center}
\setlength{\fboxsep}{8pt}%
\fbox{\parbox{0.92\textwidth}{\centering\bfseries DRAFT — NOT FOR CITATION\\[4pt]
This is a preliminary working version posted for discussion and feedback. Content may change significantly before formal submission.}}
\end{center}
\vspace{1em}

\begin{abstract}
We propose a framework that explains the phenomena of both dark matter and dark energy as dual facets of a single underlying mechanism: the conformal scaling of matter in response to a fractal vacuum noise field. Building only on the minimal foundations established in prior work (the Infinite\,–\,Clique Graph foundations and the Gravity paper)—namely matter as solitons, a noise field sourced by matter, and a sliding observational window—we derive the major observational signatures of the $\Lambda$CDM model. We show that \emph{spatial, inhomogeneous scale\,–\,downshifting} in response to local noise gradients naturally produces the isothermal halos, flat galactic rotation curves, and gravitational lensing effects attributed to dark matter. In parallel, a \emph{temporal, homogeneous scale\,–\,downshifting}, arising from the slow evolution of an observer's perceptual window through the static fractal vacuum, naturally generates an effective cosmic acceleration with an equation of state $w \approx -1$, explaining dark energy. This unified model provides a self\,–\,consistent physical narrative and makes a series of concrete, falsifiable predictions that distinguish it from standard particle\,–\,based dark sector models.
\end{abstract}

\section{Minimal Foundations (from ICG and Gravity)}
We import the following ingredients from the ICG foundations and Gravity papers and use them without re\,–\,deriving in full:
\begin{itemize}[leftmargin=*]
  \item \textbf{Localized, self\,–\,coherent matter and local units:} Matter consists of stable, localized excitations (solitons). Their characteristic size and internal frequency define local rulers and clocks, with a conformal factor $F\equiv L_{\mathrm{ref}}/L$ (hence $a_{\mathrm{eff}}=1/F$).
  \item \textbf{Matter\,–\,sourced noise and the Coulombic window:} Energy–momentum inhomogeneities source a scalar\,–\,like noise field $\rho_N$. Near sources the response is screened (Yukawa); outside any finite source, linear superposition yields an effective $1/r$ potential in the static weak\,–\,field window $\rcl \ll r \ll \ell$. In that window the windowed variance obeys the same Poisson problem as the Newtonian potential, fixing $\tau^2 \propto \Phi$ up to a reference offset.
  \item \textbf{Scale–environment coupling (downshifting):} The coarse\,–\,grained energy of a soliton satisfies $E_{\mathrm{eq}} \propto -\tau^2$, so gradients drive motion along $+\grad\tau$ with force $\mathbf F \propto \tau\,\grad\tau$, supplying the attractive sign and the Newtonian window after the $\tau^2\leftrightarrow\Phi$ identification.
  \item \textbf{Fractal/stationary background and sliding window:} The vacuum exhibits approximately scale\,–\,free statistics over broad bands. Observables are computed over a sliding window with edges $(\kmin,\kmax)$, producing a homogeneous drift that appears as an effective near\,–\,$w=\,-1$ component in cosmology and leaves bound systems approximately decoupled in local units.
\end{itemize}

\paragraph{Consistency note (cross\,–\,paper symbols).} We use $\cs$ for the universal signal speed. The local noise proxy $\tau(x)$ maps to the cosmological field via $\tau^2 \propto \rho_N$ and, in the weak\,–\,field window, to the Newtonian potential via $\tau^2 \propto \Phi$. Amplitude\,–\,sector coefficients are $\alpha_{\mathrm{grad}},\,\beta_{\mathrm{pot}}$; dimension\,–\,bias coefficients are $\alpha_{\mathrm{dim}},\,\beta_{\mathrm{dim}}$.

\subsection*{Notation (quick reference)}
\begin{itemize}[leftmargin=*]
  \item $F(t)$: conformal factor (local rulers shrink as $F$ grows); $a_{\mathrm{eff}} = 1/F$.
  \item $\Heff = -\,\dd(\ln F)/\dd t$: effective Hubble parameter; $1+z = F(t_0)/F(t_e)$.
  \item $\rho_N$: vacuum noise field; $\bar{\rho}_N$: homogeneous (mean\,–\,field) part; $\delta\rho_N$: perturbations.
  \item $S(k)$: noise power spectrum; $\kmin$, $\kmax$: IR/UV window edges.
  \item $q = -\,\ddot a\,a/\dot a^{\,2}$: deceleration parameter (standard definition).
\end{itemize}

\section{A Unified Mechanism: The Two Facets of Scale\,–\,Downshifting}
The central thesis is that the phenomena attributed to dark matter and dark energy are dual manifestations of the same underlying process: the response of matter to the vacuum noise field. We separate by response type:
\begin{itemize}[leftmargin=*]
  \item \textbf{Dark Matter} emerges from the \textbf{inhomogeneous, spatial} response of matter to local gradients in the noise field.
  \item \textbf{Dark Energy} emerges from the \textbf{homogeneous, temporal} response of our observational framework to the global baseline of the same noise field.
\end{itemize}

\subsection{The Dark Matter Branch: Gravity and Structure from Spatial Downshifting}\label{sec:dm}
Matter sources a noise field $\rho_N$; the superposition of these short\,–\,range responses from a large object creates an effective $1/r$ potential in the Coulombic window, recovering the Newtonian force law in the weak\,–\,field limit. This emergent gravity is sourced by perturbations $\delta\rho_N$ on top of a mean\,–\,field background $\bar{\rho}_N$. As this mechanism introduces no new long\,–\,range fields coupled to composition, it is consistent with stringent tests of the equivalence principle. In the weak\,–\,field, slow\,–\,motion limit, the effective metric takes on the standard conformal Newtonian form, yielding PPN parameters $\gamma_{\rm PPN} = 1$ and $\beta_{\rm PPN} = 1$ to leading order, in agreement with Solar System observations.

With this foundation, the explanation for galactic dark matter follows directly. The flat rotation curves of galaxies imply a halo with a density profile $\rho(r) \propto r^{-2}$. Such a profile is the natural equilibrium state for a self\,–\,gravitating, isothermal system. The framework provides a physical justification for this isothermal condition: in local units, the conformal compensation between the scale\,–\,free driving bath and local measurement standards yields a radius\,–\,independent velocity dispersion $\sigma$. With $\sigma \approx \mathrm{const}$, the Jeans equation dictates that the halo settles into the $\rho(r) \propto r^{-2}$ profile, which in turn yields flat rotation curves $v_c(r) = \mathrm{const}$ (see Appendix~\ref{app:jeans} for a full derivation). This leads to a concrete, falsifiable prediction: a tight relationship between a galaxy's rotation speed and its halo's velocity dispersion,
\begin{equation}
  v_c^2 \;\approx\; 2\,\sigma^2.\label{eq:vc-sigma}
\end{equation}
The model is also consistent with gravitational lensing and the observed separation of collisionless mass from collisional gas in systems like the Bullet Cluster.

\subsubsection{Galaxies with apparently little dark matter}
Some galaxies are reported to have low mass discrepancies inside the probed radii. In this framework, such cases arise naturally when one or more conditions behind the isothermal tail are violated:
\begin{itemize}[leftmargin=*]
  \item \textbf{Environmental truncation (tides/external field):} A strong host potential elevates the background and effectively shrinks the Coulombic window (the regime $\rcl \ll r \ll \ell$ where the long\,–\,range $1/r$ tail holds). Tides also strip the low\,–\,surface\,–\,brightness outskirts that carry most of the isothermal mass. Expect low discrepancies near massive hosts, with broken symmetry and outer truncation.
  \item \textbf{Non\,–\,equilibrium/anisotropy:} The relations $\rho\propto r^{-2}$ and Eq.~\eqref{eq:vc-sigma} assume steady, roughly isotropic dispersions. Young tidal dwarfs, puffy UDGs, or systems with significant anisotropy $\beta(r)$ and radially varying dispersions $\sigma_r(r)$ can yield underestimates of enclosed mass in simple isotropic Jeans fits.
  \item \textbf{Compact, baryon\,–\,dominated interiors:} High surface density compacts (e.g., UCD\,–\,like or bulge\,–\,dominated dwarfs) can genuinely be baryon\,–\,dominated within the observed aperture even though an isothermal envelope exists further out.
\end{itemize}

\noindent\textit{Predicted observational signatures:}
\begin{enumerate}[leftmargin=*]
  \item recovery of the $1/r$ tail and rising mass discrepancy beyond any tidal/truncation radius;
  \item improved mass inferences when allowing for $\beta(r)$ and $\dd\sigma_r^2/\dd\ln r$ in the Jeans analysis;
  \item alignment of low\,–\,DM inferences with proximity to a massive host and with morphological signs of disturbance.
\end{enumerate}

\subsubsection{Galaxies that are mostly or purely ``dark''}
``Dark'' here means star\,–\,poor but dynamically massive. The model requires only ordinary matter to source $\Phi$ (and thus $\tau$). Gas\,–\,rich, low\,–\,SFE systems (HI\,–\,dominated LSBs, some UDGs) are therefore expected: the isothermal envelope forms provided the Coulombic window holds, while star formation remains inefficient. Claims of “purely dark” galaxies with no detectable stars or gas must still contain baryons (e.g., cold gas, compact remnants, dust) or be tidal artifacts; otherwise they would challenge the assumption that no additional long\,–\,range field exists. The practical expectation is that deeper HI/CO/IR limits will reveal enough baryons to sustain the potential in most cases.

\subsubsection{Departures from the isothermal ideal (regimes and cores)}
Where the screening length $\ell$ is comparable to the probed radii or the Coulombic window fails ($r \not\gg \rcl$ or $r \not\ll \ell$), deviations from $\rho\propto r^{-2}$ are expected: cores (Helmholtz/Yukawa corrections), anisotropic dispersions, and broken superposition in crowded environments. In these regimes the framework predicts smooth departures rather than abrupt failures, with a return to the $1/r$ tail once $\rcl \ll r \ll \ell$ is restored.\\[0.25em]
\noindent Practical diagnostics and modeling recipes (anisotropy $\beta(r)$, slowly varying $\sigma_r(r)$, external\,–\,field/tidal truncation) are collected in SI Section~\ref{si:dm-diagnostics}.

\subsection{The Dark Energy Branch: Apparent Acceleration from Temporal Downshifting}\label{sec:de}
Any observer's measurement of the vacuum energy is necessarily windowed. Even if the underlying fractal spectrum $S(k)$ is static, a slow change in the observational window's boundaries, $\kmin(t)$ and $\kmax(t)$, will induce a drift in the measured energy
\begin{equation}
  \rho_{N,\mathrm{eff}}(t) \;=\; \int_{\kmin(t)}^{\kmax(t)} S(k)\,\dd k.\label{eq:rhoN-eff}
\end{equation}
The rate of this drift provides a new, smooth energy component in the Friedmann equation, driving an effective cosmic acceleration. For a nearly scale\,–\,free spectrum $S(k) \propto 1/k$, one finds the equation of state
\begin{equation}
  w + 1 \;\approx\; \frac{\dot F/F - \dot \Heff/\Heff}{3\,\Heff\,\Nband},\qquad \Nband \equiv \ln\!\left(\frac{\kmax}{\kmin}\right).\label{eq:w-plus-one}
\end{equation}
Derivation and generalizations (alternative $S(k)$, time\,–\,dependent edge scalings $\alpha,\beta$) are provided in SI Section~\ref{si:sliding}.
Using current cosmological parameters gives a back\,–\,of\,–\,the\,–\,envelope estimate $w+1 \approx -1.3\times 10^{-3}$, well within observational bounds. Generalizing the window edge evolution to $\kmax \propto F^{\alpha}$ and $\kmin \propto \Heff^{\beta}$ yields
\begin{equation}
  w + 1 \;\approx\; \frac{\alpha - \beta(1+q)}{3\,\Nband},\label{eq:w-general}
\end{equation}
with the natural choice $\alpha=\beta=1$ consistent with $|w+1| \lesssim 0.02$.

Backreaction of structure formation on the homogeneous baseline is a small, higher\,–\,order effect. A Buchert\,–\,style estimate of the backreaction term $Q_D$ finds it volume\,–\,suppressed and orders of magnitude too small ($|Q_D|/H^2 \lesssim 10^{-3}$) to mimic dark energy or spoil the homogeneous drift.

We retain a single symbol map and regime throughout: $\tau^2 \propto \rho_N$ by definition of the windowed variance; in the static, weak\,–\,field window the same field solves Poisson's equation with the matter source so $\tau^2 \propto \Phi$ up to a reference offset. The Newtonian/isothermal results apply in the Coulombic window $\rcl \ll r \ll \ell$, where $\ell=\sqrt{\alpha_{\mathrm{grad}}/(2\beta_{\mathrm{pot}})}$ is the screening length from the gapped amplitude sector. Under these conditions the Jeans analysis enforces $\rho\propto r^{-2}$ and Eq.~\eqref{eq:vc-sigma} (with mild corrections from anisotropy), matching the Gravity paper's weak\,–\,field sector. Cosmologically, observables are stated in local units co\,–\,scaling with the sliding window, so homogeneous drifts affect dimensionful numbers but not dimensionless laws to leading order.

\section{Consistency with the Standard Cosmological History and Falsifiable Predictions}
A viable model must not only explain the dark sector but also remain consistent with the well\,–\,established history of the early universe and make unique, testable predictions.
\begin{itemize}[leftmargin=*]
  \item \textbf{Early Universe \& BBN:} Require that the conformal factor $F(t)$ was effectively constant at early times ($z \gtrsim 1000$). The effective dark energy density at recombination is negligible: $f_{\mathrm{EDE}}(z\!\approx\!1100) \approx 10^{-9}$, preserving early\,–\,time physics.
  \item \textbf{Growth of Structure:} Gravity unmodified on large scales ($G_{\mathrm{eff}}=G$). With $H_{\mathrm{eff}}(z)$ matched to $\Lambda$CDM, linear growth obeys
  \begin{equation}
    \delta'' + \Big[2 + (H'/H)\Big] \delta' - \frac{3}{2}\,\Omega_m(a_{\mathrm{eff}})\,\delta = 0\,.
  \end{equation}
  \item \textbf{Falsifiable Predictions:}
  \begin{enumerate}[leftmargin=*]
    \item \textbf{Dark Matter:} Tight relation between a galaxy's flat rotation speed and halo dispersion: $\vflat^{\,2} \approx 2\,\sigma^2$.
    \item \textbf{Dark Energy:} Specific, tiny deviation of $w$ from $-1$, linked to $q$ and $\Nband$.
    \item \textbf{Cross\,–\,Sector (Varying Constants):} Tiny, secular drift in constants. If $\alpha \propto \bar{\rho}_N^{-\zeta}$, then $|\dd \ln \alpha/\dd t| \sim \zeta\,10^{-3} H_0$, giving $|\Delta\alpha/\alpha| \lesssim 7\times 10^{-7}$ over a Gyr for $\zeta \lesssim 10^{-2}$.
    \item \textbf{Environment diagnostic:} Mass discrepancy correlates with host tides/external field. Near massive hosts, expect truncated halos and lower discrepancies inside the truncation radius; beyond that radius, recovery toward the $1/r$ tail.
    \item \textbf{Anisotropy diagnostic:} Allowing for $\beta(r)$ and $\dd\sigma_r^2/\dd\ln r$ in Jeans analyses of “low\,–\,DM” systems should recover higher enclosed masses and approach $\rho\propto r^{-2}$ at larger radii.
    \item \textbf{Gas accounting for dark candidates:} Deep HI/CO/IR searches should reveal sufficient ordinary matter in star\,–\,poor, dynamically massive systems. Robustly baryon\,–\,free, relaxed, isolated massive halos would challenge the framework.
    \item \textbf{Coulombic\,–\,window validity:} Systems violating $\rcl \ll r \ll \ell$ should display smooth, predictable departures from isothermality (cores, slope changes), with a return to $1/r$ behavior where the window is restored.
  \end{enumerate}
\end{itemize}

\section{A Note on the Hubble Tension}
The Hubble tension refers to the significant discrepancy between early-universe (CMB-inferred) and late-universe (local distance ladder) measurements of the Hubble constant, $H_0$. Current measurements show a $\sim 4$–$5\sigma$ disagreement, with early-universe probes yielding $H_0 \approx 67$ km/s/Mpc while late-universe observations give $H_0 \approx 73$ km/s/Mpc. This tension has persisted despite increasingly precise measurements and represents one of the most pressing challenges in modern cosmology.

Unlike $\Lambda$CDM where the cosmological constant is truly constant, this framework's effective dark energy arises from the dynamics of a sliding observational window. The expansion history $\Heff(z)$ is therefore tied to the evolution of the window's boundaries $(\kmin(t), \kmax(t))$. A mild evolution in the scaling of these boundaries at late times—for instance, a small deviation in the $\alpha$ and $\beta$ parameters from their fiducial values in Eq.~\eqref{eq:w-general}—could produce an effective equation of state $w(z)$ that differs slightly from a pure cosmological constant. Specifically, if the window edges evolved more rapidly during the epoch probed by local distance measurements, the effective dark energy density could be enhanced relative to the early-universe extrapolation, leading to a higher inferred $H_0$ from late-time observations.

Such a modification must remain consistent with other key observational pillars, including Big Bang nucleosynthesis, the acoustic peaks in the CMB power spectrum, and Type Ia supernovae data. The challenge lies in finding window evolution that resolves the Hubble tension without disrupting these well-established constraints. We emphasize that this represents a promising direction for future research rather than a definitive solution. The model's inherent flexibility—through its physically motivated, time-varying effective dark energy component—offers a potential avenue to address the tension while maintaining the successful predictions of the standard cosmological model in other regimes. A preliminary exploration of viable edge\,–\,evolution bands is provided in SI Section~\ref{si:hubble}.

\section{Conclusion}
The soliton\,–\,noise framework, built on a minimal set of physical postulates, offers a unified explanation for the dark sector. It recasts dark matter and dark energy not as mysterious substances, but as the inhomogeneous and homogeneous manifestations of a single, underlying process: the conformal downshifting of matter in response to a fractal vacuum. This provides a self\,–\,consistent and highly predictive alternative to the standard $\Lambda$CDM paradigm.

\section{References Prep}
\begin{itemize}[leftmargin=*]
  \item $\Lambda$CDM background and observational pillars (CMB, BAO, SN Ia): standard reviews.
  \item Isothermal halos and Jeans analysis in galactic dynamics (e.g., Binney \& Tremaine).
  \item PPN constraints and Solar\,–\,System tests of gravity (Cassini, LLR; $\gamma_{\rm PPN}$, $\beta_{\rm PPN}$).
  \item Bounds on varying constants (quasar absorption, Oklo, atomic clocks).
  \item Backreaction and Buchert averaging (cosmological coarse\,–\,graining literature).
\end{itemize}

\appendix

\section{Dynamical Foundations of Matter and Gravity (concise)}\label{app:dynamical}
This appendix only sketches prior results used as assumptions here. Stability, relaxation hierarchy, and mean\,–\,field arguments are derived in the ICG and Gravity papers; a compact recap and symbol map are provided in SI Sections~\ref{si:notation} and \ref{si:foundations}. Readers can consult those for full proofs and parameter conditions; we rely only on their leading\,–\,order consequences in the present text.

\section{The Scale\,–\,Space Force and Downshifting Rule (concise)}\label{app:scale-space}
We justify the scale\,–\,noise coupling by showing how an attractive force toward higher noise densities emerges from a generic energy functional for localized excitations.

\paragraph{Coarse\,–\,grained energy (generic scalings for localized packets).}
Consider a localized packet with characteristic size $\sigma$ in a background with noise intensity proxy $\tau(x)$:
\begin{equation}
  E_{\mathrm{coh}}(\sigma, x) = A_{\gamma}\,\sigma^{-2} - A_{\eta}\,\tau(x)\,\sigma^{-1}.\label{eq:E-coh}
\end{equation}

\paragraph{Equilibrium size (minimize over $\sigma$).}
\begin{equation}
  \sigma^*(x) = \frac{2 A_{\gamma}}{A_{\eta}\,\tau(x)},\qquad \text{higher noise }\Rightarrow\; \text{smaller }\sigma.\label{eq:sigma-star}
\end{equation}

\paragraph{Relaxed energy and force.}
\begin{align}
  E_{\mathrm{eq}}(x) &= E_{\mathrm{coh}}\big(\sigma^*(x), x\big) = -\,\frac{A_{\eta}^{\,2}}{4 A_{\gamma}}\,\tau(x)^2,\label{eq:E-eq}\\
  \mathbf F(x) &= -\grad E_{\mathrm{eq}}(x) = +\,\frac{A_{\eta}^{\,2}}{2 A_{\gamma}}\,\tau(x)\,\grad\tau(x).\label{eq:force}
\end{align}
The force points toward higher $\tau$, providing the concentrating (negative) sign.

\paragraph{Physical rationale (two views).}
Thermodynamic: minimizing $F = E - T S$ favors contraction in hotter (higher $\tau$) regions, lowering the total free energy. Optical/action: noise acts like an index; wave packets bend toward higher index, lowering action.

\paragraph{Robustness.}
For a generic balance $E(\sigma,x)=A\,\sigma^{-p}-B\,\tau(x)\,\sigma^{-q}$ with $A,B>0$ and $p>q>0$, one finds $\sigma^*(x)\propto [A/(B\,\tau(x))]^{1/(p-q)}$, $E_{\mathrm{eq}}(x)\propto -\,\tau(x)^{p/(p-q)}$, and $\mathbf F=-\grad E_{\mathrm{eq}}\propto +\,\tau^{p/(p-q)-1}\,\grad\tau$.

\section{The Sliding Window Drift and Dark Energy (foundations)}\label{app:sliding}
This appendix provides only a short reminder. Full derivation of Eq.~\eqref{eq:w-plus-one}, alternative spectral choices, and edge\,–\,evolution parameterizations ($\alpha,\beta$) are given in SI Section~\ref{si:sliding}. Consistency with early\,–\,time physics and bounds on late\,–\,time evolution are summarized in SI Section~\ref{si:early-consistency}.

\section{Isothermal Halos from the Jeans Equation (full derivation)}\label{app:jeans}
This appendix collects the standard Jeans\,–\,equation derivation of flat rotation curves and $\rho\propto r^{-2}$ halos, and connects it to the framework used in the main text.

\subsection{Setup and assumptions}
Consider a steady, spherically symmetric, collisionless system with density $\rho(r)$ and potential $\Phi(r)$. Define the radial velocity dispersion $\sigma_r^2(r)=\langle v_r^2\rangle$ and the Binney anisotropy parameter $\beta(r):=1-\sigma_t^2/(2\sigma_r^2)$, where $\sigma_t$ is the one\,–\,dimensional tangential dispersion. The spherical Jeans equation and Poisson equation are
\begin{align}
  \frac{\dd\, (\rho\,\sigma_r^2)}{\dd r} + \frac{2\beta}{r} \, \rho\,\sigma_r^2 &= -\,\rho\,\frac{\dd\Phi}{\dd r},\\
  \frac{1}{r^2}\,\frac{\dd}{\dd r}\!\left(r^2\,\frac{\dd\Phi}{\dd r}\right) &= 4\pi G\,\rho.\label{eq:jeans-poisson}
\end{align}

\subsection{Isotropic, isothermal case}
Take $\beta=0$ and $\sigma_r(r)\equiv\sigma=\mathrm{const}$. Then the Jeans equation gives
\begin{equation}
  \sigma^2\,\frac{\dd\ln\rho}{\dd r} = -\,\frac{\dd\Phi}{\dd r}.\label{eq:jeans-iso}
\end{equation}
Insert into Poisson:
\begin{equation}
  -\,\sigma^2\,\frac{1}{r^2}\,\frac{\dd}{\dd r}\!\left(r^2\,\frac{\dd\ln\rho}{\dd r}\right) = 4\pi G\,\rho.\label{eq:poisson-iso}
\end{equation}
Seek a power\,–\,law $\rho\propto r^{-\alpha}$ so $\dd\ln\rho/\dd r = -\alpha/r$. Then
\begin{equation}
  -\,\sigma^2\,\frac{1}{r^2}\,\frac{\dd}{\dd r}\!\left(r^2\,\frac{-\alpha}{r}\right) = \frac{\alpha\,\sigma^2}{r^2} = 4\pi G\,A\,r^{-\alpha},\label{eq:power-law}
\end{equation}
which enforces $\alpha=2$ and fixes $A=\sigma^2/(2\pi G)$. Thus the singular isothermal sphere has
\begin{align}
  \rho(r) &= \frac{\sigma^2}{2\pi G}\,\frac{1}{r^2},\label{eq:rho-iso}\\
  v_c^2(r) &:= r\,\frac{\dd\Phi}{\dd r} = 2\,\sigma^2 = \mathrm{const}.\label{eq:vc-flat}
\end{align}
The enclosed mass then grows linearly with radius,
\begin{equation}
  M(r) = \frac{v_c^2\, r}{G}.\label{eq:mass-iso}
\end{equation}

\subsection{Constant anisotropy}
For constant $\beta$ and constant $\sigma_r$, the Jeans equation yields
\begin{equation}
  \sigma_r^2\,\frac{\dd\ln\rho}{\dd r} + \frac{2\beta\,\sigma_r^2}{r} = -\,\frac{\dd\Phi}{\dd r}.\label{eq:jeans-beta}
\end{equation}
Assuming a flat rotation curve (constant $v_c$ so $\dd\Phi/\dd r = v_c^2/r$) and a power\,–\,law $\rho\propto r^{-\alpha}$ gives
\begin{equation}
  v_c^2 = (\alpha - 2\beta)\,\sigma_r^2.\label{eq:vc-beta}
\end{equation}
Self\,–\,consistency with Poisson for a flat curve still imposes $\alpha=2$ ($\rho\propto r^{-2}$), yielding
\begin{equation}
  v_c^2 = 2(1-\beta)\,\sigma_r^2,\label{eq:vc-beta-iso}
\end{equation}
which reduces to the isotropic result $v_c^2=2\sigma^2$ when $\beta=0$.

\subsection{Core regularisation (pseudo\,–\,isothermal)}
The solution above is singular at $r\to0$. Real halos display finite cores. A standard empirical regularisation is the pseudo\,–\,isothermal profile
\begin{equation}
  \rho(r) = \frac{\rho_0}{1+(r/r_c)^2},\qquad
  v_c^2(r) = 4\pi G\rho_0 r_c^2\left[1-\frac{r_c}{r}\arctan\!\left(\frac{r}{r_c}\right)\right],\label{eq:pseudo-iso}
\end{equation}
which approaches $v_c^2\to 4\pi G\rho_0 r_c^2$ for $r\gg r_c$ and $\rho\sim r^{-2}$ asymptotically. In this framework, a finite screening length in the amplitude sector supplies a microphysical core scale playing the role of $r_c$.

\subsection{Connection to the framework}
The “isothermal” condition corresponds to a radius\,–\,independent one\,–\,dimensional dispersion $\sigma$ in \emph{local units}. Conformal co\,–\,scaling of rulers/clocks with the background keeps $\sigma$ approximately constant across the halo, and the Jeans analysis above then enforces the $\rho\propto r^{-2}$ tail and flat $v_c$, with $v_c^2\simeq2\sigma^2$ (or $2(1-\beta)\sigma_r^2$ with mild anisotropy).

\subsection{Deviations and corrections}
In real systems $\sigma_r$ and $\beta$ vary slowly with radius and inner cores regularise the centre. From the general Jeans equation,
\begin{equation}
  \sigma_r^2\,\frac{\dd\ln\rho}{\dd r} + \frac{\dd\sigma_r^2}{\dd r} + \frac{2\beta\,\sigma_r^2}{r} = -\,\frac{v_c^2}{r},\label{eq:jeans-general}
\end{equation}
the local logarithmic density slope is
\begin{equation}
  \alpha(r) \equiv -\,\frac{\dd\ln\rho}{\dd\ln r} = \frac{v_c^2}{\sigma_r^2} + \frac{\dd\ln\sigma_r^2}{\dd\ln r} + 2\,\beta(r).\label{eq:alpha-local}
\end{equation}
Relative to the ideal isotropic–isothermal value $\alpha=2$, slowly rising (falling) dispersions $\dd\ln\sigma_r^2/\dd\ln r>0$ ($<0$) steepen (flatten) the profile, and tangential ($\beta<0$) vs radial ($\beta>0$) anisotropy flattens vs steepens it. A finite core (screening length or coherence scale $r_c$) produces $\rho\!\approx\!\mathrm{const}$ and $v_c\!\propto\! r$ for $r\ll r_c$, transitioning to the $r^{-2}$ tail and flat $v_c$ for $r\gg r_c$.

\end{document}


