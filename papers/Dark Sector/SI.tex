% !TeX program = pdflatex
\documentclass[11pt]{article}
\usepackage[a4paper,margin=1in]{geometry}
\usepackage{amsmath,amssymb,amsfonts}
\usepackage{bm}
\usepackage{mathtools}
\usepackage{microtype}
\usepackage{enumitem}
\usepackage{hyperref}

\hypersetup{colorlinks=true,linkcolor=blue,citecolor=blue,urlcolor=blue}

% Minimal macros to match the main paper
\newcommand{\cs}{c_s}
\newcommand{\Heff}{H_{\mathrm{eff}}}
\newcommand{\Nband}{\mathcal N}
\newcommand{\rcl}{R_{\mathrm{cl}}}
\newcommand{\kmin}{k_{\min}}
\newcommand{\kmax}{k_{\max}}

\title{Supplementary Information: Dark Sector (Draft)}
\author{Franz Wollang}
\date{\small Dated: 2025-09-12}

\begin{document}
\maketitle

\section{Notation and symbol map}\label{si:notation}
We summarize only symbols that are used in the present dark-sector paper and that are not already standard. For complete foundational notation, see the Gravity and ICG papers.
\begin{itemize}[leftmargin=*]
  \item $\cs$: universal signal speed (phase sector). Used to define local light cone.
  \item $\tau(x)$, $\rho_N$: local noise proxy and windowed noise measure; $\tau^2\propto\rho_N$.
  \item $F(t)$, $a_{\rm eff}$: conformal scale factor for local units; $a_{\rm eff}=1/F$.
  \item $\Heff$: effective Hubble parameter for local units, $\Heff=-\partial_t\ln F$.
  \item $\kmin,\kmax$: IR/UV edges of the sliding window in $k$-space; $\Nband=\ln(\kmax/\kmin)$.
  \item $\alpha,\beta$: exponents parameterizing $\kmax\propto F^{\alpha}$ and $\kmin\propto\Heff^{\beta}$.
  \item $\rcl,\ell$: composite source scale and screening length; Coulombic window $\rcl\ll r\ll\ell$.
  \item $\sigma,\sigma_r,\beta(r)$: one-dimensional velocity dispersion, radial dispersion, and Binney anisotropy.
\end{itemize}

\section{Foundations recap and assumptions}\label{si:foundations}
We use the following assumptions established in the Gravity and ICG papers; we restate only what is needed here.
\begin{itemize}[leftmargin=*]
  \item \textbf{Stable localized excitations} (solitons) exist and relax quickly toward a scale set by the local environment. We assume a timescale hierarchy $\tau_{\rm rel}\ll\Heff^{-1}$ so bound systems decouple from homogeneous drift in $F$.
  \item \textbf{Mean-field clamping in the amplitude sector}: a gapped amplitude mode enforces near-constant bulk amplitude, producing a boundary-dominated (shell) response to the environment.
  \item \textbf{Screened superposition}: amplitude-mediated responses are short-range; outside finite sources, linear superposition yields a $1/r$ far-field within $\rcl\ll r\ll\ell$.
  \item \textbf{Phase-channel universality}: the phase sector defines a single Lorentz cone and composition-independent kinematics.
\end{itemize}

\section{Sliding-window formalism and dark energy}\label{si:sliding}
We derive the effective equation of state in the minimal window model and give controlled generalizations.

\paragraph{Setup.} Let the measured vacuum energy be a windowed integral
\begin{equation}
  \rho_{N,\mathrm{eff}}(t) = \int_{\kmin(t)}^{\kmax(t)} S(k)\,\mathrm dk.
\end{equation}
Define the logarithmic bandwidth $\Nband=\ln(\kmax/\kmin)$ and parameterize edge evolution by $\kmax\propto F^{\alpha}$, $\kmin\propto\Heff^{\beta}$.

\paragraph{Leibniz rule.} Differentiating yields
\begin{equation}
  \dot{\rho}_{N,\mathrm{eff}} = S(\kmax)\,\dot{\kmax} - S(\kmin)\,\dot{\kmin}.
\end{equation}
For $S(k)=A\,k^{-p}$ with $p\approx1$, one has $\rho_{N,\mathrm{eff}}\approx A\ln(\kmax/\kmin)$ and $\partial\rho_{N,\mathrm{eff}}/\partial\ln k\approx A$ at the edges, giving
\begin{equation}
  \frac{\dot{\rho}_{N,\mathrm{eff}}}{\rho_{N,\mathrm{eff}}} \;\approx\; \frac{1}{\Nband}\Big(\alpha\,\frac{\dot F}{F} - \beta\,\frac{\dot\Heff}{\Heff}\Big).
\end{equation}

\paragraph{Effective equation of state.} Matching to a smooth component with density $\rho_X$ and $w_X$ gives $\dot{\rho}_X/\rho_X = -3(1+w_X)\Heff$, hence
\begin{equation}
  w_X + 1 \;\approx\; \frac{\alpha\,\dot F/F - \beta\,\dot\Heff/\Heff}{3\,\Heff\,\Nband}.
\end{equation}
The paper uses $\alpha=\beta=1$ as a natural choice. Deviations can be explored under constraints in Section~\ref{si:early-consistency}.

\paragraph{Generalizations.} For $S(k)=A\,k^{-1+\epsilon}$ with small $\epsilon$, the bandwidth becomes $\Nband\to \frac{1}{\epsilon}\big[(\kmax/\kmin)^{\epsilon}-1\big]$, and the prefactor $1/\Nband$ acquires an $\mathcal O(\epsilon)$ correction; to leading order the expression above remains valid with $\Nband$ replaced accordingly.

\section{Early-time consistency constraints}\label{si:early-consistency}
We collect constraints ensuring the window-driven component does not spoil early-universe physics.
\begin{itemize}[leftmargin=*]
  \item \textbf{BBN:} Require that the fractional contribution of the window-driven component at $z\sim10^9$ is negligible. In practice, impose $|w_X+1|\ll1$ and a near-constant $F$ at early times, ensuring $\dot F/F\approx0$ in the radiation era.
  \item \textbf{CMB acoustic peaks:} Bound late-time deviations so that the angular diameter distance to last scattering and the sound horizon are preserved within Planck uncertainties; this restricts the integrated history of $w_X(z)$ and hence the allowed $(\alpha,\beta)$ drifts.
  \item \textbf{Early dark energy fraction:} The crude estimate $f_{\rm EDE}(z_{\ast})\lesssim10^{-3}$ suffices here, consistent with the negligible drift assumptions in the paper.
\end{itemize}

\section{Galactic diagnostics and modeling recipes}\label{si:dm-diagnostics}
We provide practical recipes for cases flagged in the main text.
\subsection*{Anisotropy and dispersion gradients}
Use the general relation
\begin{equation}
  \alpha(r) \equiv -\frac{\mathrm d\ln\rho}{\mathrm d\ln r} = \frac{v_c^2}{\sigma_r^2} + \frac{\mathrm d\ln\sigma_r^2}{\mathrm d\ln r} + 2\,\beta(r)
\end{equation}
to propagate uncertainties from $\beta(r)$ and $\sigma_r(r)$ into mass profiles. When fitting Jeans models, include a linear-in-$\ln r$ term for $\sigma_r^2$ locally and a weakly informative prior on $\beta(r)$.

\subsection*{External-field/tidal truncation}
Model truncation by imposing a taper radius $r_t$ (from tides or external-field effects) beyond which the isothermal envelope transitions smoothly to a steeper decline; use penalized splines or an error-function taper. Predict a recovery toward the $1/r$ tail beyond $r_t$ where the Coulombic window reopens.

\section{Predictions and data pipelines}\label{si:predictions}
\begin{itemize}[leftmargin=*]
  \item \textbf{Halo dispersion vs flat speed:} Test $v_c^2\simeq2\sigma^2$ using outer tracers (HI/GCs) and robust dispersion estimators; flag systems with strong anisotropy or truncation as separate cohorts.
  \item \textbf{Environment correlation:} Bin mass-discrepancy metrics by host potential or group-centric distance; expect lower discrepancies inside truncation radii with recovery beyond.
  \item \textbf{Coulombic-window diagnostics:} Use size ratios ($\rcl/r$, $r/\ell$) and surface-brightness morphology as proxies for window validity; predict smooth departures where violated.
\end{itemize}

\section{Hubble tension exploration}\label{si:hubble}
We sketch minimal parameterizations that could shift late-time $H_0$ inferences without violating early-universe constraints.
\begin{itemize}[leftmargin=*]
  \item \textbf{Edge scalings:} Let $\alpha(z)=1+\delta\alpha\,f(z)$, $\beta(z)=1+\delta\beta\,g(z)$ with transition functions $f,g$ localized at $z\lesssim1$; constrain $(\delta\alpha,\delta\beta)$ by SN Ia and BAO distance fits while holding CMB distances fixed.
  \item \textbf{Bandwidth prior:} Impose a wide prior on $\Nband$ consistent with the fractal window picture; require $\Nband\gg1$ so that $|w_X+1|\ll1$ generically, and explore how modest changes can lift $H_0$.
  \item \textbf{Sensitivity:} Linearize the distance modulus shifts in small $\delta w(z)$ induced by $(\delta\alpha,\delta\beta)$ to estimate the pull on $H_0$.
\end{itemize}

\section{Links to ICG/Gravity proofs}\label{si:links}
For full derivations of stability, induced Lorentz cone, screened superposition, and metric universality, see the Gravity paper (Secs. 2–7) and ICG (Secs. S1–S10). This SI only summarizes the parts used in the dark-sector narrative without repeating proofs.

\end{document}


