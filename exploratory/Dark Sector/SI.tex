% !TeX program = pdflatex
\documentclass[11pt]{article}
\usepackage[a4paper,margin=1in]{geometry}
\usepackage{amsmath,amssymb,amsfonts}
\usepackage{bm}
\usepackage{mathtools}
\usepackage{microtype}
\usepackage{enumitem}
\usepackage{hyperref}

\hypersetup{colorlinks=true,linkcolor=blue,citecolor=blue,urlcolor=blue}

% Minimal macros to match the main paper
\newcommand{\cs}{c_s}
\newcommand{\Heff}{H_{\mathrm{eff}}}
\newcommand{\Nband}{\mathcal N}
\newcommand{\rcl}{R_{\mathrm{cl}}}
\newcommand{\kmin}{k_{\min}}
\newcommand{\kmax}{k_{\max}}

\title{Supplementary Information: Dark Sector (Draft)}
\author{Franz Wollang}
\date{\small Dated: 2025-09-12}

\begin{document}
\maketitle

\section{Notation and symbol map}\label{si:notation}
We summarize only symbols that are used in the present dark-sector paper and that are not already standard. For complete foundational notation, see the ICG paper\footnote{ICG paper: Spacetime from First Principles: Free\,–\,Energy Foundations on the Infinite\,–\,Clique Graph (Draft), Zenodo (2025\,–\,08\,–\,29). DOI: 10.5281/zenodo.15843979.} and the Relativity paper\footnote{Relativity paper: A Sea of Noise: Relativity from a Thermodynamic Force in Scale\,–\,Space (Draft), Zenodo (2025\,–\,08\,–\,30). DOI: 10.5281/zenodo.17000665.}.
\begin{itemize}[leftmargin=*]
  \item $\cs$: universal signal speed (phase sector). Used to define local light cone.
  \item $\tau(x)$, $\rho_N$: local noise proxy and windowed noise measure; $\tau^2\propto\rho_N$.
  \item $F(t)$, $a_{\rm eff}$: conformal scale factor for local units; $a_{\rm eff}=1/F$.
  \item $\Heff$: effective Hubble parameter for local units, $\Heff=-\partial_t\ln F$.
  \item $\kmin,\kmax$: IR/UV edges of the sliding window in $k$-space; $\Nband=\ln(\kmax/\kmin)$.
  \item $\alpha,\beta$: exponents parameterizing $\kmax\propto F^{\alpha}$ and $\kmin\propto\Heff^{\beta}$.
  \item $\rcl,\ell$: composite source scale and screening length; Coulombic window $\rcl\ll r\ll\ell$.
  \item $\sigma,\sigma_r,\beta(r)$: one-dimensional velocity dispersion, radial dispersion, and Binney anisotropy.
\end{itemize}

\section{Foundations recap and assumptions}\label{si:foundations}
We use the following assumptions established in the Relativity paper and the ICG paper; we restate only what is needed here.
\begin{itemize}[leftmargin=*]
  \item \textbf{Stable localized excitations} (solitons) exist and relax quickly toward a scale set by the local environment. We assume a timescale hierarchy $\tau_{\rm rel}\ll\Heff^{-1}$ so bound systems decouple from homogeneous drift in $F$.
  \item \textbf{Mean-field clamping in the amplitude sector}: a gapped amplitude mode enforces near-constant bulk amplitude, producing a boundary-dominated (shell) response to the environment.
  \item \textbf{Screened superposition}: amplitude-mediated responses are short-range; outside finite sources, linear superposition yields a $1/r$ far-field within $\rcl\ll r\ll\ell$.
  \item \textbf{Phase-channel universality}: the phase sector defines a single Lorentz cone and composition-independent kinematics.
\end{itemize}

\section{Sliding-window formalism and dark energy}\label{si:sliding}
The formalism presented here begins from the premise, established in the main text, that a universal process of "cosmic annealing" leads to a slow, homogeneous, and monotonic shrinkage of the absolute equilibrium scale of all matter, $L_*(t)$. This fundamental contraction of local standards of measurement creates the dual perception of an expanding universe. The kinematic drift of the observational window is a direct consequence of this process.

\paragraph{Setup.} Let the measured vacuum energy be a windowed integral
\begin{equation}
  \rho_{N,\mathrm{eff}}(t) = \int_{\kmin(t)}^{\kmax(t)} S(k)\,\mathrm dk.
\end{equation}
Define the logarithmic bandwidth $\Nband=\ln(\kmax/\kmin)$ and parameterize edge evolution by $\kmax\propto F^{\alpha}$, $\kmin\propto\Heff^{\beta}$.

\paragraph{Leibniz rule.} Differentiating yields
\begin{equation}
  \frac{\mathrm d\rho_{N,\mathrm{eff}}}{\mathrm dt} = S(\kmax)\,\frac{\mathrm d\kmax}{\mathrm dt} - S(\kmin)\,\frac{\mathrm d\kmin}{\mathrm dt}.
\end{equation}
For $S(k)=A\,k^{-p}$ with $p\approx1$, one has $\rho_{N,\mathrm{eff}}\approx A\ln(\kmax/\kmin)$ and $\partial\rho_{N,\mathrm{eff}}/\partial\ln k\approx A$ at the edges, giving
\begin{equation}
  \frac{\mathrm d\rho_{N,\mathrm{eff}}/\mathrm dt}{\rho_{N,\mathrm{eff}}} \;\approx\; \frac{1}{\Nband}\Big(\alpha\,\frac{\mathrm d F/\mathrm dt}{F} - \beta\,\frac{\mathrm d\Heff/\mathrm dt}{\Heff}\Big).
\end{equation}

\paragraph{Effective equation of state.} Matching to a smooth component with density $\rho_X$ and $w_X$ gives $\mathrm d\rho_X/\mathrm dt/\rho_X = -3(1+w_X)\Heff$, hence
\begin{equation}
  w_X + 1 \;\approx\; \frac{\alpha\,\mathrm d F/\mathrm dt/F - \beta\,\mathrm d\Heff/\mathrm dt/\Heff}{3\,\Heff\,\Nband}.
\end{equation}
The paper uses $\alpha=\beta=1$ as a natural choice. Deviations can be explored under constraints in Section~\ref{si:early-consistency}.

\subsection{Edge derivation and sensitivity}\label{si:edge-derivation}
\paragraph{UV edge from local resolution.} In local units, the smallest resolvable physical feature is set by the stabilized soliton scale $L_*(t)$. A resolved wavenumber obeys $k\,L_* \gtrsim \mathcal O(1)$, hence the UV cutoff tracks $k_{\max}(t) \sim \kappa_{\rm UV}/L_*(t)$. With $F(t):=1/a_{\rm eff}(t)$ and $L_*\propto 1/F$, one obtains
\begin{equation}
  k_{\max}(t) \;\propto\; F(t)^{\alpha},\qquad \alpha=1\;\text{(natural)}.
\end{equation}
\paragraph{IR edge from causal/response reach.} The largest coherently probed comoving scale in a Hubble time is set by the response cone: $\lambda_{\rm IR}\sim c_s/ H_{\rm eff}$ up to an $\mathcal O(1)$ factor, yielding an IR cutoff
\begin{equation}
  k_{\min}(t) \;\sim\; \kappa_{\rm IR}\,\frac{H_{\rm eff}(t)}{c_s} \;\propto\; H_{\rm eff}(t)^{\beta},\qquad \beta=1\;\text{(natural)}.
\end{equation}
\paragraph{Signs and monotonicity.} Under the homogeneous secular shrinkage of matter (increasing $F$) and a decelerating expansion (decreasing $H_{\rm eff}$ at late times), one finds $\mathrm d k_{\max}/\mathrm dt>0$ and $\mathrm d k_{\min}/\mathrm dt<0$. Both edges remain positive; what governs $w+1$ is the bandwidth $\Nband=\ln(k_{\max}/k_{\min})$.
\paragraph{Sensitivity to $(\alpha,\beta)$.} With the general parameterization $k_{\max}\propto F^{\alpha}$ and $k_{\min}\propto H_{\rm eff}^{\beta}$ and a near scale-free spectrum, Eq.~(\ref{si:sliding} result) gives
\begin{equation}
  w_X+1 \;\approx\; \frac{\alpha\,\mathrm d\ln F/\mathrm dt - \beta\,\mathrm d\ln H_{\rm eff}/\mathrm dt}{3\,H_{\rm eff}\,\Nband} \;=\; \frac{\alpha - \beta(1+q)}{3\,\Nband},
\end{equation}
where $q:=-\mathrm d\ln H_{\rm eff}/\mathrm d\ln a_{\rm eff}$. Thus departures from $(\alpha,\beta)=(1,1)$ translate linearly into $w+1$ with suppression $\propto 1/\Nband$. Bounds on $w(z)$ constrain only the combination $\alpha-\beta(1+q)$.

\paragraph{Principle of Optimal Bandwidth (constructive).} We formalize the edge evolution by minimizing an information\,–\,complexity functional
\begin{equation}
  \mathcal J(\kmin,\kmax) \;=\; -\,\mathrm{SNR}_{\rm band}(\kmin,\kmax) \; +\; \lambda\,\mathcal C(\kmin,\kmax),\qquad \lambda>0,
\end{equation}
where $\mathrm{SNR}_{\rm band}$ increases with the log\,–\,bandwidth when $S(k)$ is stationary and $\mathcal C$ penalizes bandwidth/edge motion. With $S(k)=A/k$ and a local-unit variance $\sigma^2$ that decreases with bandwidth, one has the schematic scaling
\begin{equation}
  \mathrm{SNR}_{\rm band} \;\propto\; \int_{\kmin}^{\kmax} \frac{dk}{k} \;=\; \ln\!\left(\frac{\kmax}{\kmin}\right),\qquad
  \mathcal C \;=\; \alpha_c\,\ln\kmax \; -\; \beta_c\,\ln\kmin,
\end{equation}
with positive coefficients $\alpha_c,\beta_c$ that encode the asymmetric costs of pushing the UV edge (resolution/complexity) and the IR edge (memory/causal reach). The gradient conditions
\begin{equation}
  \frac{\partial\mathcal J}{\partial\ln\kmax} \;=\; -1 + \lambda\,\alpha_c\;<\;0,\qquad
  \frac{\partial\mathcal J}{\partial\ln\kmin} \;=\; +1 - \lambda\,\beta_c\;>\;0,
\end{equation}
for suitable $\lambda\in(0,\min\{1/\alpha_c,\,1/\beta_c\})$ imply $\mathrm d\kmax/\mathrm dt>0$ and $\mathrm d\kmin/\mathrm dt<0$ along descent. Thus, under broad conditions consistent with $S(k)\propto 1/k$, optimal bandwidth selection drives edges in the observed directions. Quantitative coefficients $(\alpha_c,\beta_c,\lambda)$ can be set by platform\,–\,specific criteria, but the signs are robust.

\paragraph{Proposition (Edge evolution from constrained variational/RG scheme).}
\textbf{Setup.} Define the band functional $\mathcal J[\kmin,\kmax] = -\mathrm{SNR}_{\rm band}(\kmin,\kmax) + \lambda\,\mathcal C(\kmin,\kmax)$ with $\mathrm{SNR}_{\rm band}\propto \int_{\kmin}^{\kmax} S(k)\,\frac{dk}{k}$ (stationary spectrum) and a convex complexity penalty $\mathcal C=\alpha_c \ln\kmax - \beta_c \ln\kmin$ with $\alpha_c,\beta_c>0$. Impose local-unit covariance so that rescalings that preserve co-scaling leave dimensionless predictions invariant.
\textbf{Claim.} For $S(k)=A/k$ and $\lambda\in(0,\min\{1/\alpha_c,1/\beta_c\})$, gradient descent on $\mathcal J$ yields
\begin{equation}
  \frac{d\ln\kmax}{dt} \;>\; 0\,,\qquad \frac{d\ln\kmin}{dt} \;<\; 0\,.
\end{equation}
Moreover, identifying the UV resolution scale with $k_{\max}\propto F$ and the IR causal scale with $k_{\min}\propto H_{\rm eff}$ is the unique assignment (up to $\mathcal O(1)$ factors) that preserves local-unit covariance and the preservation of the principal symbol under coarse-graining.
\textbf{Sketch of proof.} With $S(k)=A/k$, one has $\partial\,\mathrm{SNR}_{\rm band}/\partial\ln\kmax = +1$ and $\partial\,\mathrm{SNR}_{\rm band}/\partial\ln\kmin = -1$. Hence $\partial\mathcal J/\partial\ln\kmax = -1 + \lambda\alpha_c$ and $\partial\mathcal J/\partial\ln\kmin = +1 - \lambda\beta_c$. For $\lambda$ in the stated range, the steepest descent directions enforce $\dot{\ln\kmax}>0$ and $\dot{\ln\kmin}<0$. The identification $\kmax\propto F$ follows from the smallest resolvable physical feature in local units ($k\,L_*\gtrsim\mathcal O(1)$ with $L_*\propto 1/F$), while $\kmin\propto H_{\rm eff}$ follows from the causal/response cone ($\lambda_{\rm IR}\sim c_s/H_{\rm eff}$); any alternative assignments violate either local-unit covariance or the preservation of the principal symbol under coarse-graining. This establishes the sign and anchoring of edge evolution.

\paragraph{Generalizations.} For $S(k)=A\,k^{-1+\epsilon}$ with small $\epsilon$, the bandwidth becomes $\Nband\to \frac{1}{\epsilon}\big[(\kmax/\kmin)^{\epsilon}-1\big]$, and the prefactor $1/\Nband$ acquires an $\mathcal O(\epsilon)$ correction; to leading order the expression above remains valid with $\Nband$ replaced accordingly.

\subsection{Energy--momentum accounting and conservation}\label{si:tmn}
We split the effective stress--energy as
\begin{equation}
  T^{\mu\nu}_{\text{tot}} \;=\; T^{\mu\nu}_{\text{matter}} \; + \; T^{\mu\nu}_{\text{bg}},\qquad T^{\mu\nu}_{\text{bg}}\;\text{encodes the homogeneous, windowed background}.
\end{equation}
By construction (Bianchi identity), Einstein's equations enforce covariant conservation
\begin{equation}
  \nabla_\mu T^{\mu\nu}_{\text{tot}} \;=\; 0.
\end{equation}
At the level of homogeneous cosmology in local units, write the continuity equations with a possible exchange term $Q$:
\begin{align}
  \frac{\mathrm d\rho_m}{\mathrm dt} + 3\,H_{\mathrm eff}\,\rho_m &= +\,Q,\\
  \frac{\mathrm d\rho_{\mathrm eff}}{\mathrm dt} + 3\,H_{\mathrm eff}\,(1+w)\,\rho_{\mathrm eff} &= -\,Q,
\end{align}
so that $\nabla_\mu T^{\mu\nu}_{\text{tot}}=0$ holds for any $Q$. In our construction, the windowed background is 
\begin{equation}
  \rho_{\mathrm eff}(t) \;=\; \int_{k_{\min}(t)}^{k_{\max}(t)} S(k)\,\mathrm dk,
\end{equation}
whose time derivative follows the Leibniz rule
\begin{equation}
  \frac{\mathrm d\rho_{\mathrm eff}}{\mathrm dt} \;=\; S(k_{\max})\,\frac{\mathrm d k_{\max}}{\mathrm dt} \; - \; S(k_{\min})\,\frac{\mathrm d k_{\min}}{\mathrm dt}.
\end{equation}
Matching to an effective fluid with equation of state $w$ by
\begin{equation}
  \frac{\mathrm d\rho_{\mathrm eff}/\mathrm dt}{\rho_{\mathrm eff}} \;=\; -3\,H_{\mathrm eff}\,(1+w)
\end{equation}
identifies $w$ (SI Eq. for $w+1$) and corresponds to choosing $Q=0$ in local units (no explicit energy exchange with clustered matter at leading order). Alternatively, one may keep a small $Q$ to parameterize subleading backreaction; the sum still obeys $\nabla_\mu T^{\mu\nu}_{\text{tot}}=0$. Thus the sliding--window assignment preserves covariant conservation: the homogeneous background evolves as an effective fluid consistent with the windowed integral, while bound systems decouple at leading order in local units.

\subsection{Conservation: internal $\,to\,$kinetic balance and total divergence}\label{si:conservation}
Here we make explicit that scale\,–\,reduction converts internal (free) energy into center\,–\,of\,–\,mass kinetic energy within the same matter stress\,–\,energy tensor, and that total covariant conservation holds without invoking hidden sinks.

\paragraph{Internal vs bulk split (adiabatic reduction).} In the adiabatic regime, the soliton's internal size $\sigma$ relaxes rapidly to $\sigma^*(x)$ set by the local background. The relaxed internal energy is
\begin{equation}
  E_{\rm eq}(x) \;=\; -\,\frac{A_{\eta}^{\,2}}{4A_{\gamma}}\, f\!\big(\tau(x)\big)^2,
\end{equation}
and the conservative force on the center of mass is $\mathbf F= -\nabla E_{\rm eq}(x)$. Writing the matter stress\,–\,energy as
\begin{equation}
  T^{\mu\nu}_{\rm matter} \;=\; T^{\mu\nu}_{\rm bulk} \; +\; T^{\mu\nu}_{\rm int},
\end{equation}
with $T^{\mu\nu}_{\rm bulk}=(\rho+P)u^{\mu}u^{\nu}+Pg^{\mu\nu}$ for the collective coordinate and $T^{\mu\nu}_{\rm int}$ built from $(w,\partial w)$ and the effective potential, one finds the standard mechanical energy balance along the worldline (lab frame for simplicity)
\begin{equation}
  \frac{dK}{dt} \;=\; \mathbf F\!\cdot\!\mathbf v \;=\; -\,\frac{dE_{\rm eq}}{dt},\label{eq:ke-balance}
\end{equation}
so the increase of kinetic energy $K$ equals the decrease of internal free energy $E_{\rm eq}$.

\paragraph{Covariant statement.} In covariant form with 4\,–\,velocity $u^{\mu}$ and proper time $\tau$, Eq.~\eqref{eq:ke-balance} reads
\begin{equation}
  u^{\mu}\nabla_{\mu}\big( \mathcal K + E_{\rm eq}\big) \;=\; 0,\qquad \mathcal K := \tfrac12 m\, u^{\nu}u_{\nu},
\end{equation}
which is equivalent to $\nabla_{\mu}T^{\mu\nu}_{\rm matter}=0$ on solutions of the Euler\,–\,Lagrange equations. Thus the internal $\,to\,$bulk energy transfer is an internal bookkeeping within $T^{\mu\nu}_{\rm matter}$; there is no violation of conservation.

\paragraph{Total conservation with background split.} For cosmological applications it is convenient to write $T^{\mu\nu}_{\rm tot}=T^{\mu\nu}_{\rm matter}+T^{\mu\nu}_{\rm bg}$, with $T^{\mu\nu}_{\rm bg}$ capturing the homogeneous, windowed background defined by Eq.~\eqref{si:sliding}. Total conservation holds,
\begin{equation}
  \nabla_{\mu}T^{\mu\nu}_{\rm tot} \;=\; 0,\label{eq:total-cons}
\end{equation}
and choosing $Q=0$ in the continuity split corresponds to taking the leading internal $\,to\,$bulk exchange to occur within $T^{\mu\nu}_{\rm matter}$, with no leading exchange with the homogeneous background in local units. Subleading exchanges can be parameterised by a small, explicitly modelled $Q$ without affecting Eq.~\eqref{eq:total-cons}.

\subsection{Early-universe suppression via growth of inhomogeneity}\label{si:early-suppression}
The homogeneous drift arises from locally inhomogeneous downshifting averaged over many patches. A minimal coarse-grained model ties the drift amplitude to the variance of inhomogeneity. Let $\Delta_\tau^2$ denote the variance proxy of the noise field (or, equivalently, a smoothed matter variance), and let $D(a_{\rm eff})$ be the linear growth factor (normalized to unity today). For scales well inside the horizon and in the linear regime,
\begin{equation}
  \Delta_\tau^2(a_{\rm eff}) \;\propto\; D(a_{\rm eff})^2,\qquad D(a)\propto a \;\text{(matter era)},\quad D\approx\text{const} \;\text{(radiation era)}.
\end{equation}
Assuming the homogeneous drift rate is proportional to this variance,
\begin{equation}
  \frac{\mathrm d\ln F}{\mathrm dt} \;\equiv\; H_{\rm eff} \;\propto\; \mathcal C\,\Delta_\tau^2(a_{\rm eff}),\qquad \mathcal C=\text{const in local units},
\end{equation}
we obtain the qualitative behavior
\begin{equation}
  H_{\rm eff}(a) \;\propto\; \begin{cases}
    \text{const}\;\times D(a)^2 \;\sim\; a^2, & \text{matter era},\\[0.25em]
    \text{const}\;\times D(a)^2 \;\approx\; \text{constant (tiny)}, & \text{radiation era}.
  \end{cases}
\end{equation}
Thus, during the radiation-dominated epoch (and pre-recombination), growth is suppressed and the variance stays nearly constant at a tiny level, making the homogeneous drift negligible. After recombination, structure growth ($D\uparrow$) turns the drift on smoothly. This explains why the sliding-window component is negligible at early times, preserving BBN and the CMB, and only becomes cosmologically relevant as inhomogeneities grow during the matter era.

\subsection{Small spectral tilt and practical bounds}\label{si:tilt-bound}
Consider $S(k)=A\,k^{-1+\epsilon}$ with $|\epsilon|\ll 1$. The windowed background becomes
\begin{equation}
  \rho_{\mathrm eff}(t) \;=\; \int_{k_{\min}}^{k_{\max}} A\,k^{-1+\epsilon}\,\mathrm dk \;=\; \frac{A}{\epsilon}\,\big(k_{\max}^{\epsilon}-k_{\min}^{\epsilon}\big) \;=\; A\,\Nband_{\epsilon},\qquad \Nband_{\epsilon}:=\frac{k_{\max}^{\epsilon}-k_{\min}^{\epsilon}}{\epsilon}.
\end{equation}
Expanding $k^{\epsilon}=\exp(\epsilon\ln k)$ for small $\epsilon$ gives
\begin{equation}
  \Nband_{\epsilon} \;=\; \ln\!\left(\frac{k_{\max}}{k_{\min}}\right) \;\times\; \Big[\,1 \; +\; \frac{\epsilon}{2}\,\big(\overline{\ln k}\big)_{\rm band} \; +\; \mathcal O(\epsilon^2)\Big],
\end{equation}
where $(\overline{\ln k})_{\rm band}$ is a band-averaged logarithm with weight set by the endpoints. The edge-weighted Leibniz rule becomes
\begin{equation}
  \frac{\mathrm d\rho_{\mathrm eff}}{\mathrm dt} \;=\; A\,\Big[ k_{\max}^{-1+\epsilon}\,\frac{\mathrm d k_{\max}}{\mathrm dt} \; - \; k_{\min}^{-1+\epsilon}\,\frac{\mathrm d k_{\min}}{\mathrm dt}\Big].
\end{equation}
Matching to a smooth component with $w$ yields, to leading order in $\epsilon$,
\begin{equation}
  w+1 \;\approx\; \frac{\alpha\,\mathrm d\ln F/\mathrm dt - \beta\,\mathrm d\ln H_{\rm eff}/\mathrm dt}{3\,H_{\rm eff}\,\Nband_{\epsilon}} \;\approx\; \frac{\alpha - \beta(1+q)}{3\,\Nband}\,\Big[\,1\; -\; \frac{\epsilon}{2}\,\big(\overline{\ln k}\big)_{\rm band} \; +\; \mathcal O(\epsilon^2)\Big],
\end{equation}
with $\Nband=\ln(k_{\max}/k_{\min})$. Thus, small tilts correct $w+1$ only at order $\mathcal O(\epsilon/\Nband)$ after accounting for the dominant $1/\Nband$ suppression.

\paragraph{Practical bound.} Current low-$z$ constraints give $|w+1| \lesssim 0.02$ for smooth drifts. Taking a representative $\Nband\sim 10$--15 and $|\alpha-\beta(1+q)|\lesssim\mathcal O(1)$, the relative correction from a tilt satisfies
\begin{equation}
  \left|\frac{\Delta(w+1)}{w+1}\right| \;\sim\; \frac{|\epsilon|}{2}\,\big|(\overline{\ln k})_{\rm band}\big| \;\lesssim\; \mathcal O(|\epsilon|),
\end{equation}
so $|\epsilon|\lesssim 0.05$ keeps tilt-induced changes comfortably below a few percent, well within present bounds when combined with the $1/\Nband$ factor. More stringent bounds follow from a joint fit to $w(z)$; the key point is that large bandwidth renders the model insensitive to modest tilts.

\subsection{Two-scale Buchert backreaction: estimate and example}\label{si:backreaction}
We sketch a Buchert-style estimate of the kinematical backreaction term
\begin{equation}
  Q_\mathcal{D} \;=\; \frac{2}{3}\Big(\langle \theta^2 \rangle_\mathcal{D} - \langle \theta \rangle_\mathcal{D}^{\,2}\Big) \; - \; 2\,\langle \sigma^2 \rangle_\mathcal{D},
\end{equation}
where $\theta$ is the local expansion scalar, $\sigma^2$ the shear invariant, and $\langle\cdot\rangle_\mathcal{D}$ denotes a domain average. In a two-scale (wall/void) model with void fraction $f_v$ and wall fraction $f_w=1-f_v$, with local Hubble rates $H_v$ and $H_w$, one finds the variance contribution (neglecting shear for a conservative upper bound)
\begin{equation}
  Q_\mathcal{D} \;\approx\; 6\, f_v\,(1-f_v)\,\big(H_v - H_w\big)^2 \; - \; 2\,\langle \sigma^2 \rangle_\mathcal{D} \,.
\end{equation}
Normalizing to the domain Hubble rate $H$ (e.g., $H=\langle\theta\rangle/3$) and defining $\Delta_H := (H_v-H_w)/H$, we obtain
\begin{equation}
  \frac{Q_\mathcal{D}}{H^2} \;\lesssim\; 6\, f_v\,(1-f_v)\,\Delta_H^{\,2} \,.
\end{equation}
\paragraph{Numeric example.} Take representative late-time values $f_v\sim 0.6$--0.8 and a modest expansion-rate contrast $\Delta_H\sim 0.03$ (a few percent). Then
\begin{equation}
  \frac{Q_\mathcal{D}}{H^2} \;\lesssim\; 6\times 0.6\times0.4\times(0.03)^2 \;\approx\; 1.3\times 10^{-3}\,.
\end{equation}
Including $-2\langle\sigma^2\rangle_\mathcal{D}$ further reduces $Q_\mathcal{D}$. Thus, for realistic void fractions and small wall/void expansion contrasts, one expects $|Q_\mathcal{D}|/H^2\ll 10^{-2}$, typically $\mathcal O(10^{-3})$ or below, insufficient to mimic a dark-energy component or to spoil the homogeneous drift derived from the sliding window.
\paragraph{Framework note.} In the present local-unit/co-scaling interpretation, domain-dependent expansion in absolute units maps to small modulations around a common $H_{\rm eff}$ in local units, which further suppresses effective backreaction entering the homogeneous bookkeeping. A full treatment with shear retained yields even tighter suppressions; the order-of-magnitude conclusion $|Q_\mathcal{D}|/H^2\ll 1$ is robust for small $\Delta_H$.

\subsection{Bounds on varying constants and fifth-force mapping}\label{si:varying-constants}
We parameterize a representative drift of a dimensionful coupling through its dependence on the windowed background (in local units). For the fine-structure constant as example,
\begin{equation}
  \alpha(t) \;=\; \alpha_0\,\Big[\,1 + \zeta\, f\big(\rho_{N,\mathrm eff}(t)\big)\,\Big],\qquad \Rightarrow\qquad \frac{\mathrm d\ln\alpha}{\mathrm dt} \;=\; \zeta\, f'\,\frac{\mathrm d\rho_{N,\mathrm eff}}{\mathrm dt},
\end{equation}
with $\rho_{N,\mathrm eff}=\int_{k_{\min}}^{k_{\max}} S(k)\,\mathrm dk$. Using the Leibniz rule and $S(k)\propto 1/k$,
\begin{equation}
  \left|\frac{\mathrm d\ln\alpha}{\mathrm dt}\right| \;\sim\; \zeta\, \frac{1}{\Nband}\,\Big|\,\frac{\mathrm d\ln k_{\max}}{\mathrm dt} - \frac{\mathrm d\ln k_{\min}}{\mathrm dt}\,\Big| \;\sim\; \zeta\,\mathcal O(10^{-3})\,H_0,
\end{equation}
where the last estimate uses representative late-time values (large $\Nband$ and small fractional edge drifts). Atomic clock bounds give $|\dot\alpha/\alpha| \lesssim 10^{-17}\,\mathrm{yr}^{-1} \approx 3\times10^{-27}\,\mathrm s^{-1}$, while $H_0\approx 2.3\times10^{-18}\,\mathrm s^{-1}$. Hence
\begin{equation}
  \zeta \;\lesssim\; \mathcal O(10^{-2})\quad\text{(conservative)},
\end{equation}
consistent with the order-of-magnitude used in the main text (e.g., $|\Delta\alpha/\alpha|\lesssim 7\times10^{-7}$ over a Gyr for $\zeta\lesssim 10^{-2}$).

\paragraph{WEP and fifth-force constraints.} The universal (metric) coupling of the phase sector is composition-independent and respects the weak equivalence principle. Possible fifth forces arise only from (i) leakage of the gapped amplitude sector (Yukawa with range $\ell$) or (ii) explicit shift-breaking, producing a light scalar with non-derivative couplings. The former is exponentially suppressed for Solar-System scales if $\ell$ is microscopic. The latter yields a Yukawa correction with strength $\alpha_5\propto \epsilon^2$; planetary ephemerides and LLR imply $\alpha_5\lesssim 10^{-10}$--$10^{-12}$ over AU ranges. In this framework, the leading, derivative, universal coupling does not generate static $1/r$ forces between conserved sources; only tiny shift-breaking coefficients $\epsilon$ are bounded by WEP/Yukawa tests, naturally consistent with $\zeta\lesssim 10^{-2}$ above.

\subsection{Physical justifications for edge evolution}
We provide three complementary rationales for why the window edges drift monotonically and homogeneously.
\paragraph{Sign of the edge drift (from local to global).} Inhomogeneous, local downshifting accompanies matter clustering: solitons relax toward smaller equilibrium sizes in higher-noise regions, and this tendency is universal across patches. Averaged over many regions, the absolute equilibrium scale $L_*(t)$ of bound systems shrinks monotonically. In local units this implies $\kmax\sim 1/L_*\propto F$ increases with time, while the causal IR edge follows the effective expansion rate, $\kmin\propto \Heff$, which decreases as the Universe ages. Thus the signs are fixed: $\mathrm d\kmax/\mathrm dt>0$ and $\mathrm d\kmin/\mathrm dt<0$. Both edges remain strictly positive wavenumbers ($\kmin,\kmax>0$); here the inequalities denote time derivatives, not the edge values.

\paragraph{Kinematic anchoring.} The edges are tied to physical scales in local units:
\begin{itemize}[leftmargin=*]
  \item \textbf{IR edge:} $\kmin \sim a\Heff/\cs$ (inverse causal horizon in comoving coordinates). As cosmic expansion decelerates, $\Heff$ decreases monotonically, driving $\kmin$ down.
  \item \textbf{UV edge:} $\kmax \sim 1/L_*$ where $L_*$ is the characteristic scale of stabilized solitons. In local units, $L_* \propto 1/F$, so $\kmax \propto F$. As local rulers shrink ($F$ increases), $\kmax$ increases.
\end{itemize}
Both edges evolve smoothly and homogeneously, driven by cosmic kinematics rather than stochastic processes.

\paragraph{Information-theoretic bandwidth selection.} An optimal estimator minimizes an information cost $\mathcal J(\kmin,\kmax) = -\text{SNR}_{\text{band}} + \lambda \cdot \text{complexity}$. For a stationary $S(k)$ with declining local-unit variance, gradient descent yields $\partial\mathcal J/\partial\ln\kmax < 0$ and $\partial\mathcal J/\partial\ln\kmin > 0$, driving both edges toward lower absolute $k$ (monotone drift with the observed sign).

\paragraph{Coarse-graining/RG perspective.} The effective theory integrates out modes above $\kmax$ (set by local coherence) and below $\kmin$ (set by finite memory). As the background evolves, these cutoffs co-evolve to maintain a consistent coarse-graining, naturally producing the $\kmax \propto F$, $\kmin \propto \Heff$ scalings.

\section{Early-time consistency constraints}\label{si:early-consistency}
We collect constraints ensuring the window-driven component does not spoil early-universe physics.
\begin{itemize}[leftmargin=*]
  \item \textbf{BBN:} Require that the fractional contribution of the window-driven component at $z\sim10^9$ is negligible. In practice, impose $|w_X+1|\ll1$ and a near-constant $F$ at early times, ensuring $\dot F/F\approx0$ in the radiation era.
  \item \textbf{CMB acoustic peaks:} Bound late-time deviations so that the angular diameter distance to last scattering and the sound horizon are preserved within Planck uncertainties; this restricts the integrated history of $w_X(z)$ and hence the allowed $(\alpha,\beta)$ drifts.
  \item \textbf{Early dark energy fraction:} The crude estimate $f_{\rm EDE}(z_{\ast})\lesssim10^{-3}$ suffices here, consistent with the negligible drift assumptions in the paper.
\end{itemize}

\section{Galactic diagnostics and modeling recipes}\label{si:dm-diagnostics}
We provide practical recipes for cases flagged in the main text.
\subsection*{Anisotropy and dispersion gradients}
Use the general relation
\begin{equation}
  \alpha(r) \equiv -\frac{\mathrm d\ln\rho}{\mathrm d\ln r} = \frac{v_c^2}{\sigma_r^2} + \frac{\mathrm d\ln\sigma_r^2}{\mathrm d\ln r} + 2\,\beta(r)
\end{equation}
to propagate uncertainties from $\beta(r)$ and $\sigma_r(r)$ into mass profiles. When fitting Jeans models, include a linear-in-$\ln r$ term for $\sigma_r^2$ locally and a weakly informative prior on $\beta(r)$.

\subsection*{External-field/tidal truncation}
Model truncation by imposing a taper radius $r_t$ (from tides or external-field effects) beyond which the isothermal envelope transitions smoothly to a steeper decline; use penalized splines or an error-function taper. Predict a recovery toward the $1/r$ tail beyond $r_t$ where the Coulombic window reopens.

\subsection{Kinetic derivation of $\sigma\approx\mathrm{const}$ in local units}\label{si:kinetic-sigma}
We sketch a Fokker\,–\,Planck derivation showing that a stationary, scale\,–\,invariant bath yields an approximately radius\,–\,independent one\,–\,dimensional dispersion in local units within the Coulombic window.

\paragraph{Setup.} Consider collisionless tracers in a slowly varying, spherically symmetric potential $\Phi(r)$ with weak gradients ($R_{\rm cl}\ll r \ll \ell$). Let $f(r,\mathbf v,t)$ obey a Fokker\,–\,Planck equation with drift from $\Phi$ and isotropic velocity\,–\,space diffusion $\mathcal D$ induced by the phase bath in local units:
\begin{equation}
  \partial_t f + v_r\,\partial_r f - (\partial_r\Phi)\,\partial_{v_r} f \;=\; \partial_{v_i}\big[ A_i f + \tfrac12 \partial_{v_j}(B_{ij} f)\big],\qquad B_{ij}=2\mathcal D\,\delta_{ij}.
\end{equation}
Stationarity and weak anisotropy imply $A_i\approx 0$ and $B_{ij}\approx 2\mathcal D\,\delta_{ij}$. In local units the bath spectrum $S(k)\propto 1/k$ produces a velocity\,–\,space diffusion coefficient $\mathcal D\propto$ (band\,–\,averaged power) that is radially slowly varying within the window.

\paragraph{Moment equation.} Multiply by $v_r^2$ and integrate over velocities to obtain the stationary second\,–\,moment equation in spherical symmetry,
\begin{equation}
  \frac{1}{r^2}\,\frac{\mathrm d}{\mathrm dr}\big( r^2\,\rho\,\langle v_r^3\rangle \big) \;=\; -\,\rho\,\langle v_r\,\partial_r\Phi\rangle \; +\; 2\rho\,\mathcal D,\label{eq:second-moment}
\end{equation}
where $\rho(r)=\int f\,d^3v$. In the quasi\,–\,isothermal regime ($\langle v_r^3\rangle\approx 0$ to leading order; weak skewness) and for slowly varying $\mathcal D$, Eq.~\eqref{eq:second-moment} reduces locally to
\begin{equation}
  \rho\,\sigma_r^2\,\frac{\mathrm d\ln\rho}{\mathrm dr} \;=\; -\,\rho\,\frac{\mathrm d\Phi}{\mathrm dr} \; +\; 2\rho\,\mathcal D,\qquad \sigma_r^2:=\langle v_r^2\rangle.
\end{equation}
Comparing with the isotropic Jeans relation $\sigma^2\,\mathrm d\ln\rho/\mathrm dr = -\mathrm d\Phi/\mathrm dr$ shows that a small, slowly varying $\mathcal D$ acts as an additive, radius\,–\,smooth source that stabilizes $\sigma_r$ against radial drift in local units. Within the Coulombic window this yields $\sigma_r(r)\approx \mathrm{const}$ up to $\mathcal O(\mathcal D/|\partial_r\Phi|)$ corrections, recovering the isothermal envelope.

\paragraph{Scale\,–\,invariant bath and slow variation.} For $S(k)\propto 1/k$ one has equal power per log\,–\,band; with edges $k_{\min}\propto H_{\rm eff}$ and $k_{\max}\propto F$, the band\,–\,averaged power in local units is homogeneous to leading order (Sec.~\ref{si:sliding}), implying $\mathcal D\approx \mathrm{const}$ across the halo. Hence the dispersion is flat to leading order, with gentle tilts set by edge drift and anisotropy.

\paragraph{Anisotropy and corrections.} Allowing mild anisotropy and a small radial tilt in $\mathcal D$ yields the general correction
\begin{equation}
  \alpha(r) = -\frac{\mathrm d\ln\rho}{\mathrm d\ln r} = \frac{v_c^2}{\sigma_r^2} + \frac{\mathrm d\ln\sigma_r^2}{\mathrm d\ln r} + 2\beta(r),
\end{equation}
with $\mathrm d\ln\sigma_r^2/\mathrm d\ln r=\mathcal O(\mathcal D/|\partial_r\Phi|)$, consistent with the main text’s diagnostics.

\subsection{Screening-length bands and core/truncation mapping}\label{si:ell-bands}
We map microparameters $(\alpha_{\rm grad},\beta_{\rm pot})$ to observable halo scales via $\ell=\sqrt{\alpha_{\rm grad}/(2\beta_{\rm pot})}$ and summarize regimes.

\paragraph{Stability window.} The amplitude sector requires $\beta_{\rm pot}>0$, $\gamma>0$ for a stable $w_*$. For clustered sources with size $R_{\rm cl}$ and observation radii $r$ in the Coulombic window $R_{\rm cl}\ll r\ll \ell$, the far field is approximately $1/r$. Choose bands where
\begin{equation}
  \ell \;\in\; [\ell_{\min},\ell_{\max}],\qquad \ell_{\min}\gtrsim 5\,R_{\rm cl},\quad \ell_{\max}\gg r_{\rm out},
\end{equation}
with $r_{\rm out}$ the outer radii used to infer flat curves. This yields a corresponding band in $(\alpha_{\rm grad},\beta_{\rm pot})$:
\begin{equation}
  \frac{\alpha_{\rm grad}}{2\beta_{\rm pot}}\;\in\; [\ell_{\min}^2,\,\ell_{\max}^2].
\end{equation}

\paragraph{Core mapping.} When $r\lesssim \ell$, Helmholtz corrections produce a pseudo\,–\,isothermal core with $r_c\sim \mathcal O(\ell)$. Hence observed core radii constrain $\ell$ and thus the ratio $\alpha_{\rm grad}/\beta_{\rm pot}$. Reported dwarf cores $r_c\sim\,$kpc imply $\ell\sim\,$kpc, whereas systems with cuspy inner regions are consistent with $\ell\ll$ inner resolution.

\paragraph{Environment/truncation.} In strong host potentials or tides, the effective window narrows and the isothermal envelope truncates at $r_t$; observationally, mass discrepancy drops inside $r_t$ and recovers the $1/r$ tail outside. In this framework, $r_t$ marks where the Coulombic window fails (either $r\sim R_{\rm cl}$ by tides or $r\sim \ell$ by finite range). Thus $(\alpha_{\rm grad},\beta_{\rm pot})$ bands and environment jointly predict $(r_c, r_t)$.

\paragraph{Practical calibration.} Given $(r_c, r_t)$ cohorts, infer $\ell$ per system and build posteriors for $\alpha_{\rm grad}/\beta_{\rm pot}$. Cross\,–\,check with lensing/dispersion profiles and the kinetic diffusion scale from Sec.~\ref{si:kinetic-sigma}. Simulation guidance: sample $(\alpha_{\rm grad},\beta_{\rm pot})$ grids, measure far\,–\,field tails and core sizes, and fit the pseudo\,–\,isothermal mapping.

\section{Predictions and data pipelines}\label{si:predictions}
\begin{itemize}[leftmargin=*]
  \item \textbf{Halo dispersion vs flat speed:} Test $v_c^2\simeq2\sigma^2$ using outer tracers (HI/GCs) and robust dispersion estimators; flag systems with strong anisotropy or truncation as separate cohorts.\newline
  \emph{Pipeline:} (i) Select galaxies with extended HI or GC kinematics (e.g., SPARC, LITTLE THINGS). (ii) Fit outer $v_c$ plateaus and outer $\sigma$ with biweight/Huber estimators; (iii) exclude radii with $r\lesssim r_c$ (cores) or $r\gtrsim \ell$ (window failure); (iv) regress $v_c^2$ on $2\sigma^2$ with hierarchical errors; (v) report intrinsic scatter.
  \item \textbf{Environment correlation:} Bin mass-discrepancy metrics by host potential or group-centric distance; expect lower discrepancies inside truncation radii with recovery beyond.\newline
  \emph{Pipeline:} (i) Cross-match with group catalogs; (ii) define tidal/truncation proxy (e.g., $r/r_{\rm host}$, external $g_{\rm ext}$); (iii) estimate discrepancy $\Delta M(r)$ from rotation curves/dispersion; (iv) stack profiles by environment bin; (v) test for broken symmetry/truncation and recovery of $1/r$ tail beyond $r_t$.
  \item \textbf{Anisotropy-aware Jeans fits:} Use Eq.~(\ref{eq:alpha-local}) and allow $\beta(r)$ and slow $\mathrm d\ln\sigma_r^2/\mathrm d\ln r$ when fitting “low-DM” systems.\newline
  \emph{Pipeline:} (i) Fit spherical Jeans with flexible $\beta(r)$ (e.g., Baes–van Hese) and a linear-in-$\ln r$ term for $\sigma_r^2$; (ii) impose weak priors on $\beta$ and slope; (iii) compare enclosed mass to isotropic fits; (iv) check for approach to $\rho\propto r^{-2}$ at larger radii; (v) flag systems with strong anisotropy/gradients as outliers to the simple isothermal prediction.
\end{itemize}

\section{Hubble tension exploration}\label{si:hubble}
We sketch minimal parameterizations that could shift late-time $H_0$ inferences without violating early-universe constraints.

\paragraph{Localized late-time drift ansatz.} Consider smooth transitions:
\begin{align}
  \alpha(z) &= 1 + \delta\alpha \cdot \frac{1}{1 + e^{(z-z_0)/\Delta z}}, \\
  \beta(z) &= 1 + \delta\beta \cdot \frac{1}{1 + e^{(z-z_0)/\Delta z}},
\end{align}
with $z_0 \approx 0.5$–$1$, $\Delta z \approx 0.2$–$0.5$, and $|\delta\alpha|, |\delta\beta| \lesssim 0.1$. This preserves $\alpha(z \gg z_0) \approx \beta(z \gg z_0) \approx 1$ (early-time physics unchanged) while allowing enhanced late-time dark energy.

\paragraph{Constraints and sensitivity.} Early-time preservation requires that angular-diameter distances to last scattering and BBN yields remain unchanged within errors; in practice we demand $|\Delta d_A(z_*)| \lesssim 1\%$ and $|\Delta Y_p| \lesssim 0.1\%$. Late-time enhancement can be achieved by modestly boosting $w_{\rm eff}(z\lesssim 1)$ to target, e.g., $\Delta H_0 \approx +6$ km/s/Mpc. Throughout, a large bandwidth $\Nband\gg 1$ must be maintained so the small-$|w+1|$ expansion remains valid and smooth.

\paragraph{Preliminary bounds.} For the logistic ansatz above with $z_0 = 0.7$, $\Delta z = 0.3$, rough Fisher estimates suggest $|\delta\alpha|, |\delta\beta| \lesssim 0.05$ to remain consistent with Planck+SN Ia constraints while providing $\mathcal O(10\%)$ shifts in late-time $H_{\rm eff}$.

\subsection{Bounds on $w(z)$, early dark energy, and distances}\label{si:w-bounds}
Using the sliding-window relation
\begin{equation}
  w + 1 \;\approx\; \frac{\alpha - \beta(1+q)}{3\,\Nband},\qquad \Nband=\ln\!\left(\frac{k_{\max}}{k_{\min}}\right),
\end{equation}
we can obtain compact bounds under mild assumptions.

First, take the natural choice $\alpha=\beta=1$ and a representative late-time bandwidth $\Nband\sim 10$--15. When the window edges drift slowly, this yields $|w+1|\lesssim \mathcal O(10^{-3})$ at $z\lesssim 1$, compatible with current constraints on smooth departures from $w=-1$. The large $\Nband$ suppresses deviations, making the effect both small and smooth.

Second, at early times in the radiation era we assume $\dot F/F\approx 0$ so that the homogeneous co-scaling drift is effectively frozen. In this limit $w\to -1$ and the early-dark-energy fraction
\begin{equation}
  f_{\rm EDE}(z_*) \;\equiv\; \frac{\rho_X}{\rho_{\rm tot}}\Big|_{z_*}
\end{equation}
becomes negligible at recombination ($z_*\simeq 1100$); for the fiducial drift this estimate is $f_{\rm EDE}\sim \mathcal O(10^{-9})$, preserving the CMB and BBN physics assumed in the main text.

Third, distance measures remain consistent with Planck given the same suppression. With $|w+1|\ll 1$ and large $\Nband$, the integrated shift in angular diameter distance to last scattering, $\Delta d_A(z_*)/d_A(z_*)$, is $\ll 1\%$ for the fiducial drift. If one allows localized late-time deviations by taking small $\delta\alpha,\delta\beta$ confined to $z\lesssim 1$ (as in Sec.~\ref{si:hubble}), the combined constraints can be satisfied while preserving $|\Delta d_A(z_*)|\lesssim 1\%$ and keeping the low-redshift Hubble history flexible at the $\mathcal O(10\%)$ level.

These estimates support the main text’s claim that the sliding-window component yields tiny, smooth departures consistent with present data, and that controlled late-time variations can be accommodated without spoiling early-universe distances.

\paragraph{Quantitative illustration.} Writing $E(z)=H(z)/H_0$ and splitting $E^2=\Omega_m(1+z)^3+\Omega_X\exp\{3\int_0^z [1+w(z')]\,\mathrm d\ln(1+z')\}$, a small $\delta w(z)$ produces
\begin{equation}
  \delta\ln E(z) \;\approx\; \tfrac32\,\Omega_X(z)\,\int_0^z \delta w(z')\,\mathrm d\ln(1+z'),\qquad \Omega_X(z):=\frac{\rho_X}{\rho_{\rm tot}}.
\end{equation}
With $\delta w\sim (\alpha-\beta(1+q))/(3\Nband)$ and large $\Nband$, the integral is suppressed by both $\Omega_X(z)$ at high $z$ and the $1/\Nband$ factor at low $z$. The fractional distance shift then satisfies
\begin{equation}
  \delta\ln d_A(z) \;=\; -\int_0^z \frac{\delta E(z')}{E(z')}\,\frac{\mathrm d z'}{1+z'} \;\ll\; 1\%\ \text{for the fiducial drift},
\end{equation}
consistent with the bounds quoted above.

\section{Links to ICG/Gravity proofs}\label{si:links}
For full derivations of stability, induced Lorentz cone, screened superposition, and metric universality, see the Relativity paper (Secs. 2–7) and the ICG paper (Secs. S1–S10). This SI only summarizes the parts used in the dark-sector narrative without repeating proofs.

\section{Gauge equivalence: co-scaling units vs FLRW}\label{si:gauge-equivalence}
We prove that expressing cosmological predictions in local co-scaling units via a conformal map leaves redshift and distance observables invariant.

\subsection*{Setup}
Let $(\mathcal M,g_{\mu\nu})$ be an FLRW spacetime with scale factor $a(t)$. Define the co-scaling (local-unit) metric
\begin{equation}
  \hat g_{\mu\nu}(x) \;=\; F(t)^2\, g_{\mu\nu}(x),\qquad F(t)>0.
\end{equation}
Let $u^{\mu}$ be the 4-velocity of comoving observers and $k^{\mu}$ the null wavevector along a light ray. Define orthonormal tetrads $e^{a}_{\ \mu}$ for $g_{\mu\nu}$ and $\hat e^{a}_{\ \mu}=F\, e^{a}_{\ \mu}$ for $\hat g_{\mu\nu}$ so that local frequencies are $\omega = k_{\mu} u^{\mu}$ and $\hat\omega = \hat k_{\mu} \hat u^{\mu}$.

\subsection*{Null geodesics and redshift}
Conformal rescalings preserve null geodesics up to reparametrization. With co-scaling tetrads,
\begin{equation}
  \hat u^{\mu} = F^{-1} u^{\mu},\qquad \hat k_{\mu} = F^{2} k_{\mu},\qquad \Rightarrow\qquad \hat\omega = (\hat k_{\mu}\hat u^{\mu}) = F\, (k_{\mu}u^{\mu}) = F\, \omega.
\end{equation}
Hence the observable redshift
\begin{equation}
  1+z \;=\; \frac{\omega_{\rm em}}{\omega_{\rm obs}} \;=\; \frac{F_{\rm obs}\,\hat\omega_{\rm em}}{F_{\rm em}\,\hat\omega_{\rm obs}} \;=\; \frac{\hat\omega_{\rm em}}{\hat\omega_{\rm obs}}\, \frac{F_{\rm obs}}{F_{\rm em}}
\end{equation}
coincides with the standard FLRW expression because the same ratio appears in the evolution of $\hat\omega$ along the ray due to the conformal time reparametrization; explicitly, choosing $F=a$ recovers $1+z=a_{\rm obs}/a_{\rm em}$, while general $F$ produces the same observable through compensating factors in the locally measured frequency.

\subsection*{Distances and Etherington’s relation}
Angular diameter distance $D_A$ and luminosity distance $D_L$ transform under a global conformal rescaling as $\hat D_A=F\,D_A$ and $\hat D_L=F\,D_L$. Since $1+z$ is invariant, Etherington’s relation holds in both gauges:
\begin{equation}
  \hat D_L \;=\; (1+z)^2 \hat D_A \quad\Longleftrightarrow\quad D_L \;=\; (1+z)^2 D_A.
\end{equation}
Thus all standard cosmological distance relations are gauge-equivalent under the co-scaling map with locally co-scaled tetrads.

\subsection*{Worked example}
Consider a monochromatic emitter at $t_{\rm em}$ and an observer at $t_{\rm obs}$ on comoving worldlines. In FLRW,
\begin{equation}
  1+z \;=\; \frac{a(t_{\rm obs})}{a(t_{\rm em})},\qquad D_L=(1+z)^2 D_A.
\end{equation}
In the co-scaling gauge with $\hat g=F^2 g$ and $\hat e=F e$, the measured frequencies satisfy $\hat\omega=F\omega$, and the ratio yields
\begin{equation}
  1+z \;=\; \frac{\omega_{\rm em}}{\omega_{\rm obs}} \;=\; \frac{\hat\omega_{\rm em}/F_{\rm em}}{\hat\omega_{\rm obs}/F_{\rm obs}} \;=\; \frac{\hat\omega_{\rm em}}{\hat\omega_{\rm obs}}\, \frac{F_{\rm obs}}{F_{\rm em}},
\end{equation}
which equals the FLRW result once the conformal time reparametrization along the null geodesic is accounted for (the $F$ ratio cancels the change in $\hat\omega$). Hence both descriptions give identical observable $z$, $D_A$, $D_L$.

\section{Lensing and PPN cross-checks}\label{si:lensing-ppn}
In the weak-field regime we use
\begin{equation}
  ds^2 = -\Big(1+\tfrac{2\Phi}{c_s^2}\Big) c_s^2 dt^2 + \Big(1-\tfrac{2\Phi}{c_s^2}\Big) d\mathbf x^2,
\end{equation}
which has PPN parameters $\gamma=\beta=1$ at leading order. Therefore light deflection, Shapiro delay, and time-delay distances take the standard GR form with the replacement $c\to c_s$ in local units. For a singular isothermal sphere with $\rho=\sigma^2/(2\pi G r^2)$ the projected surface density is $\Sigma(R)=\sigma^2/(2G R)$ and the reduced deflection angle is constant, reproducing the phenomenology of isothermal-like halos in strong and weak lensing. Kinematics and lensing both probe the same potential $\Phi$, so the flat speed relation $v_c^2=2\sigma^2$ is consistent with the lensing normalization, up to known mass-sheet and anisotropy degeneracies discussed in the main text.

\section{Simulation plan and calibration}\label{si:sim-plan}
We outline a minimal simulation program to calibrate micro-to-macro coefficients and cross-validate analytic mappings.

First, calibrate the screening and core mapping by sampling $(\alpha_{\rm grad},\beta_{\rm pot})$ on near-regular graphs that implement the volume-normalized free energy. Relax isolated and composite sources, measure far-field tails and inner profiles, and fit the relation $\ell=\sqrt{\alpha_{\rm grad}/(2\beta_{\rm pot})}$ against recovered core radii $r_c$ and the Coulombic window extents. This anchors the $\ell$-band constraints used in Sec.~\ref{si:ell-bands}.

Second, verify the kinetic dispersion result by driving tracers in the stationary phase bath with $S(k)\propto 1/k$, estimating the velocity-space diffusion coefficient $\mathcal D$ in local units, and confirming the near-constancy of $\sigma_r(r)$ across the window (Sec.~\ref{si:kinetic-sigma}).

Third, calibrate the lensing normalization by computing deflection angles from the relaxed potentials on the same graphs and comparing to the isothermal predictions (Sec.~\ref{si:lensing-ppn}), checking consistency with the kinematic $v_c^2=2\sigma^2$ relation.

Finally, cross-check the sliding-window bookkeeping by coarse-graining phase fluctuations to estimate $\rho_{N,\rm eff}$ and its edge derivatives, confirming the suppression of $|w+1|$ by the large bandwidth $\Nband$ (Secs.~\ref{si:sliding}, \ref{si:w-bounds}).

\paragraph{Maximum-entropy derivation of $S(k)\propto 1/k$ (theorem).}
\textbf{Theorem (Band-entropy maximizer).} Among stationary, positive spectra $S(k)$ on a finite window $[k_{\min},k_{\max}]$ with fixed integrated band power and scale-invariance of information (uniform content per logarithmic interval), the entropy maximizer has the form
\begin{equation}
  S(k) \;=\; \frac{A}{k}\,,\qquad A>0\,.
\end{equation}
\textbf{Sketch of proof.} Let $u:=\ln k$ and define the band measure $d\mu(u):=du$ so that equal log\,--\,bands have equal measure. Consider the non\,--\,negative spectral density in $u$\,--\,space, $\sigma(u):=k\,S(k)$ (so that $\sigma\,du = S\,dk$). Impose: (i) stationarity and positivity, (ii) fixed total band power $\int_{u_{\min}}^{u_{\max}} \sigma(u)\,du = P_0$, and (iii) maximal Shannon entropy $\mathcal H[\sigma]= -\int \sigma(u)\,\ln\big(\sigma(u)/\sigma_0\big)\,du$ with a reference $\sigma_0$ (units). The variational problem with a single normalization constraint yields $\delta\{\mathcal H-\lambda\int\sigma\}=0\Rightarrow \ln(\sigma/\sigma_0) = -1-\lambda$, i.e., $\sigma(u)=\text{const}$ on $[u_{\min},u_{\max}]$. Transforming back gives $k\,S(k)=\text{const}$, hence $S(k)\propto 1/k$. This is precisely Jeffreys' prior for scale, reflecting the assumption that no logarithmic band is preferred.
\medskip
\textbf{Remarks.} (1) Adding mild constraints (e.g., fixed first moment of $u$) yields $S(k)=A\,k^{-1+\epsilon}$ with small tilt $\epsilon$, consistent with the small\,--\,tilt analysis in Sec.~\ref{si:tilt-bound}. (2) Under coarse\,--\,graining that preserves the Lorentzian principal symbol and local\,--\,unit invariance, the fixed point with uniform $\sigma(u)$ is attractive, providing an RG interpretation for $S(k)\propto 1/k$. (3) The finite window ensures integrability; all window dependence enters only via band\,--\,integrated quantities such as $\mathcal N=\ln(k_{\max}/k_{\min})$.

\paragraph{Multiplicative cascade to the scale-free spectrum (construction).}
\textbf{Construction.} Partition the window $[k_{\min},k_{\max}]$ into $N$ logarithmically equal bands (bins in $u=\ln k$). Start from any positive initial band weights $\{w_i^{(0)}\}_{i=1}^N$ with fixed total $\sum_i w_i^{(0)}=P_0$. Iterate the following multiplicative rebalancing map that enforces (i) stationarity of total band power and (ii) equalization of log-band content:
\begin{equation}
  w_i^{(n+1)} \;=\; w_i^{(n)}\, \exp\!\Big(-\eta\,\big[\ln w_i^{(n)} - \overline{\ln w}^{(n)}\big]\Big)\,,\qquad \overline{\ln w}^{(n)} := \frac{1}{N}\sum_{j=1}^N \ln w_j^{(n)}\,,\ \eta\in(0,2)\,.
\end{equation}
Normalize after each step to keep $\sum_i w_i^{(n)}=P_0$. Then $w_i^{(n)}\to P_0/N$ as $n\to\infty$ for any strictly positive initialization. Interpreting $w_i$ as $\int_{\text{band }i} k\,S(k)\,d(\ln k)$, the continuum limit yields $k\,S(k)=\text{const}$, hence $S(k)\propto 1/k$.
\medskip
\textbf{Convergence conditions.} (1) Strict positivity of initial band weights (no empty bands); (2) bounded step size $\eta\in(0,2)$ (gradient descent on the convex functional $\sum_i (\ln w_i - \overline{\ln w})^2$); (3) fixed window edges during the iteration. Under slow edge drift, the rebalancing remains adiabatically valid provided the drift time scale $T_{\rm drift}\gg$ the mixing time $T_{\rm mix}\sim \mathcal O(1/\eta)$, so the spectrum tracks the $1/k$ profile up to $\mathcal O}(T_{\rm mix}/T_{\rm drift})$ corrections.
\medskip
\textbf{Remark.} Adding a small deterministic tilt component (e.g., due to weak dispersion) corresponds to a bias term in the update that yields a nonzero fixed tilt $\epsilon=\mathcal O(\text{bias})$, reproducing $S(k)=A\,k^{-1+\epsilon}$ with $|\epsilon|\ll 1$ (consistent with Sec.~\ref{si:tilt-bound}).

\end{document}


