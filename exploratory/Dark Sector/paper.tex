% !TeX program = pdflatex
\documentclass[11pt]{article}
\usepackage[a4paper,margin=1in]{geometry}
\usepackage{amsmath,amssymb,amsfonts}
\usepackage{graphicx}
\usepackage{bm}
\usepackage{mathtools}
\usepackage{microtype}
\usepackage{enumitem}
\usepackage{hyperref}
\usepackage{xr-hyper}
\usepackage{fancyhdr}

% External references (optional, for future SI)
\externaldocument{SI}

\hypersetup{colorlinks=true,linkcolor=blue,citecolor=blue,urlcolor=blue}

% Footer marking the draft status
\pagestyle{fancy}
\fancyhf{}
\fancyfoot[L]{\small Draft — version posted to Zenodo on 2025-09-12}
\fancyfoot[R]{\thepage}
\renewcommand{\headrulewidth}{0pt}
\renewcommand{\footrulewidth}{0pt}

% ---------- Macros
\newcommand{\cs}{c_s}
\newcommand{\Heff}{H_{\mathrm{eff}}}
\newcommand{\Nband}{\mathcal N}
\newcommand{\dd}{\mathrm d}
\newcommand{\rcl}{R_{\mathrm{cl}}}
\newcommand{\vflat}{v_{\mathrm{flat}}}
\newcommand{\kmin}{k_{\min}}
\newcommand{\kmax}{k_{\max}}

% Shortcuts
\newcommand{\grad}{\nabla}

\title{Scale\,–\,Downshifting as a Unified Origin for Dark Matter and Dark Energy (\textit{Draft})}
\author{Franz Wollang\\ \small Independent Researcher}
\date{\small Dated: 2025-09-12}

\begin{document}
\maketitle

% Prominent draft disclaimer box
\begin{center}
\setlength{\fboxsep}{8pt}%
\fbox{\parbox{0.92\textwidth}{\centering\bfseries DRAFT — NOT FOR CITATION\\[4pt]
This is a preliminary working version posted for discussion and feedback. Content may change significantly before formal submission.}}
\end{center}
\vspace{1em}

\begin{abstract}
We present a framework in which the phenomena attributed to dark matter and dark energy arise from a single, unified mechanism: the conformal scaling of matter in response to a background field with stationary, scale-invariant statistics. This work builds upon a foundation wherein matter is described as solitons that both source and respond to this background. We show that two distinct consequences emerge from this interaction. First, the inhomogeneous, spatial response of solitons to local gradients in the background naturally gives rise to galactic-scale phenomena consistent with dark matter, such as isothermal-like halos and flat rotation curves. Second, the collective result of this ubiquitous local scaling is a homogeneous, temporal drift of the observational window, which produces an effective cosmic acceleration with an equation of state $w \approx -1$, accounting for the effects of dark energy. Cosmological predictions are expressed in local co-scaling units as an operational (conformal) gauge choice that is fully equivalent to the standard FLRW description for observables (redshift and distances are invariant; see SI). The model provides a self-consistent alternative to particle-based dark sector candidates and makes several falsifiable predictions.
\end{abstract}

\section{Minimal Foundations}
This paper builds on a physical framework developed in prior work, where matter and its interactions are emergent phenomena. In this view, matter is described as stable, localized configurations (solitons) of a single underlying complex field. These solitons collectively source a background that, in turn, mediates their interaction, causing their characteristic scales to adjust conformally to the local environment. The present work applies this foundation to cosmology, showing how dark matter and dark energy can be understood as two manifestations of this collective response.

We assume familiarity with the core concepts developed in prior papers in this series—specifically, the ICG paper\footnote{ICG paper: Spacetime from First Principles: Free\,–\,Energy Foundations on the Infinite\,–\,Clique Graph (Draft), Zenodo (2025\,–\,08\,–\,29). DOI: 10.5281/zenodo.15843979.} (Secs. S1–S10) and the Relativity paper\footnote{Relativity paper: A Sea of Noise: Relativity from a Thermodynamic Force in Scale\,–\,Space (Draft), Zenodo (2025\,–\,08\,–\,30). DOI: 10.5281/zenodo.17000665.} (Secs. 2–7). For a concise recap and symbol map, see Appendix~\ref{app:dynamical} and SI Sections~\ref{si:notation} and \ref{si:foundations}. We use these results without re\,–\,deriving them here.

\subsection{Relation to other approaches}
This framework differs from other emergent or modified gravity theories in several key respects. Unlike MOND and its relativistic extensions, no additional long-range field beyond the emergent metric is required; the inverse-square law is recovered within the Coulombic window $R_{\mathrm{cl}} \ll r \ll \ell$, and the Parametrized Post-Newtonian parameters remain $\gamma = \beta = 1$ at leading order. Unlike entropic gravity proposals, the attractive force has a concrete microphysical origin in scale-space free energy rather than emergent thermodynamic relations. The framework resembles analog gravity models in deriving effective spacetime structure from an underlying substrate, but differs in starting from a discrete, graph-theoretic foundation rather than a continuous medium. Fifth-force constraints are naturally satisfied because the massless phase sector obeys shift symmetry, restricting couplings to derivative interactions with conserved currents that do not generate static $1/r$ forces between isolated sources (see SI Section~\ref{si:varying-constants} for details).


\section{A Unified Mechanism: Two Facets of Scale\,–\,Downshifting}
The central thesis is that the phenomena attributed to dark matter and dark energy are dual manifestations of a common substrate—the vacuum noise field—operating through distinct channels: a spatial, inhomogeneous response producing effects attributed to dark matter, and a homogeneous, temporal response producing those of dark energy.

\subsection{Common Driver: The Scale-Environment Coupling}
The framework's dynamics are driven by a universal free-energy minimization principle. Locally, this drives solitons towards regions of higher noise to lower their energy, causing them to shrink and cluster; this is the chaotic, structure-forming process responsible for the effects we attribute to gravity and dark matter. Averaged over cosmological scales, this universal tendency for all matter to settle into more compact, lower-energy states results in a slow, secular, and statistically homogeneous monotonic shrinkage of the absolute equilibrium scale, $L_*(t)$, of all bound systems.

This fundamental contraction of all local standards creates the dual perception of an expanding universe through a simple inversion of perspective. The logic proceeds in three steps: (i) \textbf{Fundamental Process:} The absolute scale of all matter and bound systems, $L_*(t)$, slowly and monotonically contracts over cosmic time, meaning any physical ruler we can construct is fundamentally shrinking. (ii) \textbf{Observational Consequence:} An observer measures a fixed, absolute distance between two unbound galaxies using their shrinking local ruler. As the ruler itself gets shorter, the number of ruler-lengths needed to span the fixed distance increases, so the measured distance appears to grow over time. (iii) \textbf{Emergent Picture:} This apparent growth of all large-scale distances, measured with our shrinking local standards, is what we observe as the Hubble expansion in the dual description.
\paragraph{Energy bookkeeping.} In the adiabatic limit (internal size $\sigma$ relaxes fast), scale reduction lowers the relaxed internal free energy $E_{\rm eq}(x)$ while increasing center-of-mass kinetic energy. The mechanical balance is
\begin{equation}
  \frac{dK}{dt} \;=\; \mathbf F\!\cdot\!\mathbf v \;=\; -\,\frac{dE_{\rm eq}}{dt},
\end{equation}
with $\mathbf F=-\nabla E_{\rm eq}$. Both contributions belong to the same matter stress--energy; total conservation holds ($\nabla_{\mu}T^{\mu\nu}_{\rm tot}=0$) with no leading exchange with the homogeneous background in local units. See SI Section~\ref{si:conservation} for the covariant derivation.
\paragraph{Operational duality (gauge equivalence).} Expressing cosmological predictions in local co-scaling units is a conformal/unit choice rather than a change of physics. For an FLRW metric $g_{\mu\nu}$ with scale factor $a(t)$, the co-scaling representation uses $\hat g_{\mu\nu}=F(t)^2 g_{\mu\nu}$ with local-unit factor $F(t)$ (Sec.~\ref{sec:de}). Null geodesics are conformally invariant up to reparametrization; the observable redshift $1+z=(k_\mu u^\mu)_{\rm em}/(k_\mu u^\mu)_{\rm obs}$ and the Etherington relation $D_L=(1+z)^2 D_A$ are unchanged when tetrads co-scale with $F(t)$. A compact proof and a worked example are given in SI Section~\ref{si:gauge-equivalence}.
In what follows we develop the two channels separately: the spatial (inhomogeneous) consequences that recover gravity and dark-matter phenomenology (Sec.~\ref{sec:dm}), and the temporal (homogeneous) consequences that yield an effective near-$w=-1$ component (Sec.~\ref{sec:de}).

\subsection{Spatial Manifestation: Dark Matter Halos from Noise Gradients}\label{sec:dm}
Matter sources a noise field $\rho_N$; the superposition of these short\,–\,range responses from a large object creates an effective $1/r$ potential in the Coulombic window, recovering the Newtonian force law in the weak\,–\,field limit. This emergent gravity is sourced by perturbations $\delta\rho_N$ on top of a mean\,–\,field background $\bar{\rho}_N$. As this mechanism introduces no new long\,–\,range fields coupled to composition, it is consistent with stringent tests of the equivalence principle. In the weak\,–\,field, slow\,–\,motion limit, the effective metric reduces to the standard form
\begin{equation}
  ds^2 = -\Big(1+\frac{2\Phi}{c_s^2}\Big) c_s^2 dt^2 \; + \; \Big(1-\frac{2\Phi}{c_s^2}\Big)(dx^2+dy^2+dz^2),
\end{equation}
which implies $\gamma_{\rm PPN}=\beta_{\rm PPN}=1$ at leading order (see the Relativity paper, Sec.~7.3).

With this foundation, the explanation for galactic dark matter follows directly. The flat rotation curves of galaxies imply a halo with a density profile $\rho(r) \propto r^{-2}$. Such a profile is the natural equilibrium state for a self\,–\,gravitating, isothermal system. The framework provides a physical justification for this isothermal condition: in local units, the conformal compensation between the scale\,–\,free driving bath and local measurement standards yields a radius\,–\,independent velocity dispersion $\sigma$. With $\sigma \approx \mathrm{const}$, the Jeans equation dictates that the halo settles into the $\rho(r) \propto r^{-2}$ profile, which in turn yields flat rotation curves $v_c(r) = \mathrm{const}$ (see Appendix~\ref{app:jeans} for a full derivation). This leads to a concrete, falsifiable prediction: a tight relationship between a galaxy's rotation speed and its halo's velocity dispersion,
\begin{equation}
  v_c^2 \;\approx\; 2\,\sigma^2.\label{eq:vc-sigma}
\end{equation}
The model is also consistent with gravitational lensing and the observed separation of collisionless mass from collisional gas in systems like the Bullet Cluster.
\\[0.25em]
\noindent A kinetic derivation of the radius\,–\,independent dispersion in local units from a scale\,–\,invariant bath is provided in SI Section~\ref{si:kinetic-sigma}.

\subsubsection{Galaxies with apparently little dark matter}
Some galaxies are reported to have low mass discrepancies inside the probed radii. In this framework, such cases arise naturally when one or more conditions behind the isothermal tail are violated.

Environmental truncation is the most common cause. A strong host potential elevates the background and effectively shrinks the Coulombic window (the regime $\rcl \ll r \ll \ell$ where the long\,–\,range $1/r$ tail holds), while tides strip the low\,–\,surface\,–\,brightness outskirts that carry most of the isothermal mass. One expects low discrepancies near massive hosts, with broken symmetry and outer truncation.

Non\,–\,equilibrium and anisotropy effects provide another explanation. The relations $\rho\propto r^{-2}$ and Eq.~\eqref{eq:vc-sigma} assume steady, roughly isotropic dispersions. Young tidal dwarfs, puffy UDGs, or systems with significant anisotropy $\beta(r)$ and radially varying dispersions $\sigma_r(r)$ can yield underestimates of enclosed mass in simple isotropic Jeans fits.

Finally, some systems may have genuinely compact, baryon\,–\,dominated interiors. High surface density compacts (e.g., UCD\,–\,like or bulge\,–\,dominated dwarfs) can be baryon\,–\,dominated within the observed aperture even though an isothermal envelope exists further out.

Predicted observational signatures include: recovery of the $1/r$ tail and rising mass discrepancy beyond any tidal/truncation radius; improved mass inferences when allowing for $\beta(r)$ and $\dd\sigma_r^2/\dd\ln r$ in the Jeans analysis; and alignment of low\,–\,DM inferences with proximity to a massive host and with morphological signs of disturbance.

\subsubsection{Galaxies that are mostly or purely ``dark''}
``Dark'' here means star\,–\,poor but dynamically massive. The model requires only ordinary matter to source $\Phi$ (and thus $\tau$). Gas\,–\,rich, low\,–\,SFE systems (HI\,–\,dominated LSBs, some UDGs) are therefore expected: the isothermal envelope forms provided the Coulombic window holds, while star formation remains inefficient. Claims of “purely dark” galaxies with no detectable stars or gas must still contain baryons (e.g., cold gas, compact remnants, dust) or be tidal artifacts; otherwise they would challenge the assumption that no additional long\,–\,range field exists. The practical expectation is that deeper HI/CO/IR limits will reveal enough baryons to sustain the potential in most cases.

\subsubsection{Departures from the isothermal ideal (regimes and cores)}
Where the screening length $\ell$ is comparable to the probed radii or the Coulombic window fails ($r \not\gg \rcl$ or $r \not\ll \ell$), deviations from $\rho\propto r^{-2}$ are expected: cores (Helmholtz/Yukawa corrections), anisotropic dispersions, and broken superposition in crowded environments. In these regimes the framework predicts smooth departures rather than abrupt failures, with a return to the $1/r$ tail once $\rcl \ll r \ll \ell$ is restored.\\[0.25em]
\noindent Practical diagnostics and modeling recipes (anisotropy $\beta(r)$, slowly varying $\sigma_r(r)$, external\,–\,field/tidal truncation) are collected in SI Section~\ref{si:dm-diagnostics}.
\\[0.25em]
\noindent Parameter bands for the screening length $\ell=\sqrt{\alpha_{\rm grad}/(2\beta_{\rm pot})}$ that yield halo\,–\,scale isothermal tails without instability, and their mapping to core radii and environment\,–\,induced truncations, are derived in SI Section~\ref{si:ell-bands}.

\subsection{Temporal Manifestation: Effective Dark Energy from a Sliding Window}\label{sec:de}
The homogeneous consequence of scale--downshifting is an effective dark energy component. We describe this by modeling the background as a stationary, scale-invariant spectrum $S(k)$ viewed through a time-evolving observational window in local units. A scale-invariant spectrum, which assigns equal content per logarithmic band, naturally takes the form $S(k)\propto 1/k$; this is the maximum-entropy prior for a scale-free process and acts as a fixed point for multiplicative cascades. The effective background density is then the windowed integral
\begin{equation}
  \rho_{N,\mathrm{eff}}(t) \;=\; \int_{\kmin(t)}^{\kmax(t)} S(k)\,\dd k \;\propto\; \ln\!\left(\frac{\kmax(t)}{\kmin(t)}\right). \label{eq:rhoN-eff}
\end{equation}

The physical origin of the window's drift is the secular shrinkage of matter itself. As the absolute scale of local rulers $L_*(t)$ contracts (so $F(t) \propto 1/L_*(t)$ increases), our perception of all wavelengths shifts. The UV edge of our window, $\kmax$, is a resolution cutoff tied to this smallest resolvable feature, and thus scales as $\kmax \propto F(t)$. Conversely, the IR edge, $\kmin$, is a causal cutoff tied to the largest coherently probed scale, which in the dual picture is the Hubble horizon, scaling as $\kmin \propto \Heff(t)$. Since $\Heff$ is itself defined by the rate of change of $L_*$, both edges co-evolve. This kinematically-driven drift---improving UV resolution as rulers shrink ($\mathrm d k_{\max}/\mathrm dt > 0$) and an expanding horizon admitting longer wavelengths ($\mathrm d k_{\min}/\mathrm dt < 0$)---is the source of the effect.

Differentiating the windowed integral and matching it to a smooth component yields an effective equation of state
\begin{equation}
  w + 1 \;\approx\; \frac{\dot F/F - \dot \Heff/\Heff}{3\,\Heff\,\Nband},\qquad \Nband \equiv \ln\!\left(\frac{\kmax}{\kmin}\right).\label{eq:w-plus-one}
\end{equation}
The large logarithmic bandwidth $\Nband$ naturally suppresses deviations from $w=-1$, yielding a small, smooth drift consistent with observations (see SI Sec.~\ref{si:w-bounds} for quantitative bounds). Backreaction from structure formation is a small, higher-order effect (SI Sec.~\ref{si:backreaction}). Generalizations for edge scalings and spectral tilts are provided in SI Sec.~\ref{si:sliding}. As a complementary, operational framing, the same edge evolution can be recast as an “optimal bandwidth” principle which is useful for calibration and simulation (SI Sec.~\ref{si:sliding}).

We retain a single symbol map and regime throughout: $\tau^2 \propto \rho_N$ by definition of the windowed variance; in the static, weak\,–\,field window the same field solves Poisson's equation with the matter source so $\tau^2 \propto \Phi$ up to a reference offset. The Newtonian/isothermal results apply in the Coulombic window $\rcl \ll r \ll \ell$, where $\ell=\sqrt{\alpha_{\mathrm{grad}}/(2\beta_{\mathrm{pot}})}$ is the screening length from the gapped amplitude sector. Under these conditions the Jeans analysis enforces $\rho\propto r^{-2}$ and Eq.~\eqref{eq:vc-sigma} (with mild corrections from anisotropy), matching the Relativity paper's weak\,–\,field sector. Cosmologically, observables are stated in local units co\,–\,scaling with the sliding window, so homogeneous drifts affect dimensionful numbers but not dimensionless laws to leading order.

\section{Consistency with the Standard Cosmological History and Falsifiable Predictions}
A viable model must not only explain the dark sector but also remain consistent with the well\,–\,established history of the early universe and make unique, testable predictions. The framework achieves consistency by requiring that the conformal factor $F(t)$ was effectively constant at early times ($z \gtrsim 1000$), ensuring that the effective dark energy density at recombination is negligible ($f_{\mathrm{EDE}}(z\!\approx\!1100) \approx 10^{-9}$) and preserving the physics of Big Bang Nucleosynthesis and the Cosmic Microwave Background (see SI Sections~\ref{si:early-suppression} and \ref{si:early-consistency} for justifications). Furthermore, as gravity is unmodified on large scales ($G_{\mathrm{eff}}=G$) and the effective expansion history $\Heff(z)$ can be matched to that of $\Lambda$CDM, the model predicts an identical linear growth history for cosmic structures, governed by the standard equation:
\begin{equation}
    \delta'' + \Big[2 + (H'/H)\Big] \delta' - \frac{3}{2}\,\Omega_m(a_{\mathrm{eff}})\,\delta = 0\,.
\end{equation}
The framework's unified nature leads to several concrete and falsifiable predictions that distinguish it from standard $\Lambda$CDM:
\subsubsection{Dark matter: flat speed–dispersion relation}
A tight relation between a galaxy's flat rotation speed and halo dispersion is predicted: $\vflat^{\,2} \approx 2\,\sigma^2$. This relation follows from the isothermal envelope in local units and is robust to mild anisotropy. Deviations are expected where the Coulombic window is truncated or dispersions vary strongly with radius.

\subsubsection{Dark energy: small deviation from $w=-1$}
A specific, tiny deviation of $w$ from $-1$ is tied to the deceleration parameter $q$ and the logarithmic bandwidth $\Nband$ defined by the sliding window. The magnitude is set by the drift of local units relative to the causal edge and is suppressed by the large bandwidth $\Nband$, implying small, smooth departures consistent with current bounds.

\subsubsection{Cross–sector: varying constants}
A tiny, secular drift in constants is expected. If $\alpha \propto \bar{\rho}_N^{-\zeta}$, then $|\dd \ln \alpha/\dd t| \sim \zeta\,10^{-3} H_0$, giving $|\Delta\alpha/\alpha| \lesssim 7\times 10^{-7}$ over a Gyr for $\zeta \lesssim 10^{-2}$ (details in SI Section~\ref{si:varying-constants}). Window-driven drifts yield slow, environment-suppressed variation tied to the same bandwidth factor that governs dark energy. Existing clock and astrophysical limits bound the coupling $\zeta$ to the stated range.

\subsubsection{Environment diagnostic}
Mass discrepancy correlates with host tides/external field. Near massive hosts, expect truncated halos and lower discrepancies inside the truncation radius; beyond that radius, recovery toward the $1/r$ tail. Tidal fields reduce the effective outer envelope, lowering discrepancies within the truncation radius, with recovery predicted beyond the tide-affected region.

\subsubsection{Anisotropy diagnostic}
Allowing for $\beta(r)$ and $\dd\sigma_r^2/\dd\ln r$ in Jeans analyses of "low\,–\,DM" systems should recover higher enclosed masses and approach $\rho\propto r^{-2}$ at larger radii. Accounting for $\beta(r)$ and $\dd\sigma_r^2/\dd\ln r$ lifts mass–anisotropy degeneracies and should reconcile low-DM inferences with the isothermal expectation at larger radii.

\subsubsection{Gas accounting for dark candidates}
Deep HI/CO/IR searches should reveal sufficient ordinary matter in star\,–\,poor, dynamically massive systems. Robustly baryon\,–\,free, relaxed, isolated massive halos would challenge the framework. Since ordinary matter sources $\Phi$, star-poor but gas-rich systems are natural; truly baryon-free massive halos would disfavor the framework.

\subsubsection{Coulombic–window validity}
Systems violating $\rcl \ll r \ll \ell$ should display smooth, predictable departures from isothermality (cores, slope changes), with a return to $1/r$ behavior where the window is restored. Where $R_{\mathrm{cl}} \ll r \ll \ell$ fails, smooth, predictable departures (cores, slope shifts) occur with a return to the $r^{-2}$ tail once the window reopens.

\subsubsection{Late–time $w(z)$ drift}
The framework allows localized late–time deviations in edge scalings, $\alpha(z)=1+\delta\alpha\,f(z)$ and $\beta(z)=1+\delta\beta\,g(z)$ with $z\lesssim1$ and $|\delta\alpha|,|\delta\beta|\ll1$, preserving early–time distances while enhancing the late–time effective dark energy density. A targeted search for such small, smooth deviations in $w(z)$ provides a direct test (details in SI Section~\ref{si:hubble}). Small, localized changes in edge scalings at $z \lesssim 1$ can enhance late-time effective dark energy while preserving early-time distances. This is directly testable with precise $w(z)$ reconstructions.


\section{Conclusion}
The soliton\,–\,noise framework, built on a minimal set of physical postulates, offers a common-substrate explanation for the dark sector. It recasts dark matter and dark energy not as mysterious substances, but as the inhomogeneous and homogeneous manifestations of matter's response to a scale-invariant vacuum background through distinct but related channels. Throughout we work in an operational, co-scaling gauge for local units that is observably equivalent to standard FLRW (see SI), ensuring that redshifts and distances are unchanged while clarifying conservation and microphysical origins. This provides a self\,–\,consistent and highly predictive alternative to the standard $\Lambda$CDM paradigm.

\section{References Prep}
The $\Lambda$CDM background and observational pillars (CMB, BAO, SN Ia) are covered in standard reviews.

Isothermal halos and Jeans analysis in galactic dynamics are treated in texts such as Binney \& Tremaine.

PPN constraints and Solar\,–\,System tests of gravity include Cassini and LLR measurements of $\gamma_{\rm PPN}$ and $\beta_{\rm PPN}$.

Bounds on varying constants come from quasar absorption, Oklo, and atomic clock studies.

Backreaction and Buchert averaging are addressed in the cosmological coarse\,–\,graining literature.

\appendix

\section{Dynamical Foundations of Matter and Gravity}\label{app:dynamical}
This appendix only sketches prior results used as assumptions here. Stability, relaxation hierarchy, and mean\,–\,field arguments are derived in the ICG paper and the Relativity paper; a compact recap and symbol map are provided in SI Sections~\ref{si:notation} and \ref{si:foundations}. Readers can consult those for full proofs and parameter conditions; we rely only on their leading\,–\,order consequences in the present text.

The framework rests on a description of matter as stable, localized excitations (solitons) of a complex order parameter, $\Psi=w\,e^{i\phi}$. This field has two generic modes: a massive amplitude sector ($w$) that forms the cores of particles, and a massless phase sector ($\phi$) that mediates interactions. Solitons source a scalar background field, whose local intensity is proxied by $\tau(x)$. The gravitational interaction arises from a scale-space free energy that drives solitons toward regions of higher $\tau$, producing an attractive force. In the weak-field, static limit, this mechanism recovers the Newtonian $1/r^2$ force law, allowing for the identification $\tau^2 \propto \Phi$, where $\Phi$ is the Newtonian potential.

The universal kinematics are defined by the massless phase sector, which establishes a single light cone with characteristic speed $c_s$ and an emergent metric for all forms of matter, thus explaining the weak equivalence principle. All physical quantities are defined in local units set by the characteristic scales of these solitons. In a cosmological context, these local standards are not fixed but co-evolve with a sliding observational window. This conformal scaling of local rulers and clocks relative to the background is the foundational principle from which the dark sector phenomenology is derived in this paper.

\section{The Scale\,–\,Space Force and Downshifting Rule}\label{app:scale-space}
We justify the scale\,–\,noise coupling by showing how an attractive force toward higher noise densities emerges from a generic energy functional for localized excitations. This is a condensed summary; for the full derivation, see the Relativity paper (Sec. 3.2 and Appendix D).

\subsection*{Physical Model: Balancing Repulsion and Cohesion}
A soliton's equilibrium size, $\sigma$, is determined by a balance between two opposing effects: an internal repulsive pressure that resists compression, and an attractive cohesion that binds the configuration. These two components scale differently with size and respond differently to the environment. \textbf{Repulsion} arises from the soliton's internal gradient energy, which penalizes sharp changes in the field and creates an effective pressure that scales with size as $E_{\text{rep}} \propto \sigma^{-2}$; this term is an intrinsic property of the soliton. \textbf{Cohesion} arises from the interaction of the soliton's boundary layer with the surrounding background field and is modeled as a surface-like term, scaling as $E_{\text{coh}} \propto -\sigma^{-1}$. The key insight is that the strength of the cohesion is amplified by the local intensity of the background field, proxied by $\tau(x)$. In regions of higher background intensity, the soliton can rely on this external "pressure" for cohesion, admitting a smaller configuration at lower total free energy.

\subsection*{Coarse-grained Energy and Force Derivation}
This physical model is captured by a minimal coarse-grained energy functional for a soliton of size $\sigma$ at position $x$:
\begin{equation}
  E_{\mathrm{coh}}(\sigma, x) = A_{\gamma}\,\sigma^{-2} - A_{\eta}\,\tau(x)\,\sigma^{-1}.\label{eq:E-coh}
\end{equation}
Here, the first term represents repulsion ($A_{\gamma}>0$) and the second represents the $\tau$-dependent cohesion ($A_{\eta}>0$). Under an adiabatic assumption (where the internal scale $\sigma$ relaxes much faster than the center-of-mass motion), we can find the equilibrium size by minimizing $E_{\mathrm{coh}}$ with respect to $\sigma$:
\begin{equation}
  \frac{\partial E_{\mathrm{coh}}}{\partial \sigma} = -2A_{\gamma}\,\sigma^{-3} + A_{\eta}\,\tau(x)\,\sigma^{-2} = 0 \quad\Rightarrow\quad \sigma^*(x) = \frac{2 A_{\gamma}}{A_{\eta}\,\tau(x)}.
\end{equation}
Substituting this equilibrium size back into the energy functional gives the relaxed, position-dependent potential energy of the soliton:
\begin{align}
  E_{\mathrm{eq}}(x) &= E_{\mathrm{coh}}\big(\sigma^*(x), x\big) = -\,\frac{A_{\eta}^{\,2}}{4 A_{\gamma}}\,\tau(x)^2.
\end{align}
The force on the soliton is the negative gradient of this potential energy, $\mathbf F(x) = -\grad E_{\mathrm{eq}}(x)$, which yields:
\begin{equation}
  \mathbf F(x) = +\,\frac{A_{\eta}^{\,2}}{2 A_{\gamma}}\,\tau(x)\,\grad\tau(x).
\end{equation}
Since $\tau(x)$ is a proxy for the local density of matter, this force is always attractive, directed toward regions of higher matter density. In the weak-field, static limit where this background field recovers the Newtonian potential via the identification $\tau^2 \propto \Phi$, this mechanism provides a microphysical origin for gravity.

\subsection*{Robustness}
The attractive nature of the force is robust. For any generic balance of the form $E(\sigma,x)=A\,\sigma^{-p}-B\,\tau(x)\,\sigma^{-q}$ with pressures $p>q>0$, the resulting force is always attractive, $\mathbf F\propto +\,\tau^{p/(p-q)-1}\,\grad\tau$. The exponents $p=2, q=1$ are physically motivated by gradient energy and boundary-dominated cohesion.

\section{Isothermal Halos from the Jeans Equation}\label{app:jeans}
This appendix collects the standard Jeans\,–\,equation derivation of flat rotation curves and $\rho\propto r^{-2}$ halos, and connects it to the framework used in the main text.

\subsection{Setup and assumptions}
Consider a steady, spherically symmetric, collisionless system with density $\rho(r)$ and potential $\Phi(r)$. Define the radial velocity dispersion $\sigma_r^2(r)=\langle v_r^2\rangle$ and the Binney anisotropy parameter $\beta(r):=1-\sigma_t^2/(2\sigma_r^2)$, where $\sigma_t$ is the one\,–\,dimensional tangential dispersion. The spherical Jeans equation and Poisson equation are
\begin{align}
  \frac{\dd\, (\rho\,\sigma_r^2)}{\dd r} + \frac{2\beta}{r} \, \rho\,\sigma_r^2 &= -\,\rho\,\frac{\dd\Phi}{\dd r},\\
  \frac{1}{r^2}\,\frac{\dd}{\dd r}\!\left(r^2\,\frac{\dd\Phi}{\dd r}\right) &= 4\pi G\,\rho.\label{eq:jeans-poisson}
\end{align}

\subsection{Isotropic, isothermal case}
Take $\beta=0$ and $\sigma_r(r)\equiv\sigma=\mathrm{const}$. Then the Jeans equation gives
\begin{equation}
  \sigma^2\,\frac{\dd\ln\rho}{\dd r} = -\,\frac{\dd\Phi}{\dd r}.\label{eq:jeans-iso}
\end{equation}
Insert into Poisson:
\begin{equation}
  -\,\sigma^2\,\frac{1}{r^2}\,\frac{\dd}{\dd r}\!\left(r^2\,\frac{\dd\ln\rho}{\dd r}\right) = 4\pi G\,\rho.\label{eq:poisson-iso}
\end{equation}
Seek a power\,–\,law $\rho\propto r^{-\alpha}$ so $\dd\ln\rho/\dd r = -\alpha/r$. Then
\begin{equation}
  -\,\sigma^2\,\frac{1}{r^2}\,\frac{\dd}{\dd r}\!\left(r^2\,\frac{-\alpha}{r}\right) = \frac{\alpha\,\sigma^2}{r^2} = 4\pi G\,A\,r^{-\alpha},\label{eq:power-law}
\end{equation}
which enforces $\alpha=2$ and fixes $A=\sigma^2/(2\pi G)$. Thus the singular isothermal sphere has
\begin{align}
  \rho(r) &= \frac{\sigma^2}{2\pi G}\,\frac{1}{r^2},\label{eq:rho-iso}\\
  v_c^2(r) &:= r\,\frac{\dd\Phi}{\dd r} = 2\,\sigma^2 = \mathrm{const}.\label{eq:vc-flat}
\end{align}
The enclosed mass then grows linearly with radius,
\begin{equation}
  M(r) = \frac{v_c^2\, r}{G}.\label{eq:mass-iso}
\end{equation}

\subsection{Constant anisotropy}
For constant $\beta$ and constant $\sigma_r$, the Jeans equation yields
\begin{equation}
  \sigma_r^2\,\frac{\dd\ln\rho}{\dd r} + \frac{2\beta\,\sigma_r^2}{r} = -\,\frac{\dd\Phi}{\dd r}.\label{eq:jeans-beta}
\end{equation}
Assuming a flat rotation curve (constant $v_c$ so $\dd\Phi/\dd r = v_c^2/r$) and a power\,–\,law $\rho\propto r^{-\alpha}$ gives
\begin{equation}
  v_c^2 = (\alpha - 2\beta)\,\sigma_r^2.\label{eq:vc-beta}
\end{equation}
Self\,–\,consistency with Poisson for a flat curve still imposes $\alpha=2$ ($\rho\propto r^{-2}$), yielding
\begin{equation}
  v_c^2 = 2(1-\beta)\,\sigma_r^2,\label{eq:vc-beta-iso}
\end{equation}
which reduces to the isotropic result $v_c^2=2\sigma^2$ when $\beta=0$.

\subsection{Core regularisation (pseudo\,–\,isothermal)}
The solution above is singular at $r\to0$. Real halos display finite cores. A standard empirical regularisation is the pseudo\,–\,isothermal profile
\begin{equation}
  \rho(r) = \frac{\rho_0}{1+(r/r_c)^2},\qquad
  v_c^2(r) = 4\pi G\rho_0 r_c^2\left[1-\frac{r_c}{r}\arctan\!\left(\frac{r}{r_c}\right)\right],\label{eq:pseudo-iso}
\end{equation}
which approaches $v_c^2\to 4\pi G\rho_0 r_c^2$ for $r\gg r_c$ and $\rho\sim r^{-2}$ asymptotically. In this framework, a finite screening length in the amplitude sector supplies a microphysical core scale playing the role of $r_c$.

\subsection{Connection to the framework}
The “isothermal” condition corresponds to a radius\,–\,independent one\,–\,dimensional dispersion $\sigma$ in \emph{local units}. Conformal co\,–\,scaling of rulers/clocks with the background keeps $\sigma$ approximately constant across the halo, and the Jeans analysis above then enforces the $\rho\propto r^{-2}$ tail and flat $v_c$, with $v_c^2\simeq2\sigma^2$ (or $2(1-\beta)\sigma_r^2$ with mild anisotropy).

\subsection{Deviations and corrections}
In real systems $\sigma_r$ and $\beta$ vary slowly with radius and inner cores regularise the centre. From the general Jeans equation,
\begin{equation}
  \sigma_r^2\,\frac{\dd\ln\rho}{\dd r} + \frac{\dd\sigma_r^2}{\dd r} + \frac{2\beta\,\sigma_r^2}{r} = -\,\frac{v_c^2}{r},\label{eq:jeans-general}
\end{equation}
the local logarithmic density slope is
\begin{equation}
  \alpha(r) \equiv -\,\frac{\dd\ln\rho}{\dd\ln r} = \frac{v_c^2}{\sigma_r^2} + \frac{\dd\ln\sigma_r^2}{\dd\ln r} + 2\,\beta(r).\label{eq:alpha-local}
\end{equation}
Relative to the ideal isotropic–isothermal value $\alpha=2$, slowly rising (falling) dispersions $\dd\ln\sigma_r^2/\dd\ln r>0$ ($<0$) steepen (flatten) the profile, and tangential ($\beta<0$) vs radial ($\beta>0$) anisotropy flattens vs steepens it. A finite core (screening length or coherence scale $r_c$) produces $\rho\!\approx\!\mathrm{const}$ and $v_c\!\propto\! r$ for $r\ll r_c$, transitioning to the $r^{-2}$ tail and flat $v_c$ for $r\gg r_c$.

\subsection{Tracer formulation and mass–anisotropy degeneracy}
In observed systems the kinematic tracers (stars, gas, globular clusters) need not follow the total mass density. Let $\nu(r)$ denote the spherically symmetric tracer density. The steady spherical Jeans equation becomes
\begin{equation}
  \frac{\dd\big(\nu\,\sigma_r^2\big)}{\dd r} + \frac{2\beta}{r}\,\nu\,\sigma_r^2 = -\,\nu\,\frac{\dd\Phi}{\dd r} = -\,\nu\,\frac{G M(r)}{r^2},\label{eq:jeans-tracer}
\end{equation}
with $M(r)=4\pi\int_0^r \rho(s)\,s^2\,\dd s$. Given $(\nu, \beta)$ and a boundary condition on $\nu\,\sigma_r^2$, Eq.~\eqref{eq:jeans-tracer} yields $\sigma_r^2(r)$ for any candidate $M(r)$. The well\,–\,known mass–anisotropy degeneracy reflects the fact that different pairs $(M,\beta)$ can fit the same line\,–\,of\,–\,sight data; practical inferences benefit from external constraints on $\beta(r)$ and from multiple tracers.

\subsection{Projected observables}
The predicted line\,–\,of\,–\,sight dispersion profile for a spherical system at projected radius $R$ is
\begin{equation}
  \sigma_{\!\mathrm{los}}^2(R) \;=\; \frac{2}{\Sigma(R)} \int_R^{\infty} \left(1 - \beta\,\frac{R^2}{r^2}\right) \frac{\nu(r)\,\sigma_r^2(r)\,r\,\dd r}{\sqrt{r^2 - R^2}},\qquad
  \Sigma(R) = 2 \int_R^{\infty} \frac{\nu(r)\,r\,\dd r}{\sqrt{r^2 - R^2}}.\label{eq:sigmalos}
\end{equation}
For isotropic, isothermal tracers with $\nu\propto r^{-2}$ one finds $\sigma_{\!\mathrm{los}}\!\approx\!\mathrm{const}$ over a broad range of $R$, consistent with flat outer dispersions in halos. When $\beta\neq0$ or $\nu$ differs from $r^{-2}$, $\sigma_{\!\mathrm{los}}(R)$ gently tilts according to Eq.~\eqref{eq:sigmalos}; this behaviour underlies the diagnostics listed in SI Section~\ref{si:dm-diagnostics}.

\subsection{Boundary conditions and finite extents}
A practical choice is to impose $\nu\,\sigma_r^2\!\to\!0$ as $r\to\infty$ for extended halos, or to match at a truncation radius $r_t$ when tides/external fields are important. The latter produces the observational signatures discussed in the main text: lowered discrepancies inside $r_t$ and recovery toward the $1/r$ tail beyond it. Inner boundary conditions are set by core physics (e.g., screening length), yielding the pseudo\,–\,isothermal behaviour of Eq.~\eqref{eq:pseudo-iso}.

\subsection{Lensing and dynamical consistency}
An isothermal sphere with one\,–\,dimensional dispersion $\sigma$ produces a surface density $\Sigma(R)\propto R^{-1}$ and a constant deflection angle. In the weak\,–\,field limit used here, the same $\rho\propto r^{-2}$ profile that yields flat rotation curves also reproduces standard lensing phenomenology, reinforcing the internal consistency of the isothermal envelope within this framework (with $c\to c_s$ in local units).
\\[0.25em]
\noindent In the weak\,–\,field metric $ds^2=-(1+2\Phi/c_s^2)c_s^2dt^2+(1-2\Phi/c_s^2)d\mathbf x^2$, the leading PPN parameters are $\gamma=\beta=1$, so deflection, Shapiro delay, and time\,–\,delay distances match GR expressions with $c\to c_s$. For an isothermal sphere $\rho=\sigma^2/(2\pi G r^2)$ one has $\Sigma(R)=\sigma^2/(2GR)$ and a constant reduced deflection angle, consistent with observations. Details are summarized in SI Section~\ref{si:lensing-ppn}.

\end{document}


