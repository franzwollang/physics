\documentclass[11pt]{article}

% --- Packages ---
\usepackage[a4paper,margin=1in]{geometry}
\usepackage{amsmath,amssymb,amsfonts,amsthm}
\usepackage{mathtools}
\usepackage{graphicx}
\usepackage{xcolor}
\usepackage{microtype}
\usepackage{booktabs}
\usepackage{enumitem}
\usepackage{fancyhdr}
\usepackage{authblk}
\usepackage{cite}
\usepackage{bm}
\usepackage[colorlinks=true,linkcolor=blue,citecolor=blue,urlcolor=blue]{hyperref}
\usepackage{xr-hyper}
\externaldocument{SI}

% --- Draft Status Footer ---
\pagestyle{fancy}
\fancyhf{}
\fancyfoot[L]{\small Draft — version posted to Zenodo on \today}
\fancyfoot[R]{\thepage}
\renewcommand{\headrulewidth}{0pt}
\renewcommand{\footrulewidth}{0pt}

% --- Macros & Notation ---
\newcommand{\cs}{c_s}
\newcommand{\Lp}{L_{\min}}
\newcommand{\Mp}{M_P}
\newcommand{\Nstar}{N_\ast}
\newcommand{\rhos}{\rho_{\rm surf}}
\newcommand{\Te}{T_{\rm eff}}
\newcommand{\dd}{\mathrm{d}}

% --- Metadata ---
\title{Horizon Duality: Black Holes as Asymptotic Shells\\of a Fractal Cosmos (\textit{Draft})}
\author[ ]{Franz Wollang}
\affil[ ]{\small Independent Researcher}
\date{\small Dated: \today}

\begin{document}

\maketitle

% --- Draft Disclaimer ---
\begin{center}
\setlength{\fboxsep}{8pt}%
\fbox{\parbox{0.92\textwidth}{\centering\bfseries DRAFT --- NOT FOR CITATION\\[4pt]
This is a preliminary working version posted for discussion and feedback.
Content may change significantly before formal submission.}}
\end{center}
\vspace{1em}

% ===================================================================
\begin{abstract}
\noindent We present a unified framework for black hole horizons and cosmic horizons
based on a soliton--noise model of fundamental matter.  In this picture, the
black hole event horizon is not a mathematical coordinate singularity but a
physical, dynamic phase transition: an ``asymptotic shell'' where matter
condenses to the vacuum's minimum scale length.  We derive the thermodynamics
of this shell, finding a temperature scaling $T\propto M^{-1/2}$ that implies
a constant-power evaporation law ($P\approx\text{const}$), contrasting with
the standard Hawking result.  We further demonstrate a rigorous duality between
the interior of such a shell and an expanding Friedmann--Robertson--Walker
(FRW) universe.  By matching clocks across the horizon, we show that the
parent black hole's finite evaporation time maps to a semi-infinite
cosmological history for the interior, characterized by an effective phantom
equation of state ($w<-1$) and an asymptotic approach to a Planck-scale heat
death.  This framework resolves the information paradox through
scale-dependent bookkeeping and suggests an eternal, fractal cosmology of
nested universes.
\end{abstract}

\tableofcontents
\vspace{1em}\hrule\vspace{1em}

% ===================================================================
\section{Introduction}

The similarity between the black hole event horizon and the cosmic particle
horizon has long hinted at a deeper connection.  Both are null trapping
surfaces that limit causal access; both are associated with thermodynamic
entropies and temperatures.  However, standard General Relativity (GR) treats
them as distinct kinematic entities.  Furthermore, the standard black hole
model suffers from the information loss paradox and the singularity problem,
while standard cosmology faces challenges regarding the initial state (Big
Bang) and the nature of dark energy.

This paper proposes that these are not separate problems.  We introduce a
\emph{Soliton-Noise Framework} in which matter particles are stable
wave-packets (solitons) interacting with a background noise field.  In this
view, gravity is an emergent thermodynamic force driving solitons toward
regions of higher noise variance.

The central hypothesis of this work is \textbf{Horizon Duality}:
\begin{enumerate}
  \item \textbf{Black Hole Horizons} are local regions where an intense
    spatial noise gradient drives matter to undergo a phase transition,
    collapsing to the minimum scale allowed by the vacuum (the Planck scale)
    and forming a physical shell.
  \item \textbf{Cosmic Horizons} are the manifestation of the same phase
    transition viewed from the inside (or globally), where the universal
    evolution of the noise field creates an apparent recession of the vacuum
    structure.
\end{enumerate}
We show that this model naturally regularizes the singularity (replacing it
with a black-body cavity), predicts a falsifiable deviation in Hawking
radiation (constant power), and provides a microphysical origin for the
``phantom'' dark energy ($w<-1$) required to reconcile the parent and child
clocks.

\subsection{Key Definitions \& Notation}
\begin{center}
\begin{tabular}{lp{0.72\textwidth}}
\toprule
\textbf{Symbol} & \textbf{Meaning} \\
\midrule
$\Lambda_{\rm obs}$, $\Lambda_{\rm IR/UV}$ & Observer's reference scale and
  IR/UV cutoffs for the \emph{scale window}. \\
$\Lp$ & The window-adapted minimum cell; the effective Planck length. \\
$\Nstar\approx\alpha$ & Critical occupancy for the quantum-to-gravitational
  phase transition. \\
$\cs$ & Phase-wave speed (speed of light). \\
$\tau(x)$ & Local noise proxy, related to noise density by
  $\tau^2\propto\rho_N$. \\
$K_{\rm drag}$ & Radiative drag coefficient setting the speed-limit
  asymptote. \\
\bottomrule
\end{tabular}
\end{center}

\paragraph{Consistency note (cross-document symbols).}
We standardize on $\cs$ for the universal phase-wave/light-cone speed (any bare
$c$ should be read as $\cs$), and we relate the local proxy $\tau(x)$ to the
cosmological noise density by $\tau^2\propto\rho_N$ in weak-field regimes.  We
reserve $(\alpha_{\rm grad},\beta_{\rm pot})$ for amplitude-sector parameters
and $(\alpha_{\rm dim},\beta_{\rm dim})$ for dimensional-bias parameters to
avoid symbol overloading.

% ===================================================================
\section{Foundations: The Soliton--Noise Mechanism}

We summarize the core postulates derived in companion papers to establish the
physical baseline.

\subsection{Matter as Solitons}
Fundamental particles are not point-like excitations but stable, localized
wave packets (solitons) of a complex scalar order parameter
$\Psi=w\,e^{i\phi}$.  These solitons are sustained by a Mexican-hat
self-interaction potential $V(w)$ that balances dispersive spreading.
Crucially, the soliton's stability depends on the local noise level: to
minimize its free energy, it adjusts its characteristic size $L_c$ and internal
frequency $\omega_c$ to match the local vacuum texture.

\subsection{The Noise Field and Emergent Gravity}
Solitons source and interact with a background stochastic field $\rho_N$ (the
``noise'').  This field acts as a refractive medium with index
$\chi\approx1+\alpha\rho_N$.
\begin{itemize}
  \item \textbf{Scale Adaptation:} A soliton adapts its characteristic size to
    the local noise intensity: $L_c\propto1/\chi$.
  \item \textbf{Emergent Gravity:} Gradients in the noise field create
    gradients in the refractive index.  Solitons drift toward regions of
    higher $\rho_N$ (higher~$\chi$, smaller~$L_c$) to lower their internal
    energy.  This thermodynamic drift is identified as the gravitational
    acceleration $\bm{a}_{\rm grav}\propto-\nabla\rho_N$.
  \item \textbf{Scale Invariance:} The physics is conformally invariant.  A
    local observer's rulers and clocks, being made of the same solitonic
    substance, co-scale with the noise field.  All locally measured constants
    (e.g.\ $\cs$) remain invariant.
\end{itemize}

\subsection{Thermodynamic Drag and the Planck Floor}
An accelerating soliton experiences a Doppler-shifted noise spectrum (an
Unruh-like effect).  This interaction dissipates kinetic energy into the
vacuum, imposing a resistance to acceleration:
\begin{equation}\label{eq:drag}
  P_{\rm diss}\propto\gamma^4 a^2.
\end{equation}
Balancing mechanical work against this dissipation yields ultra-relativistic
suppression $a(v)\propto\gamma^{-1}$ under sustained forcing.  Under extreme
acceleration---such as infall into a black hole---the drag dominates.  The
soliton sheds internal energy by shrinking.  However, it cannot shrink
indefinitely: it hits the \textbf{Window Floor}~($\Lp$), the effective Planck
length defined by the observer's coarse-graining scale.  This floor is not an
absolute limit of the substrate but of any particular observer's descriptive
capacity.

% ===================================================================
\section{The Black Hole as an Asymptotic Shell}

In this framework, a black hole is not a vacuum solution with a central
singularity.  It is a macroscopic object defined by a phase transition at the
horizon.

\subsection{Formation: The ``Snowplow'' Mechanism}
\label{sec:formation}
The formation of the shell is a runaway phase transition driven by the noise
gradient.
\begin{enumerate}
  \item \textbf{Infall and Acceleration:} Matter falls down the noise gradient
    $-\nabla\rho_N$, accelerating toward~$\cs$.
  \item \textbf{Drag and Scale Collapse:} The extreme acceleration invokes
    thermodynamic drag~\eqref{eq:drag}.  The soliton sheds energy by shrinking
    ($L_c\to\Lp$).
  \item \textbf{Decoupling:} As the soliton's scale collapses to the Planck
    floor, its cross-section for interacting with the macroscopic gradient
    vanishes.  It becomes ``blind'' to the force that created it.
  \item \textbf{Transition to Constituent:} The soliton is now a maximally
    downshifted particle moving at $v\approx\cs$.  From the external
    perspective, the effective radial velocity asymptotically vanishes at the
    horizon while internal time continues.
\end{enumerate}

\paragraph{The Snowplow Effect.}
Since constituents decouple from the macroscopic gradient, they pile up at the
transition surface.  The horizon acts as a ``snowplow,'' sweeping infalling
matter into a finite-thickness shell~$\Sigma$ rather than letting it fall to
$r=0$.

\paragraph{What about matter already inside $R_S$ during collapse?}
During dynamical collapse there is, of course, a period when ordinary
solitonic matter occupies radii that later lie inside the null trapping
surface.  The shell model does not claim that no worldline ever crosses
$r<R_S$; it claims that, \emph{after trapping forms and coarse-graining is
applied}, the only long-lived, externally accountable mass-energy resides in a
finite-thickness interface layer.  Two mechanisms enforce this:
\begin{itemize}
  \item \textbf{Phase separation:} A supercritical interior
    ($\tau_{\rm eff}\gtrsim\tau_{\max}$) is not a hospitable phase for amplitude
    solitons.  The region inside the evolving surface $\Sigma(t)$ rapidly
    relaxes to an optically thick, radiation-dominated bath.  Any remaining
    amplitude-soliton content is either thermalised or swept into the
    interface---exactly as quasiparticles accumulate at a moving phase
    boundary in a first-order transition.  A minimal coarse model treats the
    noise proxy as an order parameter:
    \begin{equation}\label{eq:snowplow_F}
      F[\tau]\sim\int d^3x\left[\frac{\kappa_\tau}{2}|\nabla\tau|^2
      + V(\tau;\rho_m)\right],
    \end{equation}
    with relaxational dynamics $\partial_t\tau\propto-\delta F/\delta\tau$.
    Increasing $\rho_m$ during collapse tilts $V$ so that the interior
    prefers a higher-$\tau$ state; $\Sigma(t)$ is the associated moving
    front.
  \item \textbf{Trapping bottleneck:} Once the refractive/metric trapping
    condition is reached ($\theta_{\rm out}\to0$), outward transport becomes
    exponentially inefficient in external time.  This pins the accumulating
    interface near $R_S$, yielding a stable shell/photosphere.
\end{itemize}

\paragraph{Geometric horizon vs.\ thermodynamic onset.}
These are adjacent but distinct:
\begin{itemize}
  \item \textbf{Geometric:} The event horizon is the null trapping surface
    ($\theta_{\rm out}=0$).  It governs photon escape and is purely
    kinematic.
  \item \textbf{Thermodynamic:} The Unruh-like heating raises the effective
    noise across the shell, driving irreversible downshifting over a finite
    microphysical band.  For massless excitations, only the geometric trapping
    applies; for massive solitons, both effects operate.
\end{itemize}

\subsection{Mean-Field Nuance: Screened Links vs.\ the Exterior Monopole}
\label{sec:mean_field}
An immediate worry is whether strictly local, screened interactions can
reproduce a clean exterior $1/r$ gravitational tail for distant observers.
The answer is yes: each Planck-scale constituent emits only a short-range
Yukawa halo $\sim e^{-r/R_i}/r$, but because $R_i\ll R_{\rm shell}$, the
exponential factor is effectively unity in the far field ($r\gg R_{\rm
shell}$).  Summing over $N$ constituents therefore reproduces a monopolar
$1/r$ tail with total mass $M=\sum_i E_i/\cs^2$, while inside the shell the
far-side contributions are exponentially suppressed and gradients cancel.

\paragraph{Same mass does not mean same noise profile.}
The above statement is about \emph{integrated source strength} (the monopole
fixing the $1/r$ tail), not about the detailed spectrum/correlation structure
of the emitted noise.  A maximally downshifted shell is ``window-limited'': most
microscopic fluctuations lie at or below the external UV edge and therefore
renormalise into effective couplings plus a monopole.  Ordinary dense matter
retains resolvable microstructure and can carry multipoles and longer
correlations.  See SI~Sec.~\ref{si:observability} for a more systematic
discussion of what observables probe beyond the scalar intensity $\tau^2(x)$.

\subsection{Equilibrium: The ``Breathing'' Shell}
\label{sec:breathing}
The shell is a self-contained, dynamic system in a remarkable equilibrium.

\paragraph{The Energy Balance.}
The two key thermodynamic processes are:
\begin{itemize}
  \item \textbf{Outward Cooling ($P_{\rm out}$):} The shell constituents form
    a near-2D, strongly coupled network.  Even when geometrically dilute
    ($d\gg\Lp$), its collective vibration modes radiate energy both inward
    (creating the blackbody core) and outward (as Hawking-like radiation).  The
    outward component escapes because the radiating photosphere sits (on
    average) just outside the geometric trapping surface; as the shell
    ``breathes,'' its optical depth fluctuates, modulating the escape fraction
    without violating null trapping.
  \item \textbf{Inward Heating ($P_{\rm in}$):} The non-geodesic proper
    acceleration of constituents (due to repeated shell collisions) allows them
    to absorb energy from vacuum fluctuations via the Unruh effect.
\end{itemize}
At equilibrium, $P_{\rm in}=P_{\rm out}$.  This balance maintains the shell's
temperature and stability.

\paragraph{Confinement.}
The shell is prevented from dispersing by the prohibitive energy cost of
constituents ``re-inflating'' in the lower-density noise field outside.  It is
prevented from collapsing by the immense outward radiation pressure from the
blackbody core it creates.

\paragraph{The Blackbody Core.}
The core of the black hole is not empty.  The hot inner wall of the shell fills
the interior with a thermal bath of phase solitons (photons), creating a
perfect blackbody cavity.  For an external observer, the cavity fills in a
light-crossing time $\sim R_S/\cs$.  A Planck-scale observer co-moving with a
constituent, however, measures vastly dilated clocks and shrunken rulers; the
same photon flight spans billions of local years.

\paragraph{The Temperature Paradox.}
An external observer calculates the core temperature to be near the Planck
temperature ($\sim10^{32}$~K), with radiation wavelengths near the Planck
length.  However, for a constituent-scale observer on the shell:
\begin{itemize}
  \item Their rulers are $\sim10^{20}\times$ smaller than ours.
  \item Planck-length photons appear as extremely long-wavelength radiation.
  \item The temperature they measure is $\sim10^{20}\times$ \emph{colder}
    than what we perceive.  ``Hot'' and ``cold'' are relative to the
    observer's scale.
\end{itemize}

\paragraph{Fate of Inward-Moving Matter.}
Any amplitude soliton that receives a thermal kick into the core is immediately
stopped by the extraordinarily dense radiation field.  Radiative drag (photon
friction from Doppler shifting and aberration) dissipates its kinetic energy
back into the bath.  Matter is effectively confined to the shell.

\paragraph{Covariant Power Balance.}
We can summarise the shell thermodynamics with a covariant, frame-consistent
closure at the level of power per proper area.  Let $\mathcal{P}_{\rm in}(R)$
be the areal Unruh heating rate and
$\mathcal{P}_{\rm out}(R)=\sigma_{\rm SB}T(R)^4$ the Stefan--Boltzmann
leakage.  At steady equilibrium:
\begin{equation}\label{eq:power_balance}
  \mathcal{P}_{\rm in}(R) = \mathcal{P}_{\rm out}(R) = P_0
  \quad\text{(constant per proper area)}.
\end{equation}
Deviations (anisotropy, binaries) enter as small, trackable modulations of
$\mathcal{P}_{\rm in}$.

\subsection{Shell Thermodynamics and the Temperature Law}
\label{sec:temp_law}
The shell's temperature is determined by the characteristic acceleration of its
constituents.  We derive the key scaling from two complementary viewpoints.

\paragraph{Shell Geometry.}
The shell contains $N\propto M$ constituents over area $A\propto R_S^2\propto
M^2$.  The mean separation is $d\sim\sqrt{A/N}\propto\sqrt{M}$.  For any
macroscopic black hole, $d\gg\Lp$: the shell is a \emph{dilute network}, not a
dense fluid.

\paragraph{View 1: External (Lattice Frequency).}
From outside, the shell is a 2D lattice of $N$ point masses with spacing
$d\propto\sqrt{M}$.  The characteristic dynamical frequency is set by
nearest-neighbour signal crossing: $\omega\sim\cs/d$, giving
$a_{\rm char}\sim\cs^2/d\propto1/\sqrt{M}$.

\paragraph{View 2: Internal (Relativistic Compression).}
From a constituent's own (downshifted) frame, the spacing $d$ maps to a much
larger proper distance $d_{\rm int}=d/\varepsilon$ (where $\varepsilon\ll1$ is
the scale ratio between the constituent's rulers and external rulers).  Signals
take a proportionally long proper time $\tau_{\rm int}\sim d_{\rm int}/\cs$ to
cross the gap.  However, because $\cs$ is invariant across frames, the external
observer maps this long internal proper time back to a short coordinate time:
$\tau_{\rm ext}=\varepsilon\cdot\tau_{\rm int}=d/\cs$.  The relativistic
compression that makes the internal ``miles'' appear as external ``metres'' is
\emph{exactly} compensated by the time dilation that makes the internal
``hours'' appear as external ``seconds.''  The ratio $d/\cs$ is
frame-invariant, so both views yield $a_{\rm char}\propto1/\sqrt{M}$.

\paragraph{Why they must agree.}
Both views compute the same invariant: the proper acceleration 4-vector
magnitude of a shell constituent.  Agreement is guaranteed by the framework's
requirement that $\cs$ is constant across observation windows.

Since $T_{\rm Unruh}\propto a_{\rm char}$, we obtain the temperature law:
\begin{equation}\label{eq:temp_law}
  T_{\rm BH}\propto\frac{1}{\sqrt{M}}.
\end{equation}
This differs from the standard Hawking result ($T\propto1/M$) because the
shell model treats the horizon as a physical lattice with specific
density-dependent dynamics, rather than a vacuum boundary condition.

\subsection{Stochastic Cascade and Interior Structure}
\label{sec:cascade}
The cavity is thermally uniform only on large scales.  On small scales
$\ell$, photon number fluctuations grow as $\ell^{-3/2}$.  At some scale
$\ell_\ast$, a rare upward fluctuation satisfies a pinch-off condition
$\Delta F<0$.  Defining the occupancy
\begin{equation}
  N(\ell)=\frac{\rho_\gamma\,\ell^3}{\Mp c^2},
\end{equation}
and comparing to the critical occupancy $\Nstar\approx1$:
\begin{itemize}
  \item $N(\ell)>\Nstar$: the fluctuation collapses into a
    \textbf{mini-shell} (black hole branch).
  \item $N(\ell)<\Nstar$: it relaxes into a \textbf{particle-type droplet}
    (matter branch).
\end{itemize}

\paragraph{Classification by Topology.}
Crucially, particle-branch droplets come in two populations:
\begin{enumerate}
  \item \textbf{Scalar solitons ($l=0$):} Simple density fluctuations with no
    phase winding.  These are collisionless and form the dominant population---identified with \emph{dark matter}.
  \item \textbf{Topological defects ($l\geq1$):} Rare fluctuations that trap a
    phase winding.  These experience phase-friction and become
    dissipative---identified with \emph{baryonic matter}.
\end{enumerate}
Combinatorial statistics in the cavity favour the simpler $l=0$ state by a
factor of~$\sim5$, naturally seeding the observed cosmic abundance ratio
$\Omega_{\rm DM}/\Omega_b\approx5$.

\paragraph{The Morphological Amplifier.}
Droplets clustered in a gravitational caustic source a higher local noise
background $\tau$.  This triggers a super-linear downshifting response in their
neighbours ($\varepsilon\propto-\tau^\gamma$, $\gamma>1$), deepening the
potential well and drawing in more matter.  This cooperative feedback transforms
weak Brownian caustics into sharp, high-contrast filamentary structures---the sponge-like cosmic web observed in our universe.

% ===================================================================
\section{Advanced Dynamics and Robustness}

\subsection{Prediction: Constant-Power Evaporation}
The Stefan--Boltzmann law gives the radiated power $P\propto A\cdot T^4$.
Using $A\propto M^2$ and \eqref{eq:temp_law}:
\begin{equation}\label{eq:const_power}
  P\propto M^2\cdot\left(\frac{1}{\sqrt{M}}\right)^{\!4}
  =M^2\cdot\frac{1}{M^2}=\text{constant}.
\end{equation}
\textbf{Prediction:} Black holes radiate at a constant power $P_0$ (of order
$10^{15}$~W for standard parameters) throughout their lives, until they reach
the Planck scale.  This implies a linear lifetime $\tau\propto M$---drastically
shorter for supermassive black holes than the Hawking prediction ($M^3$).  The
rising temperature ($1/\sqrt{M}$) and shrinking area ($M^2$) have a perfect,
compensatory relationship that keeps the total energy output constant.

\subsection{Duality of Stiffness}
\label{sec:stiffness}
A profound aspect of the particle--black hole duality lies in comparing their
mechanisms for resisting collapse.
\begin{itemize}
  \item \textbf{Microscopic:} A soliton is stabilized by a $|\Psi|^4$
    potential term.  Pressure $\propto|\Psi|^4$.
  \item \textbf{Macroscopic:} The shell is stabilized by radiation pressure
    $P_{\rm rad}\propto T^4$.  Since $T\propto\sqrt{\rhos}$ and we identify
    $\rhos\sim|\Psi|^2$, we find
    $P_{\rm rad}\propto(\sqrt{\rhos})^4=\rhos^2\sim|\Psi|^4$.
\end{itemize}
The macroscopic equation of state mirrors the microscopic potential.

\subsection{Window Rescaling and the Soliton Floor}
\label{sec:soliton_floor}
Does the constant-power argument hold for an observer with a finer microscope?
Yes.  The ``floor'' at $\Lp$ is not an absolute limit of the vacuum but the
limit of the \emph{soliton's} ability to downshift.

\paragraph{The Free-Energy Floor.}
The mechanism powering a soliton's response to increased noise is
\emph{scale downshifting}: higher $\tau_{\rm eff}$ $\to$ smaller equilibrium
size $\to$ energy released along $\varepsilon_s(\tau)=-\alpha_s\tau^\gamma$.
\begin{itemize}
  \item An ordinary soliton ($L_c\gg\Lp$) sits partway up the hill.  Plenty of
    downhill headroom remains.
  \item A shell constituent ($L_c\approx\Lp$) has already rolled to the
    \emph{bottom}.  Further noise cannot drive further shrinkage.
\end{itemize}
Crucially, this floor is \emph{relational}: $\Lp$ is the UV edge of the
external observer's window.  The substrate has no absolute minimum scale.  From
the constituent's own frame it has not ``bottomed out''---it sees a perfectly
normal window with a full spectrum and normal Unruh physics.  But all of this
self-consistent internal dynamics maps, via the sliding-window formalism, to
the same fixed power $P_0$ that the external observer computes.

\paragraph{Covariant Power Balance.}
Let $P_0$ be both the inward Unruh power and the outward Stefan--Boltzmann
power per unit proper shell area.  Lorentz factors cancel: every observer
measures the same $P_0$.  The constant-power evaporation law is therefore
frame-independent.

\subsection{Asymmetric Shells and Binary Recoil}
\label{sec:binary}
In a binary system, the shell of one black hole is distorted and heated
anisotropically by its companion's noise field.  This results in anisotropic
thermal radiation from the shell's outer surface, creating a continuous,
non-gravitational \textbf{recoil force}.  If the binary separation $d$ is less
than the ambient phase-coherence length $L_{\rm phase}$, coherent phase-sheets
can also form, adding an EM-like component to the interaction.  These effects,
including velocity-dependent retardation, could produce unique signatures in
gravitational wave signals.

% ===================================================================
\section{Duality with Cosmology}
\label{sec:duality}

\subsection{Strict Orientation-Duality}
We promote the horizon duality to a strict identification.  A single nucleation
criterion defines a closed surface $\Sigma$ that may be realised with two
orientations:
\begin{itemize}
  \item \textbf{Type~I (Black Hole):} Trapped side points inward; a
    finite-thickness, optically thick shell forms and reprocesses flux to a
    photosphere.  The observer is outside.
  \item \textbf{Type~II (Cosmology):} Trapped side points outward; to interior
    observers there is no material shell, but the same thermal bath appears
    with the Gibbons--Hawking temperature.
\end{itemize}
These are opposite orientations of the same nucleation event.  The interior of
a Type~I object (a black hole) \emph{is} a Type~II universe (a Hubble patch)
to its inhabitants.

\subsection{Type~I / Type~II dictionary (at a glance)}
\label{sec:type_dictionary}
\begin{center}
\begin{tabular}{p{0.24\textwidth}p{0.34\textwidth}p{0.34\textwidth}}
\toprule
\textbf{Item} & \textbf{Type I: Black hole (external view)} & \textbf{Type II: Cosmology (internal view)} \\
\midrule
Orientation of trapping & Trapped side points inward & Trapped side points outward \\
Material shell & Yes (finite-thickness reprocessor) & No privileged material wall for interior observers \\
Dominant temperature law (external units) & Photosphere set by shell microphysics: $T\propto M^{-1/2}$ & Gibbons--Hawking: $T\propto H$ \\
Net luminosity (external units) & $P_{\rm out}\approx \text{const}$ (reprocessed) & $P\propto H^2$ (bare geometric leakage) \\
Horizon radius identification & $R_S=2GM/\cs^2$ & $r_H=\cs/H$ \\
Duality lock (late era) & $R_S=r_H$ & $r_H=R_S$ \\
\bottomrule
\end{tabular}
\end{center}
\noindent Further dictionary items and robustness notes are collected in the SI (see
SI~Secs.~\ref{si:observability} and~\ref{si:time_mapping}).

\subsection{Energy Conservation and Hubble Radius}
The total mass-energy $M$ of the parent must equal the energy in the child's
Hubble volume.  For a flat FRW interior with density $\rho$ and Friedmann
equation $H^2=(8\pi G/3)\rho$:
\begin{equation}\label{eq:E_H}
  E_H = \tfrac{4}{3}\pi r_H^3\,\rho\,\cs^2
  = \frac{\cs^5}{2GH}.
\end{equation}
Setting $M\cs^2=E_H$:
\begin{equation}\label{eq:H_M}
  \boxed{H = \frac{\cs^3}{2GM},\qquad r_H = \frac{\cs}{H}
  = \frac{2GM}{\cs^2} = R_S.}
\end{equation}
The child's Hubble radius equals the parent's Schwarzschild radius---\emph{derived}, not assumed.

\paragraph{Local vs.\ Global Caveat.}
The identification $M\cs^2=E_H$ applies when the cavity is a single causal
domain ($\mathcal{N}\to1$; see SI~Appendix~\ref{si:time_mapping}).  During
earlier eras with many Hubble patches ($\mathcal{N}\gg1$), $H=\cs^3/(2GM)$
describes energy per patch, not total $M$.

\paragraph{Epistemic Consequence.}
Because $H(\tau)$ is locally uniform (FRW symmetry), an observer has no local
measurement revealing their distance to the wall.  The late-time phantom phase
($w<-1$) causes $r_H$ to \emph{shrink}, making new causal contact with the
wall harder, not easier.

\subsection{Flux Reconciliation}
\label{sec:flux}
A key apparent contradiction is the flux law:
\begin{itemize}
  \item \textbf{Type~I (BH):} $P\approx\text{const}$.
  \item \textbf{Type~II (Cosmo):} $P\propto H^2$ (Gibbons--Hawking).
\end{itemize}
These are reconciled by the shell.  The geometric (pre-reprocessing) intake on
both sides scales $\propto H^2$ via $A\propto H^{-2}$, $T\propto H$.  In the
Type~I case, the optically thick shell reprocesses this $H^2$ input to a
photospheric output with $T_{\rm shell}\propto M^{-1/2}$,
$A_{\rm shell}\propto M^2$, giving $P_{\rm shell}\approx\text{const}$.
Without a reprocessor (Type~II), the observer sees the bare $H^2$ leakage.

\subsection{Resolving Apparent Contradictions}
\begin{itemize}
  \item \textbf{Shell presence/absence:} Type~II lacks a material reprocessor
    for interior observers because they sit inside the asymptote; the same
    asymptote carries a real shell in the parent frame.  Both satisfy the same
    nucleation and balance laws.
  \item \textbf{Expansion vs.\ downscaling:} Metric expansion is traded for
    matter downscaling via the sliding window; horizon thermodynamics and
    trapping are invariant under this gauge.
  \item \textbf{Isotropy:} Hyperbolic optics makes the asymptote effectively
    concentric; leading observables (redshift, dimming) are isotropic to first
    order.  Observed anisotropies (CMB, filaments) arise from interior
    condensates.
\end{itemize}

\subsection{Boundary-condition imprint: an expected dipolar power modulation}\label{sec:cmb-dipole}
The duality picture above is \emph{not} committed to exact spherical symmetry.
In fact, a finite cavity with a physical boundary generically provides a
natural route to a \emph{dipolar} large-scale modulation, even when the mean
interior evolution is well approximated by FRW.

\paragraph{Key observation: the lowest nontrivial mode of a bounded domain is a dipole.}
Any scalar background field in the interior that is set, even weakly, by a
boundary condition at the parent shell (e.g.\ a windowed noise proxy $\tau$, a
refractive index $\chi$, or an effective ``clock''/scale factor field in the
sliding-window description) admits an expansion in eigenmodes of the interior
operator (Laplacian or wave operator, depending on regime).  In a simply
connected cavity that is close to spherical, the lowest non-constant harmonic
is the $l=1$ spherical harmonic: a dipole.  Therefore, \emph{any} small
departure from perfect symmetry at the boundary will preferentially populate an
$l=1$ component in the interior background, with higher multipoles suppressed
by additional gradients and by phase-bath smoothing.

\paragraph{Why symmetry is not exact.}
Exact isotropy would require an exactly spherical, perfectly homogeneous parent
shell and perfectly isotropic inflow/outflow.  In realistic parent dynamics,
small asymmetries are unavoidable (spin, anisotropic accretion, companion
perturbations; cf.\ Sec.~\ref{sec:binary}).  The ``breathing'' equilibrium and
hyperbolic interior optics tend to isotropize the \emph{mean} evolution, but
they do not forbid a small residual low-$l$ imprint.

\paragraph{From a boundary dipole to a CMB power dipole.}
In this framework, the interior radiation/matter bath inherits its statistics
from the windowed fluctuation environment.  If the relevant background quantity
that sets fluctuation amplitudes admits a weak dipole,
\begin{equation}
  \tau(\hat n) \;=\; \tau_0\,\big[1+\varepsilon_1\,(\hat p\!\cdot\!\hat n)\big],\qquad |\varepsilon_1|\ll 1,
\end{equation}
then any observable whose variance depends monotonically on $\tau$ will exhibit
a corresponding \emph{hemispherical} (dipolar) modulation.  A minimal
phenomenological representation for CMB temperature anisotropies is
\begin{equation}
  \Delta T(\hat n) \;=\; \big[1 + A\,(\hat p\!\cdot\!\hat n)\big]\;\Delta T_{\rm iso}(\hat n),
  \label{eq:dipole-modulation}
\end{equation}
where $\Delta T_{\rm iso}$ is a statistically isotropic field and $A$ is a
small modulation amplitude.  In the simplest local-variance picture one expects
$A=O(\varepsilon_1)$ up to an $\mathcal O(1)$ transfer coefficient that depends
on which windowed quantity controls the relevant primordial mode amplitudes.
The point is structural: a finite cavity with slightly imperfect boundary data
naturally supplies an $l=1$ ``super-mode'' that modulates power across the sky.

\paragraph{Scale dependence (why the effect is largest at low multipoles).}
Because the modulation originates as the lowest cavity mode, it is a
very-long-wavelength field.  Its primary effect is therefore on the largest
angular scales: low multipoles probe coherence across a substantial fraction of
the cavity, while higher-$\ell$ modes average over many modulation wavelengths
and become insensitive at leading order.  In practice this implies a
scale-dependent $A(\ell)$ that is largest for low $\ell$ and decays toward zero
at high $\ell$, consistent with how a boundary-conditioned dipole should appear
after projection through transfer functions.

\paragraph{Interpretation and limits.}
This mechanism does \emph{not} claim that a bulk observer can infer a unique
``direction to the wall'' (cf.\ the epistemic caveat above).  Rather, it says
that if the parent boundary conditions are not exactly symmetric, then a
dipolar modulation is the \emph{generic first} large-scale anisotropy to appear
in the interior statistics.  Whether the resulting $A$ is comparable to the
observed few-percent hemispherical power anomaly depends on quantitative
details of shell asymmetry, smoothing, and the mapping from $\tau/\chi$ to the
primordial fluctuation transfer function; these are left as open calculations.

% ===================================================================
\section{Cosmological Implications}
\label{sec:cosmo}

\subsection{Observational Anchors}
We estimate the parent mass using two numbers:
\begin{enumerate}
  \item Internal CMB temperature at last scattering:
    $T_{\rm int}\approx3000$~K.
  \item Internal look-back time to that epoch:
    $\tau_{\rm int}\approx13.7$~Gyr.
\end{enumerate}

\subsection{Parent Mass Estimate}
The interior--exterior conformal factor $F$ relates internal and external
clocks and temperatures:
\begin{equation}
  T_{\rm int}=\frac{T_{\rm BH}}{F},\qquad
  \tau_{\rm int}=F\,\tau_{\rm ext}=F\,\kappa M,
\end{equation}
where $\kappa\equiv\cs^2/P_0$ is the inverse evaporation rate (units:
s\,kg$^{-1}$) and $T_{\rm BH}=k/\sqrt{M}$ defines the proportionality
constant~$k$.  Eliminating $F$:
\begin{equation}\label{eq:parent_mass}
  M_{\rm parent}=\left(\frac{\tau_{\rm int}\,T_{\rm int}}{\kappa k}\right)^2.
\end{equation}
Using provisional Hawking-normalised constants yields
$M_{\rm parent}\sim10^{21}$~kg (comet scale), with an external lifetime of a
few years.  These are not final predictions but demonstrate self-consistency.

\subsection{The Hubble tension as window drift (operational $H_0$)}
\label{sec:hubble_tension}
In this framework the ``scale factor'' is not a fundamental stretching of an
absolute metric; it is the homogeneous component of the global-average
downshifting.  As a result, different observational pipelines need not measure
the same quantity when they implicitly use different unit conventions (different
``windows'').  The conversion factor $F$ above should therefore be treated as an
\emph{operational} mapping from interior FRW time $\tau$ to locally realised
clocks/rulers.

\paragraph{Two inferred $H_0$'s from one history.}
Let $a(\tau)$ denote the homogeneous downshift mode, and suppose the locally
realised unit system drifts via $F(\tau)$.  Then the expansion rate inferred by
a probe calibrated to local microphysics is schematically
\begin{equation}
  H_{\rm op}(\tau)\;\equiv\;\frac{d}{d\tau}\ln\!\big[a(\tau)\,F(\tau)\big]
  \;=\;H(\tau)+\frac{\dot F}{F}.
  \label{eq:H_operational}
\end{equation}
Early-universe inferences (CMB/BAO) effectively fix the physical ruler at
recombination and propagate it to $\tau_0$ assuming a time-independent mapping,
returning an $H_0$ close to $H(\tau_0)$.  Late-time distance ladders
(Cepheids/SNe), by contrast, are calibrated directly in the present-epoch window
and return $H_0$ closer to $H_{\rm op}(\tau_0)$.  The observed ``Hubble tension''
is then naturally interpreted as a measurement of the late-time drift term:
\begin{equation}
  \Delta H_0 \;\equiv\; H_0^{\rm ladder}-H_0^{\rm CMB}
  \;\approx\;\left.\frac{\dot F}{F}\right|_{\tau_0}.
  \label{eq:H0_gap}
\end{equation}
\noindent A minimal one-parameter fit model is to tie the drift to the onset of
the late-time, globally constrained regime:
\begin{equation}
  \frac{\dot F}{F}(\tau)\;\approx\;\eta\,H(\tau)\,S(\tau),\qquad
  S(\tau)\equiv\frac{1}{2}\!\left[1+\tanh\!\left(\frac{\tau-\tau_\ast}{\Delta\tau}\right)\right],
  \label{eq:F_drift_ansatz}
\end{equation}
where $\tau_\ast$ is the DE onset time (when $\mathcal{N}\to1$ and the lock
becomes operative), $\Delta\tau$ controls how sharp the turn-on is, and the
single dimensionless amplitude $\eta$ is fixed phenomenologically by the gap as
$\eta\sim\Delta H_0/H_0$ when $S(\tau_0)\approx1$.

\paragraph{Why the drift turns on late.}
The multi-era accounting (Sec.~\ref{sec:multi_era}) implies that the interior
homogeneous mode becomes globally constrained only as $\mathcal{N}\to1$, at
which point the duality lock \eqref{eq:H_M} and the constant-power parent
evolution feed directly into the late-time $H(\tau)$ history.  This is also the
era in which the framework requires an effective phantom phase
($w_{\rm eff}<-1$; Eq.~\ref{eq:phantom}), i.e.\ a regime where $H(\tau)$ grows and
the mapping between microphysical units and FRW time becomes maximally
sensitive to the boundary-conditioned window dynamics.

\paragraph{A clean discriminator: standard sirens.}
Gravitational-wave ``standard sirens'' measure luminosity distances through the
waveform amplitude and phase evolution, without relying on stellar calibrators.
They therefore provide an operational route closer to $H(\tau)$ than to
microphysics-calibrated $H_{\rm op}(\tau)$.  As the statistics improve, this
framework predicts siren-inferred $H_0$ should converge toward the CMB-inferred
value if the tension is dominated by the drift term in \eqref{eq:H0_gap}.

\paragraph{Phenomenological target and falsification.}
Taking the observed gap as a fitting target,
$H_0^{\rm ladder}\simeq 73$ and $H_0^{\rm CMB}\simeq 67$ km\,s$^{-1}$\,Mpc$^{-1}$,
implies a late-time drift of order
\begin{equation}
  \left.\frac{\dot F}{F}\right|_{\tau_0}\sim \Delta H_0 \sim 0.1\,H_0,
\end{equation}
i.e.\ a few-to-ten percent effect in the operational mapping over a Hubble time.
This interpretation is falsifiable as next-generation geometric measurements
reach percent-level precision:
\begin{itemize}
  \item \textbf{Standard sirens:} if BNS/BBH siren-inferred $H_0$ converges to
    the \emph{ladder} value at $\lesssim1$--$2\%$ precision (rather than to the
    CMB value), the drift interpretation \eqref{eq:H0_gap} is disfavoured.
    Conversely, convergence toward the CMB value supports a genuinely
    microphysics-calibration origin of the tension.
  \item \textbf{Redshift dependence:} because the drift is tied to the late
    $\mathcal{N}\to1$ and phantom regime, it should be a \emph{late-time}
    phenomenon.  High-$z$ geometric reconstructions (e.g.\ inverse distance
    ladder using BAO+SNe) should therefore approach $H(\tau)$ and reduce the
    discrepancy relative to local calibrators.
  \item \textbf{Other geometric clocks/rulers:} strong-lens time delays and
    redshift-drift measurements provide complementary, largely microphysics-free
    routes to the late-time expansion history.  Agreement of these with the
    ladder value at percent precision would disfavor a dominant $F$-drift
    explanation, while agreement with the CMB inference would support it.
\end{itemize}
\noindent In short: if multiple independent geometric probes (siren, lensing,
redshift drift) converge to the ladder value at percent precision, then the
tension cannot be primarily an $F$-drift effect and must reflect either early
physics beyond the minimal mapping or residual ladder systematics.

% ===================================================================
\section{Dynamics: Time Mapping and the Heat Death}
\label{sec:dynamics}

A critical test of the duality is the synchronization of clocks.

\subsection{The Time Mapping Derivation}
\label{sec:time_map}
From $H=\cs^3/(2GM)$ and constant-power evaporation
$M(t_p)=M_0-(P/\cs^2)\,t_p$, we derive the parent-to-child clock ratio
(detailed in SI~Appendix~\ref{si:time_mapping}):
\begin{equation}\label{eq:clock_ratio}
  \frac{dt_p}{d\tau}=-\frac{3\cs^5}{4GP_0}(1+w_{\rm eff}).
\end{equation}
For both clocks to run forward ($dt_p/d\tau>0$):
\begin{equation}\label{eq:phantom}
  1+w_{\rm eff}<0\implies\boxed{w_{\rm eff}<-1.}
\end{equation}
\textbf{Result:} The framework \emph{requires} a phantom equation of state
during the DE era.  Current observational constraints ($w\approx-1.03\pm0.03$)
are consistent with---and mildly favour---this prediction.

\subsection{Multi-Era Accounting}
\label{sec:multi_era}
During radiation and matter eras, the cavity contains $\mathcal{N}\gg1$
causally disconnected Hubble patches.  The total BH mass distributes as
$M\cs^2=\mathcal{N}\cdot E_H$.  As the Hubble sphere grows into the cavity,
$\mathcal{N}$ shrinks until $\mathcal{N}\to1$ at DE onset, when
\eqref{eq:H_M} locks in.  The early-universe evolution is effectively
\emph{decoupled} from the parent's evaporation (see
SI~Appendix~\ref{si:time_mapping} for details).

\paragraph{Meaning of $\mathcal{N}\to1$ (and what it is \emph{not}).}
Here $\mathcal{N}$ is a bookkeeping measure of how many \emph{effectively
independent} Hubble-sized domains can coexist inside the cavity for the purpose
of global energy accounting and for whether the boundary condition can be
treated as a single coherent constraint on the interior's homogeneous mode.
Thus $\mathcal{N}\to1$ should be read as: the dynamics relevant to the
homogeneous downshift/clock field is controlled by a single coupled domain.
It is \emph{not} the claim that every galaxy remains in two-way signal contact.
In an accelerating (and a fortiori phantom) era, the Hubble radius
$r_H=\cs/H$ is not a hard causal horizon, and a cosmological event horizon can
exist even in a globally FRW solution.  Objects crossing $r_H$ therefore do not
imply $\mathcal{N}>1$; they reflect late-time acceleration, not the presence of
multiple independent interior domains.

\subsection{Predicted Value of $\epsilon=|1+w|$}
From \eqref{eq:clock_ratio}, $P\sim(\cs^5/G)\cdot f$ where $f$ is a
dimensionless microphysical factor.  Then $dt_p/d\tau\sim3\epsilon/(4f)$.
For the child to have a long lifetime ($dt_p/d\tau\ll1$), we need
$\epsilon\ll f$.  If $f\sim\mathcal{O}(1)$ (natural for Planck-scale
microphysics):
\begin{equation}
  \boxed{|1+w|=\epsilon\ll1.}
\end{equation}

\subsection{The Endpoint: Asymptotic Heat Death}
\label{sec:heat_death}
If $P_0$ and $\epsilon$ remained constant, the child would end in a Big Rip
($H\to\infty$ in finite time $\tau_{\rm end}=\tau_\ast+2/(3\epsilon H_\ast)$).
However, the Planck pivot intervenes:
\begin{enumerate}
  \item \textbf{Parent Side:} As $M\to\Mp$, evaporation power switches to
    $P\propto M^4$ (quantum regime), taking infinite time to reach $M=0$.  A
    Planck-mass remnant forms.
  \item \textbf{Child Side:} As $H\to H_P\equiv\cs/(2G\Mp)$, the Hubble
    volume shrinks to a single Planck cell.  Window drift runs out of UV
    headroom, forcing $\epsilon\to0$.  The simplest smooth parameterisation:
    \begin{equation}
      \epsilon(H)\approx\epsilon_0\!\left(1-\frac{H}{H_P}\right).
    \end{equation}
\end{enumerate}
With this $\epsilon(H)$ and $\dot{H}=\frac{3}{2}\epsilon H^2$, linearising
near $H_P$ ($x\equiv1-H/H_P\ll1$):
\begin{equation}
  \dot{x}\approx-\frac{3\epsilon_0 H_P}{2}\,x
  \quad\Longrightarrow\quad
  x(\tau)=x_0\,e^{-\frac{3\epsilon_0 H_P}{2}\tau}.
\end{equation}
$H$ approaches $H_P$ exponentially, taking \emph{infinite} child cosmic time.
The Big Rip is replaced by an \textbf{Asymptotic Heat Death}: the child
universe dilutes toward an empty de Sitter state at $H=H_P$, approached but
never reached.

% ===================================================================
\section{The Infinite Hierarchy}
\label{sec:hierarchy}

The model implies a fractal structure: universes nested within black holes,
nested within larger universes.

\subsection{Impedance Matching}
Standard Hawking radiation ($P\propto M^{-2}$) forbids nesting: a small inner
shell would vaporize its parent (``infrared explosion'').
\textbf{Constant power ($P\approx\text{const}$)} is the unique scaling law
that allows impedance matching: $P_{\rm inner}=P_{\rm outer}=P_0$.  The inner
shell feeds the outer at exactly the rate the outer dissipates, permitting
indefinite stable nesting.

\subsection{Piecewise Arrows of Time}
The infinite hierarchy yields a natural, local arrow of time at every level:
\begin{itemize}
  \item \textbf{Local definition:} Within each cavity, the coarse-grained free
    energy has a monotone scale direction: $\dot\sigma<0$ (solitons downshift).
    Thermodynamic dissipation (constant-power leakage) makes this irreversible.
  \item \textbf{Finite durations per level:} Constant-power emission drains
    each cavity's UV reserve in finite local time, defining a local past
    boundary (entry from parent shell) and future boundary (Planck de Sitter
    asymptote).
  \item \textbf{No global beginning or end:} The hierarchy extends unboundedly
    in both directions.  Concatenating piecewise arrows yields an eternal,
    globally past-and-future-unbounded structure with locally well-defined time
    everywhere.
  \item \textbf{Consistency with entropy:} At each level, coarse-grained
    entropy increases along the downshift.  UV exhaustion at a shell resets the
    window and renormalises couplings.  There is no global maximum entropy.
\end{itemize}

\subsection{No Global Time Paradox}
The hierarchy is eternal despite finite local lifetimes.  The parent-to-child
clock ratio from \eqref{eq:clock_ratio} gives
$r\equiv dt_{\rm parent}/d\tau_{\rm child}\ll1$.
\begin{itemize}
  \item \textbf{Looking Down:} A level-$n$ observer hosts an arbitrarily deep
    cascade inside its own finite lifetime, because the projected durations
    form a convergent geometric series: $\sum_{k\geq1}r^k<\infty$.
  \item \textbf{Looking Up:} Projected upward, parent lifetimes scale by
    $\sim1/r>1$ per level.  The upward chain diverges: there is no ``top''
    level with a finite end.
\end{itemize}
Finite proper time at every level is compatible with an eternal hierarchy
because ``eternity'' lives in the unbounded tower of scales and the compounding
inter-level time dilation, not in an infinite duration of any single clock.

% ===================================================================
\section{Conclusion}

We have presented a self-consistent model where black holes are material shells
and their interiors are expanding universes.  The framework replaces the
singularity with a blackbody cavity and standardizes the thermodynamics of
horizons across scales.

Key falsifiable predictions include:
\begin{enumerate}
  \item \textbf{Constant Power Evaporation:} Black holes radiate at
    $P\approx\text{const}$, not $P\propto1/M^2$.
  \item \textbf{Phantom Dark Energy:} The equation of state must be slightly
    phantom ($w\lesssim-1$) with $|1+w|\ll1$.
  \item \textbf{Binary Signatures:} Anisotropic shell heating produces
    non-gravitational recoil forces detectable in precision waveform analyses.
  \item \textbf{Shell Thickness:} A finite photosphere with
    $\delta R/R\propto M^{-1/2}$, potentially observable in horizon-scale
    imaging.
\end{enumerate}

This duality suggests that our Big Bang was the formation of a shell in a
parent universe, and our eventual heat death corresponds to that shell's
relaxation to a Planck remnant.  The cosmos is an infinite, self-similar
cascade of such events, eternal and devoid of singularities.

% ===================================================================
\end{document}
