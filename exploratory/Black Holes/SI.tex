\documentclass[11pt]{article}

% --- Packages (match main paper) ---
\usepackage[a4paper,margin=1in]{geometry}
\usepackage{amsmath,amssymb,amsfonts,amsthm}
\usepackage{mathtools}
\usepackage{graphicx}
\usepackage{xcolor}
\usepackage{microtype}
\usepackage{booktabs}
\usepackage{enumitem}
\usepackage{fancyhdr}
\usepackage{authblk}
\usepackage{cite}
\usepackage{bm}
\usepackage[colorlinks=true,linkcolor=blue,citecolor=blue,urlcolor=blue]{hyperref}
\usepackage{xr-hyper}
\externaldocument{paper}

% --- Draft Status Footer ---
\pagestyle{fancy}
\fancyhf{}
\fancyfoot[L]{\small Draft Supplementary Information — \today}
\fancyfoot[R]{\thepage}
\renewcommand{\headrulewidth}{0pt}
\renewcommand{\footrulewidth}{0pt}

% --- Macros (keep consistent with main) ---
\newcommand{\cs}{c_s}
\newcommand{\Lp}{L_{\min}}
\newcommand{\Mp}{M_P}
\newcommand{\Nstar}{N_\ast}
\newcommand{\rhos}{\rho_{\rm surf}}
\newcommand{\dd}{\mathrm{d}}

\title{Supplementary Information for\\
\textit{Horizon Duality: Black Holes as Asymptotic Shells of a Fractal Cosmos}}
\author{Franz Wollang}
\date{\today}

\begin{document}
\maketitle

\begin{abstract}
\noindent This document collects technical derivations, robustness arguments,
and extended quantitative sketches supporting the main paper.  It is intended
to preserve detail while keeping the narrative flow of the main text clean.
\end{abstract}

\tableofcontents
\vspace{1em}\hrule\vspace{1em}

\section{Notation and cross-document conventions}
\label{si:notation}
We standardize on $\cs$ for the universal phase-wave/light-cone speed (any bare
$c$ should be read as $\cs$).  In weak-field regimes we relate the local proxy
to cosmological noise density by $\tau^2\propto\rho_N$.  We reserve
$(\alpha_{\rm grad},\beta_{\rm pot})$ for amplitude-sector parameters and
$(\alpha_{\rm dim},\beta_{\rm dim})$ for dimensional-bias parameters to avoid
symbol overloading.

% ===================================================================
\section{Mean-field nuance: screened links, monopoles, and observability}
\label{si:observability}

\subsection{Yukawa kernels vs.\ the exterior point-source tail}
Every Planck-scale constituent in the shell emits only a short-range Yukawa
halo,
\begin{equation}
  \Phi_i(r)\sim \frac{e^{-r/R_i}}{r},
\end{equation}
with screening length $R_i\ll R_{\rm shell}$.  For an exterior observer at
$r\gg R_{\rm shell}$, each exponential factor is $\approx 1$, so the sum over
$N$ constituents reproduces an ordinary monopole $1/r$ field with total
source strength
\begin{equation}
  M=\sum_i \frac{E_i}{\cs^2}.
\end{equation}
Inside the shell, far-side halos are exponentially suppressed, so the local
gradient vanishes to leading order.  Thus a strictly local link interaction
can yield a point-source exterior field for distant observers while producing
no further pull on already-collapsed constituents.

\subsection{``Same mass'' does not mean ``same noise profile''}
The statement ``the fine-grained noise averages to a point source'' concerns
integrated source strength (the monopole fixing the $1/r$ tail), not the
detailed spectrum or correlation structure of the emitted field.

A black-hole shell is a maximally downshifted, window-floor system: most of its
microscopic fluctuations sit at or below the external UV edge and are
unresolved by external observers.  They renormalise into effective couplings
plus a monopole.  Ordinary dense matter (e.g.\ a neutron star) is not
maximally downshifted: it retains internal structure and fluctuations at
wavelengths well within our window, so its noise field can carry resolvable
multipoles, longer correlation lengths, and material-specific channels.

\subsection{When does the profile matter observationally?}
If an effect couples only through the scalar intensity $\tau^2(x)$ (a
window-integrated variance), then two sources with very different spectra can
be indistinguishable at leading order: the spectrum is ``forgotten'' by the
integral and reappears only as renormalised constants plus a monopole.

Differences become measurable only for probes sensitive to more than $\tau^2$:
\begin{itemize}
  \item correlation functions (finite correlation length/time),
  \item higher moments (non-Gaussianity; nonlinear response),
  \item multipole/anisotropy structure,
  \item frequency-selective or window-overlap probes (including accelerated or
    near-shell observers where Doppler/Unruh promotion changes which portion of
    the spectrum falls inside the operative window).
\end{itemize}

\section{Duality dictionaries, membrane balance, and discriminants}
\label{si:duality_dictionaries}

This section ports the longer ``dictionary'' and robustness material behind
the Type~I/Type~II horizon duality.

\subsection{Geometric dictionary and trapping}
\label{si:geom_dictionary}
\begin{itemize}
  \item \textbf{Black hole:} $r_H^{\rm BH}$ is the null trapping radius of the
    exterior effective metric; interior optics is hyperbolic.  Define an
    interior curvature scale $H_{\rm in}:=\cs/R_{\rm shell}$.
  \item \textbf{Cosmology:} $r_H^{\rm FRW}=\cs/H$ is the apparent horizon of the
    FRW patch.
  \item \textbf{Areas:} $A_{\rm in}=4\pi R_{\rm shell}^2=4\pi(\cs/H_{\rm in})^2$
    and $A_{\rm FRW}=4\pi(\cs/H)^2$.
  \item \textbf{Trapping:} trapping holds where outward null expansion first
    vanishes, $\theta_{\rm out}=0$.  Emission is set by a photosphere/last
    scattering layer where optical depth to infinity is $\sim 1$.
\end{itemize}

\subsection{Thermodynamic dictionary and flux reconciliation}
\label{si:thermo_dictionary}
\begin{itemize}
  \item \textbf{Temperatures:} interior BH side $T_{\rm in}\propto H_{\rm in}$;
    FRW horizon $T_{\rm GH}=\hbar H/(2\pi k_B)$.
  \item \textbf{Geometric (pre-reprocessing) fluxes in external units:}
    \begin{align}
      P_{\rm in,geom} &\propto A_{\rm in}\,T_{\rm in}^4 \propto (1/H_{\rm in}^2)\,H_{\rm in}^4 = H_{\rm in}^2,\\
      P_{\rm FRW} &\propto A_{\rm FRW}\,T_{\rm GH}^4 \propto (1/H^2)\,H^4 = H^2.
    \end{align}
  \item \textbf{Reprocessing (Type~I only):} an optically thick shell enforces
    $T_{\rm shell}\propto M^{-1/2}$, $A_{\rm shell}\propto M^2$, giving a
    photospheric output $P_{\rm shell}\propto A_{\rm shell}T_{\rm shell}^4\approx
    \text{const}$ in external units.
\end{itemize}

\subsection{Membrane balance and conservation}
\label{si:membrane_balance}
On the nucleation surface $\Sigma$ we impose local balance in window units,
\begin{equation}
  P_{\rm in}(\Sigma)=P_{\rm out}(\Sigma),\qquad \nabla_\mu T^{\mu\nu}_{\rm tot}=0,
\end{equation}
with $T^{\mu\nu}_{\rm tot}$ including membrane stress and radiative flux.  For
Type~I, this closes with a diffusive transport law through the shell to the
photosphere; for Type~II, detailed balance holds locally (GH bath) and the net
leakage seen in parent/external units is $\propto H^2$.

\subsection{Robustness: resolving apparent contradictions}
\label{si:duality_robustness}
\begin{itemize}
  \item \textbf{Shell vs.\ no shell:} Type~II lacks a material reprocessor for
    interior observers because they sit inside the asymptote; the same asymptote
    carries a real shell in the parent frame.  Both satisfy the same nucleation
    and balance laws.
  \item \textbf{Flux laws:} ``const'' (Type~I photosphere) vs.\ $H^2$ (Type~II GH
    leakage) reflect reprocessing vs.\ bare geometric emission; they are two
    faces of the same intake flux.
  \item \textbf{Isotropy vs.\ structure:} the parent boundary supplies leading
    near-FRW isotropy (hyperbolic optics with a concentric asymptote).  Observed
    anisotropies arise from interior condensates seeded by the Planck-pivot
    cascade.
\end{itemize}

\subsection{Predictions and discriminants}
\label{si:discriminants}
\begin{itemize}
  \item Finite shell thickness at the asymptote; hyperbolic interior optics.
  \item Luminosity laws: $P_{\rm shell}\approx\text{const}$ (Type~I photosphere)
    vs.\ $P_{\rm FRW}\propto H^2$ (Type~II leakage) in a single external frame.
  \item \textbf{Spectral/size imprint:} a last-scattering/photosphere radius
    produces a transfer-function feature (deviation from a perfect Planckian)
    differing between Type~I/II by the presence/absence of reprocessing.
\end{itemize}

\subsection{Species and multipoles in the heating closure}
\label{si:species_multipoles}
In the main text we use a coarse covariant closure in which Unruh-like heating
balanced by Stefan--Boltzmann cooling sets the scaling $T\propto 1/\sqrt{M}$ and
thus $P\approx\text{const}$.  This closure is meant as an \emph{effective}
description: ``areal heating'' is an average over the shell's composite
single-field solitonic constituents and their worldline multipoles/topological
textures; no Standard-Model point-particle assumption is required.

In an EFT language, the lowest symmetry-allowed gradient coupling to the
massless phase fixes the baseline exponent ($p=4$).  Species that suppress this
operator contribute higher-$p$ channels with reduced weight.  The mixture
renormalises only an effective coefficient (schematically $\overline{C}$);
the scaling exponents, and thus the constant-power result, are unchanged.

% ===================================================================
\section{Statistical derivation of the Planck pivot}
\label{si:planck_pivot}

We show that the four-power flip in surface density arises from a single
coarse-grained free-energy functional.

\subsection{Free energy for an $N$-grain droplet}
Consider a spherical droplet of radius $R$ containing $N$ identical grains
(bare mass $\mu$).  A minimal rotationally symmetric free energy is
\begin{equation}\label{eq:si_droplet_F}
  F(R;N)=\underbrace{\frac{\alpha N\hbar c}{R}}_{\text{zero-point / kinetic}}
  +\underbrace{\frac{\eta_0}{2}\frac{N^2}{R^3}}_{\text{link (cohesion) energy}}
  -\underbrace{\frac{G(N\mu)^2}{R}}_{\text{self-gravity}},
\end{equation}
where $\alpha$ is an $\mathcal{O}(1)$ geometry factor.

\subsection{Critical occupancy}
Setting $\partial F/\partial R=0$ gives a physical extremum only if
\begin{equation}
  G\mu^2 N^2=\alpha N\hbar c
  \quad\Longrightarrow\quad
  \Nstar=\frac{\alpha\hbar c}{G\mu^2}.
\end{equation}
The corresponding mass is $M_\ast=\Nstar\mu\simeq\sqrt{\hbar c/G}$ (up to
$\sqrt\alpha$), i.e.\ the Planck mass.

\subsection{Two minimisers $\Rightarrow$ two area laws}
\begin{itemize}
  \item \textbf{Quantum regime ($N<\Nstar$):} kinetic + link balance gives
    $R\propto 1/N$, so $A\propto 1/N^2$ and $\rho_{\rm surf}\propto N^3$.
  \item \textbf{Gravitational regime ($N>\Nstar$):} link + gravity balance gives
    $R\propto N$, so $A\propto N^2$ and $\rho_{\rm surf}\propto 1/N$.
\end{itemize}
The mass-dependence of area flips by four powers at $\Nstar$.  Temperature
inherits this via $T\propto\sqrt{\rho_{\rm surf}}$:
particles get hotter with mass ($T\propto m^{3/2}$) while black holes get
colder ($T\propto 1/\sqrt{M}$).

\subsection{Conformal consistency}
Because both $\gamma$ (micro) and $G$ (macro) arise from the same normalised
links, they renormalise with ambient noise in lock-step: $G,\hbar\propto\rho_N$.
Their ratio, and hence the pivot mass, is invariant in every conformal frame.

\subsection{Cross-window consistency and re-windowing covariance (why the crossover shape is not arbitrary)}
\label{si:rewindowing}
The ``window'' language in the main paper is doing more than bookkeeping.  It
functions as the EFT principle that \emph{overlapping descriptions must agree
on the same physical story} in their overlap domain.  We make that implicit
constraint explicit here and explain what extra assumption would be needed to
promote a qualitative ``smooth crossover'' into a more unique functional form.

\paragraph{Scope note (assumption used for the pivot closure).}
In the present work we will \emph{assume} the strong form of this principle
\emph{locally near the Planck pivot}, i.e.\ in the regime where the
particle-like and shell-like windows most directly overlap and must patch.
This should be read as a \emph{closure assumption} (to be justified more
microscopically by an explicit RG treatment of the coarse-grained free energy
and its beta functions), not as a theorem proven here.
The motivation is internal: (i) the framework treats windows as coarse-grained
descriptions of one underlying graph/free-energy dynamics (no privileged
window), so the overlap region should admit a fixed-form universality-class
description; and (ii) without such a fixed-form assumption, the crossover
shape contains an essentially arbitrary interpolation freedom that cannot be
constrained by the two asymptotics alone.  Away from the pivot, we allow
finite-window/non-adiabatic corrections that can spoil exact covariance while
preserving the weaker overlap-consistency constraints.

\paragraph{Cross-window consistency (implicit, already used).}
Let $\mathcal{W}_1$ and $\mathcal{W}_2$ be two coarse-grained descriptions
(two observers, or the same observer using two different effective resolutions)
whose domains of validity overlap.  The framework repeatedly relies on the
requirement:
\begin{quote}
  \emph{In the overlap, $\mathcal{W}_1$ and $\mathcal{W}_2$ must agree on
  invariants and on the sign/ordering of physical effects (existence of a
  pivot, direction of monotonic trends, absence of catastrophic runaways),
  even if they parametrise them with different variables and renormalised
  couplings.}
\end{quote}
This is the hidden logic behind several ``no-go'' arguments in the main text
(e.g.\ the impossibility of extending one asymptotic scaling law through the
domain where a neighbouring window would predict the opposite trend).

\paragraph{Exact re-windowing covariance (a natural strengthening).}
The natural next tightening is to demand not just agreement in overlaps, but
fixed \emph{form} under change of window:
\begin{quote}
  \emph{Under re-windowing (coarse-graining), the effective equations retain
  the same functional form; only parameters/couplings run.}
\end{quote}
This is an RG-style covariance principle.  It is not strictly required for the
Planck pivot to exist, but it is consistent with the fractal/nested ethos: no
particular window is ontologically privileged, so the effective laws should
live in a universality class stable under re-windowing.

\paragraph{What covariance does \emph{not} fix by itself.}
Even exact covariance does not automatically single out one unique
interpolating curve for $T(M)$; it fixes a \emph{class} of allowed smooth
crossovers.  To pick a specific shape one needs an additional structural
closure---for example an involutive ``UV$\leftrightarrow$IR'' mapping (a
self-duality-like constraint) or an explicit rule for how independent
contributions combine in the coarse-grained free energy.

\paragraph{A minimal closure that yields a ``hyperbolic bounce''.}
One economical way to encode ``saturation + minimisation'' is to model the
effective hardness/drive as two competing channels (particle-like localisation
and shell-like extensification) that combine by a harmonic-mean rule.  For
instance, if a proxy $X(M)$ satisfies
\begin{equation}
  X_{\rm part}(M)\propto M^3,
  \qquad
  X_{\rm BH}(M)\propto \frac{1}{M},
\end{equation}
then the simplest smooth interpolant that reproduces both limits without
introducing a new scale is
\begin{equation}\label{eq:si_harmonic_mean_X}
  \frac{1}{X(M)}=\frac{1}{X_{\rm part}(M)}+\frac{1}{X_{\rm BH}(M)}.
\end{equation}
If one takes $X\propto T^2$ (as suggested by the local scaling
$T\propto\sqrt{\rho_{\rm surf}}$), Eq.~\eqref{eq:si_harmonic_mean_X} yields a
single smooth maximum near the equality point $X_{\rm part}=X_{\rm BH}$, i.e.\
near the pivot $M\sim\Mp$.  In this sense the ``bounce'' is not an arbitrary
curve-fit; it is the minimal analytic closure that (i) respects the two
asymptotics, (ii) encodes gradual saturation of marginal localisation benefit,
and (iii) is stable under reparametrisation across overlapping windows.

\paragraph{What a true self-duality would add (optional, stronger).}
If one further postulates an \emph{involutive} re-windowing map exchanging the
two sides, e.g.\ a reciprocity $M\mapsto \Mp^2/M$ (or an equivalent statement
about core vs.\ shell measures), then the pivot becomes a fixed point of a
symmetry rather than merely a crossover.  Such a requirement can further
restrict the allowed interpolants, often selecting additive/quadrature forms
as the simplest functions compatible with the involution.  The present paper
does not assume this stronger constraint; it is best viewed as a plausible
next axiom if one wishes to make the crossover \emph{more} unique than the
generic smooth class already implied by windowing + minimisation.

% ===================================================================
\section{Parent--child time mapping derivation}
\label{si:time_mapping}

This appendix derives the quantitative relationship between the parent black
hole's evaporation and the child universe's cosmological evolution.

\subsection{Notation}
\begin{center}
\begin{tabular}{ll}
\toprule
$t_p$ & Parent coordinate time (external ``lab'' frame) \\
$\tau$ & Child cosmic time (interior FRW proper time) \\
$M(t_p)$ & Parent BH mass; $M(t_p)=M_0-(P/\cs^2)\,t_p$ \\
$H(\tau)$ & Child Hubble parameter \\
$P$ & Constant evaporation power \\
$t_e=M_0\cs^2/P$ & Total parent evaporation time \\
\bottomrule
\end{tabular}
\end{center}

\subsection{Derivation of $r_H=R_S$}
From the Friedmann equation $H^2=(8\pi G/3)\rho$ and the energy in a Hubble
volume $E_H=(4\pi/3)r_H^3\rho\cs^2=\cs^5/(2GH)$, setting $M\cs^2=E_H$ yields
\begin{equation}
  H=\frac{\cs^3}{2GM},\qquad r_H=\frac{2GM}{\cs^2}=R_S.
\end{equation}

\subsection{Junction consistency}
For a quasi-static shell separating Schwarzschild exterior from FRW interior,
the Israel junction condition reads
\begin{equation}
  \beta_+-\beta_-=-\frac{4\pi G\sigma R}{\cs^4},\qquad
  \beta_\pm=\sqrt{1-\frac{2GM}{\cs^2 R}}\;\text{or}\;
  \sqrt{1-\frac{H^2R^2}{\cs^2}}.
\end{equation}
At $R=R_S=r_H$ we have $\beta_+=\beta_-=0$, giving $\sigma=0$.  A tension-free
shell is consistent with the trapping surface coinciding on both sides.

\subsection{Multi-era accounting}
During radiation and matter eras the cavity contains
\begin{equation}
  \mathcal{N}=\left(\frac{R_{\rm cavity}}{r_H}\right)^3\gg 1
\end{equation}
causally disconnected Hubble patches, so the total mass distributes as
$M\cs^2=\mathcal{N}\,E_H$.  As the Hubble sphere grows
($r_H\propto \tau$ in radiation, $\propto\tau^{2/3}$ in matter),
$\mathcal{N}\to 1$ at DE onset.  In the early phase, the two extreme time
dilations (exterior ``frozen star'' and interior horizon slowdown) cancel,
yielding $dt_p/d\tau|_{\rm early}\sim\mathcal{O}(1)$.

\subsection{DE-era time mapping and the phantom sign}
From $H=\cs^3/(2GM)$ and $t_p=t_e(1-H_0/H)$:
\begin{equation}
  \frac{dt_p}{d\tau}
  =\frac{t_e H_0}{H^2}\dot{H}
  =-\frac{3}{2}t_e H_0(1+w_{\rm eff})
  =-\frac{3\cs^5}{4GP}(1+w_{\rm eff}),
\end{equation}
using $t_e H_0=\cs^5/(2GP)$.  Forward-running clocks require $w_{\rm eff}<-1$.

\subsection{Self-consistency check}
With $w_{\rm eff}=-1-\epsilon$ (constant $\epsilon>0$),
$\dot{H}=(3\epsilon/2)H^2$ gives
$H(\tau)=H_\ast/[1-(3\epsilon H_\ast/2)(\tau-\tau_\ast)]$ and
\begin{equation}
  \Delta t_p=\frac{3\cs^5\epsilon}{4GP}\cdot\frac{2}{3\epsilon H_\ast}
  =\frac{\cs^5}{2GPH_\ast}=\frac{M_\ast\cs^2}{P}=t_e(M_\ast).\;\checkmark
\end{equation}

\subsection{Planck-pivot regularisation: heat death}
Across the Planck pivot the evaporation power flips:
\begin{center}
\begin{tabular}{lcccc}
\toprule
\textbf{Regime} & $M$ range & $T$ & $A$ & $P$ \\
\midrule
Gravitational & $M>\Mp$ & $\propto 1/\sqrt{M}$ & $\propto M^2$ & const \\
Quantum & $M<\Mp$ & $\propto M^{3/2}$ & $\propto 1/M^2$ & $\propto M^4$ \\
\bottomrule
\end{tabular}
\end{center}

In the quantum regime $dM/dt_p=-\alpha M^4/\cs^2$, so the remaining lifetime
$\Delta t_p=(\cs^2/3\alpha)(1/M_f^3-1/\Mp^3)\to\infty$ as $M_f\to 0$: a Planck
remnant forms.

On the child side, UV exhaustion forces $\epsilon\to 0$ as $H\to H_P$.  With
$\epsilon(H)\approx \epsilon_0(1-H/H_P)$, linearising near $H_P$ yields an
exponential approach to $H_P$ in infinite child time.  The Big Rip is softened
into an asymptotic heat death.

\subsection{Local vs.\ global caveat and epistemic consequence (expanded)}
The Hubble radius $r_H(\tau)=\cs/H(\tau)$ is a local statement: in an
approximately FRW bulk it is the same for all comoving observers at a given
$\tau$, but the \emph{Hubble sphere} is always centred on the observer.  In a
finite cavity with a distinguished boundary (the shell), the phrase ``the Hubble
sphere fills the interior'' is position-dependent: an observer at location $x$
has filled the cavity only once
\begin{equation}
  r_H(\tau)\gtrsim d_{\rm wall}(x,\tau),
\end{equation}
where $d_{\rm wall}$ is that observer's proper distance to the wall.

\paragraph{Epistemic consequence.}
Because FRW symmetry makes $H(\tau)$ locally the same for all bulk observers,
there is no local measurement revealing $d_{\rm wall}$ unless the wall enters
the observer's past light cone (producing anisotropy/inhomogeneity).  In the
late-time phantom phase ($w<-1$), $r_H$ shrinks, making new causal contact with
the wall \emph{harder}, not easier: an observer who has not already seen
wall-influenced signals will generically never see them.

\subsection{Predicted value of $\epsilon=|1+w|$ (expanded)}
From the clock map one may write
\begin{equation}
  \frac{dt_p}{d\tau}=\frac{3\cs^5\epsilon}{4GP},
\end{equation}
where $P$ is the evaporation power (in the constant-power regime).  Writing
$P\sim (\cs^5/G)\,f$ with $f$ a dimensionless microphysical factor (greybody,
numerical constants), gives
\begin{equation}
  \frac{dt_p}{d\tau}\sim \frac{3\epsilon}{4f}.
\end{equation}
For strong time dilation ($dt_p/d\tau\ll 1$) we require $\epsilon\ll f$.  If
$f\sim\mathcal{O}(1)$, then $\epsilon\ll 1$, consistent with current
observational bounds on $w$.

\subsection{Summary table and open questions}
\begin{center}
\begin{tabular}{p{0.64\textwidth}p{0.28\textwidth}}
\toprule
\textbf{Result} & \textbf{Status} \\
\midrule
$r_H=R_S$ (Hubble radius equals Schwarzschild radius) & Derived \\
Junction consistency ($\sigma\approx 0$ at leading order) & Consistent \\
$w_{\rm eff}<-1$ required for forward-running clocks & Derived \\
Early universe decoupled from evaporation ($\mathcal{N}\gg 1$) & Derived \\
Big Rip at constant $\epsilon$, softened to heat death by Planck pivot & Derived \\
$|1+w|=\epsilon\ll 1$ (small phantom departure) & Predicted \\
\bottomrule
\end{tabular}
\end{center}

\paragraph{Open questions.}
\begin{itemize}
  \item A more detailed derivation of the $\mathcal{O}(1)$ early-era clock map
    (dynamical junction in the full index/noise description).
  \item First-principles calculation of $P$ (and thus $\epsilon$) from shell
    microphysics (greybody factors, species content, etc.).
  \item The mechanism that turns on window drift (DE) at the correct epoch when
    the Hubble sphere grows to encompass the cavity.
\end{itemize}

\section{Hierarchical condensation and interior large-scale structure}
\label{si:condensation}

This section records the more quantitative sketches behind the claim that
stochastic fluctuations in the blackbody cavity seed a two-phase hierarchy of
structures and, ultimately, a sponge-like cosmic web rather than a single
central core.

\subsection{Black-hole-branch seeds: radial drift and stall}
\label{si:seeds}

\paragraph{Accretion rate.}
A minimal estimate (geometric cross-section plus gravitational focusing) gives
\begin{equation}
  \dot{M}\;\simeq\;\pi R_{\rm mini}^{2}\,\rho_{\gamma}\,\cs\,
  \left(1+\frac{v_{\rm esc}^{2}}{\cs^{2}}\right),
  \qquad
  v_{\rm esc}^{2}=\frac{2GM}{R_{\rm mini}}.
\end{equation}

\paragraph{Jeans limit in a photon gas.}
For a photon gas of temperature $T$,
\begin{equation}
  M_{\rm J}(T)=\frac{\pi^{5/2}}{6}\,\frac{\cs^{3}}{\sqrt{G^{3}\rho_{\gamma}}}
  \;\propto\;T^{-3/2}.
\end{equation}
With $T\propto\rho_{\gamma}^{1/4}\propto M^{1/4}$ (local scaling), growth halts
after $\mathcal{O}(10)$ quanta: a seed cannot grow to a macroscopic fraction of
the parent shell mass.

\paragraph{Radial migration.}
The drift acceleration from anisotropic photon flux scales as
$a_{\rm rad}\propto A/M\propto N$, so a seed spirals toward the inner wall on a
timescale
\begin{equation}
  \tau_{\rm drift}\sim\frac{R_{\rm shell}}{a_{\rm rad}}\propto\frac{R_{\rm shell}}{N}.
\end{equation}
Once parked against the wall it merges with neighbours, thickening the parent
breathing layer into a sub-shell.

\subsection{Particle-branch droplets: Brownian network and filament formation}
\label{si:droplets}

For $N=1$ droplets the net drift acceleration is parametrically smaller:
\begin{equation}
  a_{\rm rad}\propto\frac{A}{M}\approx\frac{\Lp^{2}}{\mu},
\end{equation}
so motion is dominated by Brownian kicks with rms speed
$v_{\rm rms}\approx\sqrt{T/\mu}$.  The mean free path
\begin{equation}
  \lambda_{\rm B}\sim\frac{1}{n_{\rm seed}\pi R_{\rm mini}^{2}}
\end{equation}
can exceed seed--seed spacing, yielding a random walk through a time-dependent
seed network.

\paragraph{Topology classification ($l=0$ vs.\ $l\ge 1$).}
Droplets split into two populations:
\begin{enumerate}
  \item \textbf{Scalar solitons ($l=0$):} density fluctuations with no phase
    winding (collisionless; dark matter).
  \item \textbf{Topological defects ($l\ge 1$):} droplets trapping a phase
    winding (dissipative; baryonic matter).
\end{enumerate}
Combinatorial statistics favour the simpler $l=0$ population by a factor
$\sim 5$, naturally producing $\Omega_{\rm DM}/\Omega_b\approx 5$.

\paragraph{Morphological amplifier.}
Droplets clustered in a caustic source a higher local noise background $\tau$,
which triggers a super-linear downshifting response in neighbours
($\varepsilon\propto-\tau^\gamma$, $\gamma>1$), deepening the potential well and
capturing further matter.  A minimal modified diffusion--drift model is
\begin{equation}
  \partial_t n = D\nabla^{2}n - \nabla\!\cdot\!(n\mathbf v_{\rm drift})
  + \alpha_{\rm morph}\,n^2,
\end{equation}
where $D$ is set by Brownian motion and $\\mathbf v_{\rm drift}$ by the seed
potential.  Numerical integration (future work) yields filamentary,
sponge-like networks.

\subsection{Impact on early-epoch anisotropies}
\label{si:anisotropies}

Mini-shell seeds induce milli-Kelvin scale temperature dipoles (in external
units) in the photon bath on angular scales
$\theta\sim R_{\rm mini}/R_{\rm shell}$.  A toy spectrum is
\begin{equation}
  \frac{\Delta T}{T}(\ell)\approx
  \left(\frac{M_{\rm seed}}{M_{\rm shell}}\right)^{1/2}
  \ell_{\rm seed}^{2}\,e^{-\ell/\ell_{\rm seed}},
  \qquad
  \ell_{\rm seed}\sim\frac{R_{\rm shell}}{R_{\rm mini}}.
\end{equation}
Because $M_{\rm seed}\ll M_{\rm shell}$, such anisotropies remain below the
cavity's intrinsic shot noise and do not spoil breathing equilibrium.

\paragraph{Local vs.\ external units.}
Internal observers measure every energy against a shrunken local unit (their
$k_{\rm B}T$ rescales by $\chi^{-1}$), so the same dipole appears as
$(\Delta T/T)_{\rm loc}=\chi^{-1}(\Delta T/T)_{\rm ext}$.  Relative contrast
therefore stays $\mathcal{O}(10^{-3})$ across frames; only the baseline
temperature differs.

\subsection{Effective geometry of the interior}
\label{si:interior_geometry}

If radiation energy drains preferentially into seeds, a simple refractive index
profile $\chi(r)$ can decrease toward the centre.  Ray trajectories diverge,
giving an effective negative curvature.  A minimal parameterisation,
\begin{equation}
  \chi(r)=1+\epsilon\left(1-\frac{r}{R_{\rm shell}}\right),\qquad \epsilon\sim 10^{-2},
\end{equation}
gives a curvature scale $K\approx -2\epsilon/R_{\rm shell}^2$, consistent with a
weak but everywhere negative curvature and an ``optical horizon'' in which
timelike paths require arbitrarily long proper times to catch the wall.

\subsection{Open questions (kinetics of the stochastic cascade)}
\label{si:cascade_open_questions}
A full kinetic description of the stochastic cascade remains open, including
merger rates of sub-shells and the resulting statistical distribution of
particle-branch vs.\ shell-branch species in the interior bath.

% ===================================================================
\section{Discussion addendum: infinite zooms, finite energies}
\label{si:discussion}

This section collects clarifications that support the eternal-hierarchy picture
without requiring infinite energy, information, or computation in any single
window.

\subsection{Infinite zoom cascade with local time at every level}
The clique graph admits an unbounded hierarchy of nested bubbles.  Each zoom-in
by a fixed factor opens a fresh UV band, and the scale-window carried by an
observer shrinks in physical metres as it descends the hierarchy.  Thus the
newly opened UV band always carries finite energy/entropy in local units, and
an endless cascade of downshifts is possible without violating global bounds.

Every level has a nonzero scale-force slope $\sigma(t)$, so time exists
everywhere: ``timelessness'' from outside is merely the superposition of
incommensurate local clocks.

\subsection{Scale-window principle prevents divergences}
An observer carries a window $[\Lambda_{\rm IR},\Lambda_{\rm UV}]$ centred on
its effective Planck cell $\Lp(\Lambda_{\rm obs})$.  Energy, entropy, and
computational capacity inside the window are $\mathcal{O}(1)$ in local units,
independent of how many zooms are performed.  To access deeper UV information,
one must shrink; the energetic cost of shrinking equals the potential energy of
newly opened modes, preventing ``free lunches.''

\subsection{Infinite nested zooms inside a finite-lifetime shell}
External lifetime $\tau_{\rm ext}=M_{\rm init}\cs^2/P_0$ is finite, but
hyperbolic dilation near the wall makes internal proper time unbounded.  An
internal observer can execute countably many zoom steps before $t=\tau_{\rm
ext}$.  Each level has its own constant-power cascade, and the geometric series
of sub-energies sums to $M_{\rm init}\cs^2$.

\subsection{Impedance matching: why constant power is necessary for nesting}
Standard Hawking radiation ($P\propto M^{-2}$) forbids nesting: an inner shell
would radiate $P_{\rm inner}\gg P_{\rm outer}$, vaporising the parent from
inside.  Constant power ($P\propto M^0$) is the unique scaling that allows
impedance matching: $P_{\rm inner}=P_{\rm outer}=P_0$.

\subsection{Seed gradient, filament fate, and wall overtake}
\label{si:wall_overtake}
\begin{itemize}
  \item Seeds ($N>\Nstar$) form preferentially near the wall where the local
    Jeans mass is lowest.
  \item Radiation-pressure drift moves seeds inward; Brownian droplets
    ($N=1$) diffuse in a sponge-like filament network.
  \item As the shell contracts, rising noise variance and negative optical
    curvature compress and eventually absorb the filaments; external time for
    capture is finite while internal time can diverge logarithmically.
\end{itemize}

\subsection{Cosmic horizon as a Type-II gravitational droplet}
An apparent cosmic horizon in an FRW patch can be reinterpreted as the inner
surface of a larger-scale shell (Type-II, outward-oriented) when zooming out
one level.  In the super-observer's units, the region outside our horizon
contains deeper-UV modes (higher noise floor, shorter $\Lp$).  From our interior
vantage that region is stretched to infinite volume and redshifted to near-zero
temperature, giving the impression of an empty cold outside.

\subsection{Why the hierarchy alternates 2D shells and 3D bulks}
A finite UV window forces resolved subgraphs toward integer geometries: with
only short links the least-action arrangement is a regular lattice.  On a
closed interface the lattice is effectively 2D (a shell); in a filled region it
is 3D (a bulk).  Fractional-D tilings require holes (entropy loss) or weighted
long links (energy cost), so the free energy ``snaps'' to the nearest integer.

\paragraph{The $e$ conjecture (heuristic).}
If we weight bulk and shell with probabilities $p$ and $1-p$, the average
dimension is $D_{\rm eff}=3p+2(1-p)=2+p$.  Setting $D_{\rm eff}=e$ gives
$p=e-2\simeq 0.718$, corresponding to an approximate 3:1 split in log-depth
(three bulk units per shell unit).  This remains a motivated conjecture rather
than a derived necessity.

\subsection{Shell-thickness scaling and observational outlook}
\label{si:shell_thickness}
Rebalancing Stefan--Boltzmann flux (out) with Unruh absorption (in) gives
\begin{equation}
  T(R)=\left(\frac{P_0}{4\pi\sigma_{\rm SB}R^2}\right)^{1/4}\propto R^{-1/2}.
\end{equation}
Requiring optical depth corresponding to a mean-free-path count
$n_\ast\sim 10$--$30$ yields
\begin{equation}
  \boxed{\;\frac{\delta R}{R}=K\left(\frac{M}{10\,M_\odot}\right)^{-1/2}},\qquad
  K\simeq 0.02.
\end{equation}

\paragraph{Log-step reinterpretation.}
Let $\Delta\sigma_{\rm shell}$ be the drop in log-scale coordinate across one
shell.  For $\Delta\sigma_{\rm shell}\ll 1$,
$\delta R/R\simeq \Delta\sigma_{\rm shell}$ (linearising $R\to Re^{-\Delta\sigma}$).
Thus
\begin{equation}
  \Delta\sigma_{\rm shell}(M)=K\left(\frac{M}{10\,M_\odot}\right)^{-1/2}.
\end{equation}
If the bulk:shell bookkeeping remains 3:1, then
$\Delta\sigma_{\rm bulk}=3\Delta\sigma_{\rm shell}$ and
$\Delta\sigma_{\rm cycle}=4\Delta\sigma_{\rm shell}$.

\paragraph{Numerical check.}
\begin{center}
\begin{tabular}{rccc}
\toprule
$M/M_\odot$ & $\delta R/R$ & $\Delta\sigma_{\rm shell}$ & $q\equiv R/\delta R$ \\
\midrule
10 & 0.020 & 0.020 & 50 \\
30 & 0.012 & 0.012 & 83 \\
$4\times 10^6$ & $3\times 10^{-4}$ & $3\times 10^{-4}$ & $3\times 10^3$ \\
\bottomrule
\end{tabular}
\end{center}

\paragraph{Finite-shell corrections in binaries.}
For a companion approaching within $\lesssim 3\,\delta R$,
tidal quadrupoles and overlapping photon baths can add corrections, but at
LIGO-band separations these effects are far below current sensitivity.  They
become plausible targets for extreme mass-ratio inspirals grazing supermassive
shells or for horizon-scale VLBI probing $\delta R$ directly.

\section{Key formulae (quick reference)}
\label{si:key_formulae}
\begin{itemize}
  \item \textbf{Droplet free energy:}
    $F(R;N)=\frac{\alpha N\hbar c}{R}+\frac{\eta_0}{2}\frac{N^2}{R^3}-\frac{G(N\mu)^2}{R}$.
  \item \textbf{Critical occupancy:}
    $\Nstar=\alpha\hbar c/(G\mu^2)\sim \mathcal{O}(1)$ in window units.
  \item \textbf{Dual temperature laws:}
    $T_{\rm sol}\propto m^{3/2}$ (particle side) vs.\ $T_{\rm BH}\propto 1/\sqrt{M}$
    (shell side).
  \item \textbf{Evaporation power:} $P_{\rm rad}\propto \text{const}$ (gravitational regime),
    crossing to $P\propto M^4$ below the Planck pivot.
  \item \textbf{Duality lock (late era):} $H=\cs^3/(2GM)$ and $r_H=\cs/H=R_S$.
  \item \textbf{Clock map (DE era):}
    $\frac{dt_p}{d\tau}=-(3\cs^5/(4GP))(1+w_{\rm eff})$ requiring $w_{\rm eff}<-1$.
\end{itemize}

\section{Related approaches (orientation pointers)}
\label{si:related_work}
\begin{itemize}
  \item Gravastars / dark-energy stars (Mazur \& Mottola): de Sitter core matched to
    Schwarzschild exterior via a thin relativistic shell.
  \item String fuzzballs (Mathur): horizon replaced by microstate ``surface''.
  \item Planck stars (Rovelli \& Vidotto): loop-quantum-gravity bounce core.
  \item Quantum N-portrait / graviton BEC (Dvali \& Gomez): black hole as critical condensate.
  \item Regular black holes / modified gravity (Bardeen; Hayward; asymptotic safety; nonlocal models):
    de Sitter-like cores with transition layers.
  \item Membrane paradigm / firewalls (Thorne--Price--Macdonald; Almheiri et al.):
    effective dissipative or energetic surface near the horizon.
\end{itemize}

\subsection{Piece-wise arrows of time in an eternal hierarchy}
Each cavity has a locally defined arrow of time: a monotone scale direction
($\dot{\sigma}<0$) plus dissipation (constant-power leakage) makes the dynamics
irreversible.  Each level has finite accessible information and energy in its
window, but the hierarchy is past-and-future unbounded in scale.

% ===================================================================
\section{Spectral-dimension selection: why $e$ (and hence 3)}
\label{si:spectral_dimension}

This appendix records a heuristic argument that a neutral information-per-cost
criterion selects an effective coordination $k\approx e$, with integer
feasibility biasing to $k=3$ and thus spectral dimension $D_s\approx 3$.

\subsection{Setup and assumptions}
\begin{itemize}
  \item \textbf{Linear per-node budget (volume-normalised links):} outbound
    maintenance/throughput is linear in link weights, with a minimum useful link
    weight (noise/quantisation floor) making each additional active neighbour
    consume approximately constant budget.
  \item \textbf{Local choice and path entropy:} with $k$ active neighbours used
    roughly uniformly, the local branching entropy rate is $\hat H=\ln k$ (nats
    per step) while cost scales $\sim k$.
  \item \textbf{UV window and adiabaticity:} a minimum cell bounds admissible
    protocol speed via $\tau \gtrsim L/\omega_{1,P}$ for Fisher/thermodynamic
    distance $L$.
  \item \textbf{Window invariance (unit-fixing):} keep local units stable during
    protocols, implying a roughly constant excess-power ceiling and thus
    constant metric speed along optimal schedules.
\end{itemize}

\subsection{Local efficiency objective}
Define a unit-free efficiency (benefit per cost)
\begin{equation}
  \eta(k):=\frac{\ln k}{k}.
\end{equation}
Continuous optimisation gives
\begin{equation}
  \eta'(k)=\frac{1-\ln k}{k^2}=0\quad\Longrightarrow\quad \ln k=1
  \quad\Longrightarrow\quad k=e.
\end{equation}
With integer feasibility, the nearest stable coordination is $k=3$.

An equivalent MDL-style derivation: representing $M$ alternatives with radix
$k$ costs $\propto k(\ln M/\ln k)=(\ln M)\,[k/\ln k]$, minimised at $\ln k=1$.

\subsection{Time--dissipation gauge}
Finite-time thermodynamics gives a bound
\begin{equation}
  \Sigma \gtrsim \frac{L^2}{\tau},
\end{equation}
where $\Sigma$ is total dissipation.  UV and window invariance fix a constant
metric-speed gauge that saturates this bound on geodesic schedules.  These
constraints set the feasible time/dissipation scale; the local allocation
$k\approx e$ follows from the neutral information-per-cost objective above.

\subsection{From branching to spectral dimension}
On near-regular, locally homogeneous graphs, effective coordination controls
diffusivity/expansion and thus the spectral dimension via heat-kernel scaling.
With $k\approx e$ and integer feasibility, $k=3$ is preferred, yielding
$D_s\approx 3$ in the vacuum fixed point.

\subsection{Caveats and empirical checks}
If per-link costs are nonlinear or minimum useful weights vary strongly, the
optimum shifts.  Suggested checks:
\begin{itemize}
  \item optimise $(\ln k)/k$ under linear row-sum budgets and noise floors in
    vacuum-relaxation simulations and measure the resulting mean coordination,
  \item verify constant-metric-speed schedules numerically saturate
    $\Sigma=L^2/\tau$ between nearby equilibria,
  \item measure heat-kernel slopes to confirm $D_s\approx 3$ robustness under
    heterogeneity.
\end{itemize}

\end{document}

