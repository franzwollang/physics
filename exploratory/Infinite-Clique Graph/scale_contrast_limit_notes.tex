\documentclass[11pt]{article}
\usepackage[a4paper,margin=1in]{geometry}
\usepackage{amsmath,amssymb,amsfonts}
\usepackage{bm}
\usepackage{mathtools}
\usepackage{microtype}
\usepackage{hyperref}

\hypersetup{
  colorlinks=true,
  linkcolor=blue,
  citecolor=blue,
  urlcolor=blue
}

\title{Notes on Vacuum Relaxation and Maximum Supported Scale Contrast}
\author{Franz Wollang (working notes)}
\date{\small \today}

\begin{document}
\maketitle

\section{Aim and Setup}

These notes sketch how the \emph{environmental} vacuum--relaxation drive in the Infinite--Clique Graph framework constrains the \emph{maximum scale contrast} that a near--regular, three--dimensional subgraph can support. The central idea is that a soliton's ability to shrink is not determined in isolation, but by how much its \emph{local environment}---the surrounding vacuum graph and noise field---can reorganise without leaving the 3D fixed point.

The qualitative claim is:
\begin{quote}
  On a near--regular vacuum patch with spectral dimension $d^\star\simeq 3$, there is a finite range of admissible local scale ratios (e.g.\ soliton size vs.\ ambient coarse--grain size) that can be maintained while keeping the \emph{environment} close to its dimensional fixed point. Beyond a certain scale contrast, the cost in the vacuum functional from distorting the local Laplacian spectrum and its dimension exceeds the scale--space free--energy gain from further downscaling, so further shrinking \emph{raises} total free energy.
\end{quote}
We make this precise at the level of an effective theory that couples:
\begin{itemize}
  \item the vacuum relaxation functional $F_{\rm vac}[J]$ for link weights $J_{ij}$,
  \item the heat--kernel definition of the local intrinsic dimension $d_{\text{int}}(x)$,
  \item and a coarse scale field $\lambda(x)$ that encodes local resolution / grain size of the \emph{vacuum environment}.

\end{itemize}

\section{Vacuum Relaxation and Dimensional Bias}

From the ICG Supplementary Information, vacuum marginalisation over the fast phase sector yields an effective functional for the slow link weights,
\begin{equation}
  F_{\rm vac}[J] \;=\; \frac{T_{\rm eff}}{2}\,\ln\det{}' L(J) \;-\; T\,S[J] \;+\; \beta_{\mathrm{dim}}\big(\langle d_{\text{int}}\rangle[J] - d^\star\big)^2,
  \label{eq:Fvac-J}
\end{equation}
subject to per--node budget constraints $\sum_j J_{ij}=Z_i$ and nonnegativity $J_{ij}\ge 0$. Here:
\begin{itemize}
  \item $L(J)$ is the weighted graph Laplacian induced by $J$.
  \item $S[J]=\sum_i\sum_j (J_{ij}/Z_i)\,\ln(J_{ij}/Z_i)$ is the entropy of normalised out--weights.
  \item $\langle d_{\text{int}}\rangle[J]$ is the intrinsic (spectral) dimension extracted from the heat kernel of $L(J)$ in a chosen diffusive window via
    \begin{equation}
      K(t) = \Tr\big(e^{-tL(J)}\big),\qquad
      \langle d_{\text{int}}\rangle(t) := -2\,\frac{d\ln K(t)}{d\ln t}.
    \end{equation}
\end{itemize}
The dimensional bias term with coefficient $\beta_{\mathrm{dim}}>0$ penalises link patterns that raise or lower $\langle d_{\text{int}}\rangle$ away from a preferred value $d^\star$, which for the vacuum is near $3$ (Lambert--$W$ fixed point in the Dark Sector and ICG SI).

On large, homogeneous patches one can replace $\langle d_{\text{int}}\rangle[J]$ by an effective scalar field $d(x)$ that varies slowly in space; the coarse--grained form of the dimensional term is then, to leading order,
\begin{equation}
  F_{\mathrm{dim}}[d] \;\approx\; \int \Big[ A\,(d(x)-d^\star)^2 \;+\; B\,|\nabla d(x)|^2 \Big]\,d^3x,
  \label{eq:Fdim-eff}
\end{equation}
with $A\propto\beta_{\mathrm{dim}}$ and $B>0$ an effective stiffness generated by the $\ln\det{}'L$ and entropy terms when expanded around a near--regular vacuum.

\section{Scale Field and Its Coupling to Dimension}

The key observation is that a soliton is defined only relative to its \emph{environment}: the surrounding vacuum graph, with its own coarse--grain scale and Laplacian spectrum. To describe this environmental scale we introduce a coarse \emph{scale field} $\lambda(x)$ that captures local resolution or grain size of the vacuum:
%
\medskip

Operationally, $x$ here labels points \emph{within a single emergent 3D component}: a near--regular, lattice--like subgraph on which the effective Laplacian converges to a Laplace--Beltrami operator. In such a component, $x$ can be given a metric interpretation and distances $d_{\rm 3D}(x_1,x_2)$ are finite. For points $x_1$ and $x_2$ that lie in \emph{different} 3D components (separated by a scale--domain boundary where the 3D EFT fails), there is no single smooth $x$ chart spanning both; in that effective 3D geometry one may regard $d_{\rm 3D}(x_1,x_2)=\infty$. The \emph{environmental} scale field $\lambda$ can still be defined abstractly (via local coarse--graining) on each component, and its values can be compared across components, but there is no finite 3D distance between those points.

We introduce a coarse \emph{scale field} $\lambda(x)$ that captures local resolution or grain size:
\begin{equation}
  \lambda(x) := \ln\frac{\ell_{\mathrm{env}}}{\ell(x)} \;=\; -\ln\frac{\ell(x)}{\ell_{\mathrm{env}}}\,,
\end{equation}
where $\ell(x)$ is the local coarse--grain length and $\ell_{\mathrm{env}}$ a reference background value. Positive $\lambda$ corresponds to finer local scale (smaller $\ell$).

The microscopic link statistics tie local coordination and neighbourhood radius to both:
\begin{itemize}
  \item the effective coarse--grain scale $\ell(x)$ (or $\lambda(x)$),
  \item and the local spectral dimension $d(x)\approx \langle d_{\text{int}}\rangle(t;x)$ extracted from the heat kernel in a window where diffusion probes scales $\sim \ell(x)$.
\end{itemize}
To leading order, changing the local scale while keeping the vacuum near--regular and sparse induces a monotone relation
\begin{equation}
  d(x) - d^\star \;\simeq\; \chi\,\lambda(x),
  \label{eq:d-lambda-linear}
\end{equation}
with $\chi$ an $\mathcal O(1)$ susceptibility: making the graph locally finer (larger $\lambda$) without drastically altering its connectivity pattern tends to increase the local spectral dimension, whereas coarsening (negative $\lambda$) tends to reduce it.

This linear relation is not exact but suffices for an effective theory in which both $d$ and $\lambda$ vary slowly. More generally one may write $d-d^\star = f(\lambda)$ with $f'(0)=\chi$, and expand $f$ to first order for modest $\lambda$.

Substituting Eq.~\eqref{eq:d-lambda-linear} into Eq.~\eqref{eq:Fdim-eff} yields an effective \emph{environmental} cost for $\lambda$,
\begin{equation}
  F_{\mathrm{dim}}[\lambda] \;\approx\; \int \Big[ A'\,\lambda(x)^2 \;+\; B'\,|\nabla \lambda(x)|^2 \Big]\,d^3x,
  \label{eq:Flambda-eff}
\end{equation}
with $A'=A\chi^2>0$ and $B'=B\chi^2>0$. Thus, the same vacuum relaxation that stabilises the spectral dimension also generates:
\begin{itemize}
  \item a \emph{local penalty} for large departures of the scale from its vacuum value ($\lambda^2$ term),
  \item a \emph{gradient stiffness} that penalises large spatial variations of scale between neighbouring regions ($|\nabla\lambda|^2$ term).
\end{itemize}

\section{Soliton Downscaling and Local Scale Anomalies}

Consider now a localised soliton core of characteristic size $\sigma$ embedded in a vacuum with ambient coarse scale $\ell_{\mathrm{env}}$. The soliton is a \emph{perturbation of this environment}: it modifies the link weights in its interior, creating a localized scale profile $\lambda(x)$ relative to the ambient:
\begin{equation}
  \lambda(x) \;\approx\;
  \begin{cases}
    \lambda_{\mathrm{core}} & |x-x_0|\lesssim \sigma,\\[0.3em]
    0 & |x-x_0|\gg \sigma,
  \end{cases}
  \qquad \lambda_{\mathrm{core}}:=\ln\frac{\ell_{\mathrm{env}}}{\ell_{\mathrm{core}}}\,,
\end{equation}
with $\ell_{\mathrm{core}}$ the effective internal scale in the core region. This profile is smoothed in practice, but the step--like picture is sufficient to estimate costs.

The soliton's own scale--space free energy (Relativity paper, scale--space appendix) drives $\sigma$ and $\ell_{\mathrm{core}}$ (equivalently $\lambda_{\mathrm{core}}$) toward smaller values as the background noise proxy $\tau(r)$ increases:
\begin{equation}
  E_{\mathrm{coh}}(\sigma,r) = A_\gamma\,\sigma^{-2} - A_\eta\,\tau(r)\,\sigma^{-1},\qquad
  \sigma^*(r)=\frac{2A_\gamma}{A_\eta\,\tau(r)}.
\end{equation}
Downscaling the core reduces $E_{\mathrm{coh}}$, but it also increases $|\lambda_{\mathrm{core}}|$ and steepens the transition in $\lambda(x)$ between core and environment, thereby increasing $F_{\mathrm{dim}}[\lambda]$. In other words, there is a trade--off: the soliton can lower its \emph{internal} energy by shrinking, but only by forcing the \emph{environment} to deviate ever more strongly from its preferred scale and spectral dimension.

\subsection{Estimate of the dimensional relaxation cost}

For a spherical core of radius $\sigma$, approximate $\lambda(x)$ by a smooth interpolation from $\lambda_{\mathrm{core}}$ inside to $0$ outside, over a boundary layer of thickness $\sim\sigma$. Then the dominant contributions to Eq.~\eqref{eq:Flambda-eff} are of order
\begin{align}
  F_{\mathrm{dim}}[\lambda] &\sim
  \underbrace{A'\,\lambda_{\mathrm{core}}^2\,\sigma^3}_{\text{bulk term}} \;+\;
  \underbrace{B'\,\lambda_{\mathrm{core}}^2\,\sigma}_{\text{gradient term}}\;,
  \label{eq:Flambda-scaling}
\end{align}
up to shape factors of $\mathcal O(1)$. The bulk term comes from the volume where $|\lambda|\approx|\lambda_{\mathrm{core}}|$; the gradient term comes from the shell where $|\nabla\lambda|\sim|\lambda_{\mathrm{core}}|/\sigma$ over area $\sim\sigma^2$.

As the soliton shrinks and the internal scale becomes finer, the scale contrast $\lambda_{\mathrm{core}}$ grows:
\begin{equation}
  \lambda_{\mathrm{core}} = \ln\frac{\ell_{\mathrm{env}}}{\ell_{\mathrm{core}}} \;=\; \ln\frac{\sigma}{\ell_{\mathrm{core}}}\;+\;\text{const}.
\end{equation}
In the simple scale--space model, the soliton maintains a fixed shape in scaled coordinates, so $\ell_{\mathrm{core}}$ tracks $\sigma$; the physically relevant contrast is then between the soliton scale and the environment,
\begin{equation}
  \lambda_{\mathrm{core}} \sim \ln\frac{\ell_{\mathrm{env}}}{\sigma}.
\end{equation}
Thus, as $\sigma$ decreases at fixed $\ell_{\mathrm{env}}$, both $|\lambda_{\mathrm{core}}|$ and the gradient across the core boundary grow.

\section{Balance Between Scale--Space Gain and Dimensional Cost}

The total coarse--grained free energy associated with the soliton's scale is the sum of its scale--space contribution and the induced dimensional--relaxation cost,
\begin{equation}
  F_{\mathrm{tot}}(\sigma,r) \;\approx\;
  E_{\mathrm{coh}}(\sigma,r) \;+\;
  F_{\mathrm{dim}}[\lambda(\sigma)].
  \label{eq:total-F-sigma}
\end{equation}
Using the scalings above,
\begin{align}
  E_{\mathrm{coh}}(\sigma,r) &\sim A_\gamma\,\sigma^{-2} - A_\eta\,\tau(r)\,\sigma^{-1},\\
  F_{\mathrm{dim}}[\lambda(\sigma)] &\sim A'\,\lambda_{\mathrm{core}}(\sigma)^2\,\sigma^3 + B'\,\lambda_{\mathrm{core}}(\sigma)^2\,\sigma.
\end{align}
For moderate scale contrasts $\lambda_{\mathrm{core}}\sim\mathcal O(1)$, the dimensional term is small and the scale--space energy dominates, yielding the familiar $\sigma^*(r)=2A_\gamma/(A_\eta\tau)$ behaviour.

However, as $\sigma$ is driven to be much smaller than $\ell_{\mathrm{env}}$ at fixed $r$ and $\tau(r)$, the contrast $\lambda_{\mathrm{core}}$ grows logarithmically,
\begin{equation}
  \lambda_{\mathrm{core}}^2(\sigma) \sim \big(\ln(\ell_{\mathrm{env}}/\sigma)\big)^2,
\end{equation}
and the gradient term in $F_{\mathrm{dim}}$ scales roughly as
\begin{equation}
  F_{\mathrm{dim,grad}} \sim B'\,\big(\ln(\ell_{\mathrm{env}}/\sigma)\big)^2\,\sigma.
\end{equation}
The net derivative of $F_{\mathrm{tot}}$ with respect to $\sigma$ then behaves, for small $\sigma$, like
\begin{equation}
  \frac{\partial F_{\mathrm{tot}}}{\partial\sigma}
  \;\sim\; -2A_\gamma\,\sigma^{-3} + A_\eta\,\tau(r)\,\sigma^{-2} \;+\;
  B'\,\frac{\partial}{\partial\sigma}\Big[\big(\ln(\ell_{\mathrm{env}}/\sigma)\big)^2\,\sigma\Big],
\end{equation}
where the last term grows in magnitude as $\sigma\to 0$ because both the log and its derivative diverge. There will therefore exist a finite $\sigma_{\min}(r)$ at which
\begin{equation}
  \frac{\partial F_{\mathrm{tot}}}{\partial\sigma}\Big|_{\sigma=\sigma_{\min}(r)} = 0,
\end{equation}
and for $\sigma<\sigma_{\min}(r)$ the dimensional--relaxation cost dominates, making further shrinking unfavourable:
\begin{equation}
  \frac{\partial F_{\mathrm{tot}}}{\partial\sigma} > 0\quad\Rightarrow\quad F_{\mathrm{tot}}(\sigma)\ \text{increases as }\sigma\downarrow.
\end{equation}

In other words, the same vacuum relaxation that enforces a near--3D spectral dimension provides a dynamical \emph{lower bound} on the soliton scale for given environment, set by the balance of:
\begin{itemize}
  \item scale--space gain from shrinking the core (favouring smaller $\sigma$),
  \item and dimensional--relaxation cost from the resulting scale anomaly and its gradient (favouring larger $\sigma$).
\end{itemize}

\section{Maximum Supported Scale Contrast}

We can express this as a constraint on the maximum scale contrast that a near--regular 3D subgraph can support. Define the contrast
\begin{equation}
  \mathcal C(r) := \frac{\ell_{\mathrm{env}}}{\sigma_{\min}(r)}.
\end{equation}

By definition of $\sigma_{\min}(r)$, pushing the core scale any smaller than $\ell_{\mathrm{env}}/\mathcal C$ at radius $r$ would require an increase of $F_{\mathrm{tot}}$ and is therefore disfavoured by the microdynamics. Thus:
\begin{quote}
  On a near--regular 3D vacuum patch, the largest supported local scale contrast between a soliton core and its environment at radius $r$ is of order $\mathcal C(r)$, set by the balance between scale--space and dimensional--relaxation terms.
\end{quote}

In particular, in strong--field regimes (e.g.\ near black holes), the environment scale $\ell_{\mathrm{env}}$ is itself reduced by downshifting and the local noise $\tau(r)$ grows. The allowed $\mathcal C(r)$ then reflects both the strength of the scale--space drive (via $\tau(r)$) and the stiffness of the dimensional--relaxation sector (via $A',B'$). This provides a pathway to relate:
\begin{itemize}
  \item the observed ``cosmic window'' between IR (horizon--scale) and UV (Planck--like) effective scales,
  \item and the micro--parameters $(\beta_{\mathrm{dim}}, T_{\mathrm{eff}}, A_\gamma,A_\eta)$ governing vacuum relaxation and soliton scale--space dynamics.
\end{itemize}

Quantitatively, one would:
\begin{enumerate}
  \item Calibrate $A',B'$ and the $d$--$\lambda$ susceptibility $\chi$ from graph simulations that measure how local spectral dimension responds to imposed scale variations.
  \item Solve $\partial F_{\mathrm{tot}}/\partial\sigma=0$ for $\sigma_{\min}(r)$ in representative environments (galaxy halos, near--horizon regions) to obtain $\mathcal C(r)$.
  \item Compare the resulting maximum scale contrasts with the phenomenological window between cosmological and microscopic scales inferred from Dark Sector and Gravity papers.
\end{enumerate}

\section{Domain Structure in Scale and Position}

So far we have treated the scale contrast around a \emph{single} soliton as a local problem. In practice, the scale field $\lambda(x)$ is defined over extended regions and its dynamics are coupled in space. The effective cost functional Eq.~\eqref{eq:Flambda-eff} generalises to
\begin{equation}
  F_{\mathrm{eff}}[\lambda] \;\sim\; \int d^3x\,\Big[
  \tfrac12 K\,|\nabla\lambda(x)|^2 \;+\; V\big(\lambda(x)\big)
  \Big],
  \label{eq:Feff-domain}
\end{equation}
where
\begin{itemize}
  \item $K$ is the coarse--grained stiffness generated by the vacuum relaxation functional,
  \item $V(\lambda)$ is the local part of the cost, which contains at least the quadratic piece $A'\lambda^2$, and may have additional structure.
\end{itemize}
In the simplest case, $V(\lambda)$ has a single minimum at $\lambda=0$, encoding the preferred vacuum scale and $d(x)\simeq d^\star$. If the microscopic graph and bath are approximately self--similar across scales, it is natural for $V(\lambda)$ to have \emph{multiple} minima at different $\lambda_i$, each corresponding to a quasi--self--similar, near--3D fixed point at a different scale band. In that case the scale field realises a multi--well ``landscape'' in log--scale.

It is important to stress that the spatial coordinate $x$ appearing here is an \emph{operational} parameter: it labels points on the emergent 3D lattice--like subgraph on which the effective Laplacian and metric are defined. This $x$ is meaningful only within a single, near--regular 3D domain where the kernel converges to a Laplace--Beltrami operator; it is \emph{not} a fundamental coordinate on the underlying clique.

The gradient term $K|\nabla\lambda|^2$ penalises large scale differences between neighbouring points. The minimisers of Eq.~\eqref{eq:Feff-domain} are therefore:
\begin{itemize}
  \item extended \emph{domains} in $x$ where $\lambda(x)$ sits near a single minimum $\lambda_i$ (one effective 3D window),
  \item separated by \emph{domain walls} where $\lambda(x)$ interpolates between different minima.
\end{itemize}
Within each domain, the environment supports a coherent 3D Laplacian and solitons with a characteristic size set by $\lambda_i$, and $x$ can be promoted to a smooth metric coordinate. Crossing a domain wall corresponds to moving into a region where either the preferred scale band or even the effective spectral dimension changes; there, the 3D EFT breaks down and the notion of a single smooth $x$ chart connecting both sides is lost. After domains decouple in this sense, one has distinct pieces of 3D subgraph, each with its own local $x$--coordinates, but no globally smooth 3D coordinate system that covers all domains at once, even though the underlying clique remains connected.

On the underlying infinite clique there is no primitive notion of spatial location; all nodes are, in principle, mutually linked. ``Distance'' and ``separation'' are always defined relative to a chosen effective description. In the present context, two regions are ``far apart'' in the 3D sense if their \emph{effective connectivity through the 3D band} is weak: propagation of phase or diffusion between them via the relevant eigenmodes of $L(J)$ is strongly suppressed, or equivalently, the appropriate Green's function / heat kernel entries are tiny in the window that defines the 3D EFT. In a single connected 3D domain this weak connectivity manifests as a large emergent separation in $x$; once domains have decoupled, there is no single $x$ that spans them at all, only separate coordinate systems on each domain tied to their own effective Laplacians.

Fluctuations of $\lambda(x)$ around a given minimum correspond to ``scale oscillations'' of the environment and are gapped by the curvature of $V(\lambda)$ at that point. Tunnelling (or nucleation) between minima corresponds to a patch of vacuum reorganising its scale, with an amplitude suppressed by the barrier height in $V(\lambda)$ and the stiffness $K$ (large patches are extremely unlikely to jump between minima).

\section{Horizon and Shell as Scale Domain Walls}

In the Dark Sector paper, the cosmic horizon appears as the \emph{infrared} edge of the window $[\kmin,\kmax]$ over which the phase kernel and noise spectrum are actively sampled. In log--scale, this is the statement that $\lambda(x)$ sits in one well only over a finite band of scales; beyond a certain large length the relevant modes are effectively frozen into the background. In the black hole paper, the asymptotic shell appears as a \emph{ultraviolet} saturation of the same scale--space mechanism: solitons are driven down in scale until they hit the environment--dependent $\sigma_{\min}$, and the interior experiences almost no further scale--space drive.

Both can be reinterpreted in the present language as manifestations of domain structure in $(x,\lambda)$:
\begin{itemize}
  \item \textbf{Cosmic horizon as an IR domain wall.} On very large scales, the sliding window and vacuum relaxation select a particular band in log--$k$ where $S(k)\propto 1/k$ and $d(x)\simeq 3$. The IR edge at $k\sim \kmin\sim H_{\mathrm{eff}}/c_s$ is the point beyond which one cannot maintain a \emph{single} 3D domain linking our patch to regions so far away: the cost in $F_{\mathrm{dim}}$ and $F_{\mathrm{eff}}$ of enforcing a coherent $\lambda(x)$ across those scales becomes prohibitive. In the long run, regions beyond the horizon relax into their own 3D domains (possibly in different wells), effectively decoupled from ours at the level of smooth 3D geometry, even though the underlying graph remains connected.
  \item \textbf{Black hole shell as a UV domain wall.} Near a black hole, the local noise $\tau(r)$ and the scale--space drive attempt to push soliton cores to ever smaller $\sigma$. The balance analysed above shows that at some radius $r_{\mathrm{shell}}$ the environment--dependent minimum $\sigma_{\min}(r)$ is reached. At that radius the cost of further shrinking---encoded in the dimensional deviation and scale--gradient terms of $F_{\mathrm{dim}}$---matches the scale--space gain; the system forms a high--contrast layer of maximally downscaled solitons: a shell. Interior to this shell, $\nabla\tau\approx 0$ and the 3D soliton EFT effectively shuts off. In $(x,\lambda)$ this is a UV domain wall between the ``our--scale'' 3D domain outside and a different regime (high--noise interior) inside.
\end{itemize}
In both cases, the same vacuum relaxation and scale--field stiffness that stabilise a low--integer spectral dimension also enforce \emph{scale separation}: smooth 3D soliton dynamics are supported only within certain bands of $\lambda$; attempting to extend them continuously beyond those bands runs into a domain wall where either the preferred scale or the effective dimension changes.

This provides a unified, ``domain wall'' view of:
\begin{itemize}
  \item the \emph{cosmic horizon}, which bounds how far out a single 3D window can extend coherently,
  \item and \emph{black hole shells}, which bound how far down in scale a given environment can drive solitons.
\end{itemize}
Both are scale--domain boundaries arising from the same underlying free--energy structure.

\section{Summary}

The vacuum relaxation drive, through its dimensional--bias term and the induced gradient stiffness of the spectral dimension, provides a natural mechanism that limits how sharply scale can vary in space. When coupled to the soliton scale--space functional, this yields a finite, environment--dependent lower bound on the soliton scale: shrinking the core beyond this bound would create too large an anomaly in the local spectral dimension and its gradient, raising the total free energy.

Consequently, there is a \emph{maximum supported scale contrast} between soliton cores and the surrounding vacuum in any near--regular 3D patch. This constraint plays a central role in strong--field regimes (e.g.\ black hole interiors) and offers a route to connect the observed ``cosmic window'' of scales to the micro--parameters of the graph substrate and its vacuum relaxation dynamics.

\end{document}


