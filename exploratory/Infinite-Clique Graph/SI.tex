% !TeX program = pdflatex
\documentclass[11pt]{article}
\usepackage[a4paper,margin=1in]{geometry}
\usepackage{amsmath,amssymb,amsfonts}
\usepackage{graphicx}
\usepackage{physics}
\usepackage{hyperref}
\usepackage{xr-hyper}
\externaldocument{paper}
\usepackage{bm}
\usepackage{mathtools}
\usepackage{microtype}
\usepackage{enumitem}
\usepackage{authblk}
\usepackage{fancyhdr}

\hypersetup{
  colorlinks=true,
  linkcolor=blue,
  citecolor=blue,
  urlcolor=blue
}

\title{Supplementary Information\\Spacetime from First Principles: Free\mbox{\,-}Energy Foundations on the Infinite\mbox{\,-}Clique Graph}
\author[ ]{Franz Wollang}
\affil[ ]{\small Independent Researcher}
\date{\small Dated: 2025-08-29}

% Number sections as S1, S2, ...
\renewcommand{\thesection}{S\arabic{section}}
\renewcommand{\thesubsection}{S\arabic{section}.\arabic{subsection}}

\begin{document}
\maketitle

% Footer marking the draft status
\pagestyle{fancy}
\fancyhf{}
\fancyfoot[C]{\small Supplementary Information — Draft}
\renewcommand{\headrulewidth}{0pt}
\renewcommand{\footrulewidth}{0pt}

% Prominent draft disclaimer box
\begin{center}
\setlength{\fboxsep}{8pt}%
\fbox{\parbox{0.92\textwidth}{\centering\bfseries DRAFT — NOT FOR CITATION\\[4pt]
This Supplementary Information accompanies the manuscript ``Spacetime from First Principles: Free\,Energy Foundations on the Infinite\,Clique Graph'' and collects technical derivations and interpretive material referenced in the main text.}}
\end{center}
\vspace{1em}

\tableofcontents
\vspace{1em}
\vspace{1em}


\section{Discrete\,\textrightarrow\,Continuum Convergence of the Laplacian}\label{si:disc-cont-conv}
We justify the continuum limit $L \Rightarrow -\nabla\!\cdot(c_s^2\nabla)$ in two complementary ways: (i) a local Taylor expansion under isotropy, and (ii) quadratic\,form (Dirichlet energy) convergence.

\paragraph{Assumptions (compact).}
\begin{itemize}[leftmargin=*]
  \item Near\,regular vacuum: degree and weights vary slowly; local neighborhood approximately isotropic.
  \item Smooth embedding: nodes lie near a smooth manifold patch with coordinates $x_i$; fields $f_i\equiv f(x_i)$ are restrictions of a $C^2$ function.
  \item Volume\,normalised couplings $w_{ij}=J'_0/V_K$; bounded, finite neighborhood radius relative to curvature scales.
\end{itemize}

\subsection*{(i) Taylor expansion under local isotropy}
Consider the graph Laplacian action $(L f)_i = \sum_j w_{ij}(f_i-f_j)$. Expand $f(x_j)$ about $x_i$:
\begin{equation}
  f(x_j) = f(x_i) + (x_j-x_i)^\mu\partial_\mu f(x_i) + \tfrac12 (x_j-x_i)^\mu (x_j-x_i)^\nu \partial_\mu\partial_\nu f(x_i) + O(|x_j-x_i|^3).
\end{equation}
The constant term cancels; the linear term vanishes by local isotropy,
\begin{equation}
  \sum_j w_{ij} (x_j-x_i)^\mu \;=\; 0.
\end{equation}
Thus
\begin{equation}
  (L f)_i \;=\; -\tfrac12 \sum_j w_{ij} (x_j-x_i)^\mu (x_j-x_i)^\nu \partial_\mu\partial_\nu f(x_i) + O(R^3\,\|\nabla^3 f\|),
\end{equation}
with $R$ the neighborhood radius. Define the local second\,moment tensor
\begin{equation}
  M^{\mu\nu}(x_i) := \sum_j w_{ij} (x_j-x_i)^\mu (x_j-x_i)^\nu.
\end{equation}
Near isotropy implies $M^{\mu\nu}(x_i) \approx C(x_i)\,\delta^{\mu\nu}$, whence
\begin{equation}
  (L f)_i \;\approx\; -\tfrac12 C(x_i)\, \delta^{\mu\nu}\, \partial_\mu\partial_\nu f(x_i) \;=\; -\tfrac12 C(x_i)\, \nabla^2 f(x_i).
\end{equation}
Identifying $c_s^2(x_i) \propto C(x_i)$ yields the continuum generator $-\nabla\!\cdot(c_s^2\nabla)$ up to a conventional factor (absorbed into $J'_0$ and units).

\subsection*{(ii) Quadratic form (Dirichlet energy) convergence}
The discrete Dirichlet form is
\begin{equation}
  E_{\rm disc}[f] \,=\, \tfrac12 \sum_{i,j} w_{ij}\, \big(f_i-f_j\big)^2.
\end{equation}
Write $f_j-f_i \approx (x_j-x_i)^\mu \partial_\mu f(x_i)$ to leading order and pass to a Riemann sum over neighbors within radius $R$ under near\,regular sampling. Then
\begin{equation}
  E_{\rm disc}[f] \;\to\; \tfrac12 \int d^D x\; \Big( \sum_j w_{ij} (x_j-x_i)^\mu (x_j-x_i)^\nu \Big) \partial_\mu f\, \partial_\nu f \;=\; \int d^D x\; c_s^2(x)\, |\nabla f|^2,
\end{equation}
with $c_s^2(x) \propto \tfrac12\,\mathrm{tr}\,M^{\mu\nu}(x)$. Hence the graph Dirichlet energy converges to the continuum Dirichlet energy, establishing $L\Rightarrow -\nabla\!\cdot(c_s^2\nabla)$ in the weak (form) sense.

\paragraph{Remarks.} (1) Anisotropy appears as an effective metric $g^{\mu\nu}\propto M^{\mu\nu}$, yielding a Laplace\,Beltrami operator. (2) Boundary conditions and higher\,order corrections scale with $R$ and vanish in the dense, near\,regular limit. (3) A spectral validation (low\,lying eigenvalue matching) provides an empirical cross\,check in simulations.

\section{Fluctuation Spectrum: Gapped Amplitude and Massless Phase}\label{si:gapped-massless}
We derive the low-energy fluctuation spectrum around the homogeneous vacuum and show that (i) amplitude fluctuations are \emph{gapped} with mass $m_\xi>0$ set by the local Mexican-hat potential, and (ii) the phase fluctuation is a \emph{massless} Goldstone mode with linear dispersion $\omega_\phi^2=c_s^2 k^2$.

\paragraph{Assumptions (compact).}
\begin{itemize}[leftmargin=*]
  \item Homogeneous vacuum patch with near-regular connectivity; volume-normalised couplings.
  \item Local on-site potential for the coarse amplitude $w$: $V(w)=-\beta_{\rm pot} w^2 + \gamma w^4$ with $\beta_{\rm pot},\gamma>0$ (stability).
  \item Phase sector quadratic governed by the Laplacian: $\tfrac{\kappa w_*^2}{2}|\nabla\phi|^2$ in the continuum; $c_s^2=\kappa w_*^2$.
  \item Narrowband/SVEA regime for separating amplitude and phase around a nonzero vacuum $w_*$. 
\end{itemize}

\subsection*{Vacuum and parameter definitions}
The on-site potential $V(w)=-\beta_{\rm pot} w^2 + \gamma w^4$ is minimised at
\begin{equation}
  w_*^2 = \frac{\beta_{\rm pot}}{2\gamma}, \qquad V''(w_*) = 2\beta_{\rm pot} \equiv m_\xi^2 > 0.
\end{equation}
Clamping to the homogeneous vacuum ($w\approx w_*$) and writing $\Psi = (w_*+\delta w)\,e^{i\phi}$, the coarse-grained leading-gradient energy is
\begin{equation}
  \mathcal L \;=\; \tfrac{1}{2}\,\alpha_{\rm grad} (\nabla\,\delta w)^2 \; -\; \tfrac{1}{2}\, m_\xi^2 (\delta w)^2 \; +\; \tfrac{\kappa w_*^2}{2}\, (\partial_t\phi)^2 \; -\; \tfrac{\kappa w_*^2 c_s^2}{2}\, (\nabla\phi)^2 \; +\; \cdots,
\end{equation}
where dots denote higher-order and amplitude–phase mixing terms suppressed in the SVEA.

\subsection*{Quadratic expansion and equations of motion}
To quadratic order and neglecting mixing, the Euler–Lagrange equations are
\begin{align}
  &\text{Amplitude:} && \alpha_{\rm grad}\,\nabla^2\,\delta w \; -\; m_\xi^2\,\delta w \;=\; 0, \\
  &\text{Phase:} && \partial_t^2\phi \; -\; c_s^2\,\nabla^2\phi \;=\; 0.
\end{align}
Fourier decomposing $\delta w,\phi \propto e^{i(\mathbf k\cdot\mathbf x - \omega t)}$ yields the dispersion relations
\begin{equation}
  \omega_w^2(\mathbf k) \;=\; m_\xi^2 \; +\; \alpha_{\rm grad}\, |\mathbf k|^2, \qquad \omega_\phi^2(\mathbf k) \;=\; c_s^2\, |\mathbf k|^2.
\end{equation}
Thus $\omega_w(\mathbf k\to 0)=m_\xi>0$ (\emph{gapped}), whereas $\omega_\phi(\mathbf k\to 0)=0$ (\emph{massless}).

\subsection*{Goldstone protection of the phase mode}
The phase field inherits a shift symmetry $\phi\to\phi+\text{const}$ from global $U(1)$ invariance of the coarse free energy. This forbids a non-derivative $\phi^2$ term in the quadratic Lagrangian and protects the masslessness of $\phi$ at leading order. Allowed couplings at lowest order are derivative (e.g., $\partial_\mu\phi\,J^\mu$), which do not generate a static mass term in the homogeneous vacuum.

\subsection*{Robustness and mixing}
Amplitude–phase mixing terms (e.g., $\delta w\, (\partial\phi)^2$) are suppressed near the homogeneous vacuum and in the narrowband/SVEA regime. They renormalise coefficients but do not close the phase gap or remove the amplitude gap. Breaking the $U(1)$ explicitly would introduce a small $\phi$ mass; in the present framework this is forbidden at leading order by symmetry and vacuum relaxation.

\paragraph{Outcome.} The low-energy spectrum on near-regular vacuum patches consists of (i) a massive amplitude mode with Yukawa screening length $\ell=\sqrt{\alpha_{\rm grad}/m_\xi^2}$ and (ii) a massless phase mode with wave speed $c_s=\sqrt{\kappa}\,w_*$.

\section{Vacuum Marginalisation (full derivation)}\label{si:vac-marg}
We derive the coarse functional $F_{\rm vac}[J]$ by integrating out the fast phase field on a fixed connectivity background. Let $L(J)$ be the graph Laplacian of $J$ on a connected graph. Fix the zero mode by the gauge $\sum_k \phi_k=0$ (equivalently, use the pseudo-determinant $\det{}'\!L$). Consider
\begin{equation}
  F[J,\phi] \,=\, \frac{\kappa}{2}\,\phi^T L(J)\,\phi \; -\; T\, S[J] \; +\; \mathcal B[J],\qquad S[J]=\sum_k\sum_j \Big(\tfrac{J_{kj}}{Z_k}\Big)\ln\Big(\tfrac{J_{kj}}{Z_k}\Big),
\end{equation}
with $Z_k=\sum_j J_{kj}$ and a convex bias $\mathcal B[J]$ that penalises increases of the intrinsic/spectral dimension. The partition function of the fast sector is Gaussian,
\begin{equation}
  Z[J] \,=\, \int \!\mathcal D\phi\; e^{-F[J,\phi]/T_{\rm eff}} \;\propto\; \big[\det{}' L(J)\big]^{-1/2},
\end{equation}
so the effective functional for slow links is, up to an additive constant,
\begin{equation}
  \Gamma[J] \,=\, -T_{\rm eff}\ln Z[J] \,=\, \frac{T_{\rm eff}}{2}\,\ln\det{}' L(J) \; -\; T\,S[J] \; +\; \mathcal B[J].
\end{equation}
Using the heat-kernel identity $\ln\det{}' L = -\int_{\varepsilon}^{\infty} \!\frac{dt}{t}\,[K(t)-1]$ with $K(t):=\Tr(e^{-tL})$, and defining the intrinsic dimension in the diffusive window by $\langle d_{\rm int}\rangle:=-2\,d\ln K/d\ln t$, we choose the convex surrogate $\mathcal B[J]=\beta\, (\langle d_{\rm int}\rangle[J]-d^\star)^2$. Dropping constants gives
\begin{equation}
  F_{\rm vac}[J] \,=\, \frac{T_{\rm eff}}{2}\,\ln\det{}' L(J) \; -\; T\,S[J] \; +\; \beta\,(\langle d_{\rm int}\rangle[J]-d^\star)^2,
\end{equation}
subject to the per-node budget constraints $\sum_j J_{kj}=Z_k$ and nonnegativity $J_{kj}\ge 0$. Stationarity w.r.t. a single link yields
\begin{equation}
  0 \,=\, \frac{T_{\rm eff}}{2}\,\Tr\big[ L(J)^{-1}\,\partial L/\partial J_{kj}\big] \; -\; \frac{\partial H}{\partial J_{kj}} \; +\; 2\beta\,(\langle d_{\rm int}\rangle-d^\star)\,\frac{\partial\langle d_{\rm int}\rangle}{\partial J_{kj}} \; +\; \lambda_k,
\end{equation}
with $\partial L/\partial J_{kj} = E_{kk}+E_{jj}-E_{kj}-E_{jk}$ and $\partial H/\partial J_{kj}=(1/Z_k)\,(1+\ln(J_{kj}/Z_k))$. This makes the link update/equilibrium conditions fully explicit.

\section{Discrete vs. Continuum Scaling}\label{si:scaling}

\subsection{The Scaling Dichotomy}

The emergent stiffness $\gamma$ scales differently in discrete and continuum models:

\textbf{Note.} Throughout the main text we adopt the volume-normalised discrete-to-continuum mapping where the effective on-site stiffness scales extensively, $\gamma \propto V_K$. The quadratic continuum scaling below refers to the unnormalised double-integral counting and is included for contrast.

\textbf{Discrete Model}: $\gamma \propto V_K$ (linear in grain node count)

\textbf{Continuum Model}: $\gamma \propto V_K^2$ (quadratic in geometric volume)

\subsection{Mathematical Origin}

This difference arises from how interactions are counted:
\begin{itemize}
\item \textbf{Discrete}: Sum over $V_K^2/2$ pairs of grain nodes, each with strength $\eta_0/V_K$
\item \textbf{Continuum}: Double integral $\iint \eta_\rho |\Psi(x)|^2|\Psi(y)|^2 dV_x dV_y$
\end{itemize}

The scaling difference is mathematically necessary when moving from discrete sums to continuous integrals.

\subsection{Practical Application}

\begin{itemize}
\item \textbf{Use linear scaling} ($\gamma \propto V_K$) for graph simulations and computational work
\item \textbf{Use quadratic scaling} ($\gamma \propto V_K^2$) for analytical calculations and continuum field theory
\end{itemize}

\section{Graph Laplacian and Gradient Energy}\label{si:laplacian}

\subsection{Quadratic Form Representation}

The discrete gradient energy can be written as:
\begin{equation}
E_{\text{gradient}} = \frac{1}{2} A^T L A
\end{equation}
where $L$ is the graph Laplacian matrix and $A$ is the field vector.

\subsection{Spectral Properties}

For a linear gradient $A(x) = p\cdot x$, the energy becomes:
\begin{equation}
E_{\text{gradient}} = \frac{1}{2} p^T M p
\end{equation}
where $M = \int x x^T dV$ is the moment tensor of the grain.

\subsection{Isotropic Case}

For a spherical grain: $M \propto V_K R_K^2 I$, giving $E_{\text{gradient}} \propto V_K R_K^2 |p|^2$ and thus $M_p \propto V_K R_K^2$.

\section{Renormalization and Stability}\label{si:renorm}

\subsection{Volume Normalization as Renormalization}

The volume normalization $J_{ij} = \eta_0/V_K$ is a form of mean-field renormalization that:
\begin{itemize}
\item Prevents divergences in the continuum limit
\item Ensures extensive scaling of emergent parameters
\item Maintains finite energy densities
\end{itemize}

\subsection{Self-Organization Principle}

The model contains built-in stability: high-energy gradient configurations across long-range links are suppressed by their own weakness ($J \propto |\Psi_i|^2|\Psi_j|^2$), leading to emergent local smoothness.

\subsection{Numerical Considerations}

Simulations should:
\begin{itemize}
\item Use volume-normalized couplings
\item Monitor energy density for stability
\item Implement appropriate cutoffs for very long-range interactions
\end{itemize}

\subsection*{Scale-space RG fixed point for the bath spectrum}
\textbf{Theorem (RG fixed point, stability).} Consider coarse-graining that (i) preserves the Lorentzian principal symbol of the phase operator, (ii) implements local-unit (conformal) rescaling of tetrads, and (iii) integrates out modes outside a comoving window $[k_{\min},k_{\max}]$ while rescaling $k\to k/b$ (with $b>1$) to restore the window. Let $\sigma(u):=k\,S(k)$ with $u=\ln k$ the logarithmic wavenumber. Then:
\begin{itemize}
  \item The uniform spectrum in $u$\,--\,space, $\sigma(u)=\text{const}$ (equivalently $S(k)\propto 1/k$), is a fixed point of the coarse-graining map.
  \item Small tilts $\sigma(u)\propto e^{\epsilon u}$ (i.e., $S(k)\propto k^{-1+\epsilon}$ with $|\epsilon|\ll1$) flow back to the fixed point for a broad class of window\,--\,rescaling schemes, i.e., the fixed point is stable (attractive) to first order in $\epsilon$.
\end{itemize}
\textbf{Sketch of proof.} Under a single RG step: (1) integrate out a thin outer band and rescale $k\to k/b$ to restore the original window; (2) preserve the phase principal symbol $-\nabla\!\cdot(c_s^2\nabla)$ and apply the local-unit rescaling (tetrads scaled so locally measured $c_s$ remains invariant). In the log variable $u=\ln k$, this amounts to a shift $u\to u-\ln b$ plus normalization that maintains the total band power per $du$ (stationarity in local units). The induced map on $\sigma(u)$ is a shift followed by reweighting that preserves $\int_{u_{\min}}^{u_{\max}}\sigma(u)\,du$. Hence $\sigma(u)=\text{const}$ is a fixed point. For a small tilt $\sigma(u)=\sigma_0 e^{\epsilon u}$, Taylor expanding shows that after the shift $u\to u-\ln b$ and normalization, the effective tilt rescales as $\epsilon' = \epsilon\,(1-\alpha\ln b)$ with $\alpha>0$ set by the normalization procedure and window geometry; thus $|\epsilon'|<|\epsilon|$ for modest $\ln b>0$, proving linear stability. This is the RG counterpart of the maximum-entropy result in the Dark Sector SI: equal content per logarithmic band is both the entropy maximizer and the RG fixed point under admissible coarse-graining compatible with the principal symbol and local units.
\medskip
\textbf{Remarks.} (R1) The specific value of $\alpha$ depends on the edge treatment (how the integrated-out power is redistributed under rescaling) but remains positive for any scheme that preserves stationary band power in local units. (R2) Finite windows ensure integrability and introduce only $\mathcal O(1/\mathcal N)$ corrections controlled by the log\,--\,bandwidth $\mathcal N=\ln(k_{\max}/k_{\min})$. (R3) Anisotropies or mild dispersion alter higher\,--\,order flow but not the existence nor the attractiveness of the $1/k$ fixed point at leading order in the Coulombic window.

\section{Dimensional Fixed\,Point (Lambert\,W derivation)}\label{si:dim-fp}

\subsection{Notation and scope}

Throughout this appendix, $d$ denotes the effective intrinsic/spectral dimension obtained from the heat\,kernel definition in a chosen diffusive window for a locally homogeneous vacuum patch. It is real\,valued in general. The symbol $d^\star$ denotes the stationary (target) value that minimises the reduced cost functional $F(d)$; in the vacuum this acts as a fixed point toward which coarse\,grained dynamics drive $d$.

\subsection{Spectral dimension from the heat kernel}

For a weighted graph with Laplacian $L(J)$, the heat kernel is $K(t) = \text{Tr}[e^{-t L(J)}]$. In a diffusive window of times $t$, the spectral dimension is defined by
\begin{equation}
\langle d_{\text{int}} \rangle(t) := - 2 \cdot \frac{d \ln K(t)}{d \ln t}
\end{equation}

This is a derived observable of the propagation kernel; no integer constraint is imposed. In practice, we work in a window where $\langle d_{\text{int}} \rangle(t)$ is approximately flat in $t$.

\subsection{Information\,theoretic cost and dimensional stiffness}

Two ingredients determine the vacuum's preferred intrinsic dimension:

1. A storage/description cost that decreases with dimension because screening shortens effective range and sparsifies long links. A simple model is
\begin{equation}
L(d) = b_V + b_E \cdot e^{- \alpha d}
\end{equation}
with $b_V, b_E, \alpha > 0$. Here $b_E e^{-\alpha d}$ is the \emph{coarse-grained information cost} of active links at dimension $d$—the effective description length induced by integrating out microscopic fluctuations (cf. the $-T S[J]$ term in Axiom~2). Long links and high degree raise spectral dimension via the Laplacian spectrum (Sec.~\ref{si:laplacian}), increasing this cost.

2. A dimensional stiffness that penalises departures from the vacuum's compressive preference, captured by a convex quadratic $\beta d^2$ at leading order (coarse\,grained surrogate of the bias used in Section 6.1).

We therefore minimise the reduced cost (dropping constants):
\begin{equation}
F(d) = b_E e^{- \alpha d} + \beta d^2
\end{equation}

\subsection{Fixed\,point and Lambert\,W}

Stationarity gives
\begin{align}
\frac{\partial F}{\partial d} &= - \alpha b_E e^{- \alpha d} + 2 \beta d = 0\\
&\Rightarrow 2 \beta d = \alpha b_E e^{- \alpha d}\\
&\Rightarrow (\alpha d) e^{\alpha d} = \frac{\alpha^2 b_E}{2 \beta}
\end{align}

Hence the target intrinsic dimension is
\begin{equation}
d^\star = \frac{1}{\alpha} \cdot W\left( \frac{\alpha^2 b_E}{2 \beta} \right)
\end{equation}
where $W$ is the Lambert\,W function. For plausible parameter ranges, this lands near a small integer ($\approx 3$), explaining the preference for low\,integer coordination in the vacuum.

\subsection{Remarks}

\begin{itemize}
\item Using spectral dimension makes the criterion basis\,independent and sensitive to both degree and long\,link patterns.
\item The quadratic stiffness is the leading convex surrogate; higher\,order corrections can be absorbed into renormalised $(\alpha, b_E, \beta)$ without changing the Lambert\,W structure.
\item \textbf{Robustness to information\,cost choice.} The exponential $b_E e^{-\alpha d}$ is the simplest monotone decreasing ansatz. Replacing it by other reasonable models still yields a unique stable minimum: (i) for a power\,law $L(d)=b_E d^{-p}$ with $p>0$, minimising $F(d)=L(d)+\beta d^2$ gives $d_*\propto (b_E/\beta)^{1/(p+2)}$; (ii) for a logarithmic form $L(d)=-b_E\log(\alpha d)$ one finds $d_*\propto\sqrt{b_E/\beta}$. Thus the existence of a small, stable fixed point is generic; the exponential ansatz gives the compact Lambert\,W expression.
\item Section 6.1 uses a weak\,bias regime where the bias chiefly sets a coordination ``temperature,'' stabilising small\,integer valence without overriding the local Boltzmann link form.
\end{itemize}

\section{Measurement Double\,Well and Pointer States}\label{si:measure-dw}
We outline how the coupled system–apparatus landscape generically acquires a double\,well that implements projective, latching measurement.

\subsection*{Coarse\,grained apparatus model}
Let the apparatus $A$ be a grain (or a small bundle of grains) with an internal phase current $p_A$ that couples to a system mode (phase orientation) $p_S$. The coarse\,grained free energy at fixed amplitude reads
\begin{equation}
  F(p_A;p_S) \;=\; \tfrac12 M_p |p_A|^2 \; -\; C\,(p_A\!\cdot\!p_S) \; +\; \lambda |p_A|^4,
\end{equation}
with $M_p>0$ the phase inertia, $C>0$ an attractive phase\,channel coupling (composition\,independent), and $\lambda>0$ a weak stabiliser summarising higher\,order terms and amplitude–phase backreaction.

For weak coupling ($C$ small), the unique minimum is $p_A=0$ (no record). Above a threshold $C\gtrsim C_\mathrm{crit}\sim M_p^2/\lambda$ the origin loses stability and two symmetry\,related minima appear at $p_A=\pm p_*\,\hat p_S$ with $p_*\propto C/M_p$ (pitchfork bifurcation). These minima correspond to macroscopically distinct apparatus records aligned/anti\,aligned with the system phase.

\subsection*{Relative\,phase reduction}
Clamping amplitudes near $w_*$ and projecting onto the single relative phase $\theta=\phi_A-\phi_S$ gives the effective interaction $V_{\mathrm{rel}}(\theta)\approx -K\cos\theta+O(\cos2\theta)$ with $K\propto C$. The minima at $\theta=0\,(\mathrm{mod}\,2\pi)$ (and, when higher harmonics are relevant, $\theta=0,\pi$) furnish the discrete pointer states.

\subsection*{Decoherence and hysteresis}
The apparatus–bath coupling acts through the same phase sector. Off\,diagonal coherences between distinct minima dephase at a rate set by the bath's low\,frequency power, $\Gamma_{\mathrm{dephase}}\sim S_\xi(0)/(2\hbar_{\mathrm{eff}})$, as the macroscopically different phase\,current patterns imprint distinguishable bath states. The quartic stabiliser creates a barrier $\Delta E_b$; with damping $\zeta$ and effective temperature $T_{\mathrm{eff}}$, the Kramers escape rate $\Gamma\approx(\omega_0\omega_b/2\pi\zeta)\,e^{-\Delta E_b/(k_B T_{\mathrm{eff}})}$ is exponentially small in the measurement regime. Thus the minima are dynamically selected (einselection) and latched (hysteresis), realising projective measurement onto the apparatus' eigenbasis.

\section{Tipping Point Criteria for Measurement Regimes}\label{si:thresholds}
As introduced in Section 11 of the main paper, the transition from continuous tracking to projective measurement involves crossing several distinct physical thresholds. Here, we provide the precise mathematical conditions for these tipping points, formulated within the free\,energy landscape of Eq.~(\ref{eq:VtotalSA}) of the main text.

The key parameters are the system–apparatus coupling ($K \propto C$), the apparatus's internal even-harmonic potential scale ($B$), its stabilizing nonlinearity ($\lambda_A$), its phase inertia ($M_p$), and the bath properties (damping $\zeta$, effective temperature $T_{\rm eff}$, and low-frequency noise power $S_\xi(0)$).

\paragraph{1. Onset of Bistability (Two Outcomes).} The fundamental switch from a single continuous pointer to two discrete pointer states occurs when the apparatus potential develops a double-well structure. This happens when the even-harmonic term overcomes the centering effect of the system coupling. The exact threshold is:
\begin{equation}
  B = B_{\rm crit} = \frac{K}{4}.
\end{equation}
For $B > B_{\rm crit}$, two stable pointer minima exist.

\paragraph{2. Onset of Orthogonality (Which-Way Information).} For the two pointer states to represent distinct, non-interfering information, their overlap must be negligible. This is governed by the action separation $\mathcal{A}$ between them. Using the macroscopic phase-current difference $\Delta p_A \approx 2C/M_p$, the tipping point where interference is suppressed is:
\begin{equation}
  \frac{\mathcal{A}}{\hbar_{\mathrm{eff}}} \approx \frac{\pi \, \Delta p_A}{\hbar_{\mathrm{eff}}} \approx \frac{2\pi C}{M_p \, \hbar_{\mathrm{eff}}} = 1.
\end{equation}

\paragraph{3. Onset of Decoherence (Loss of Superposition).} A superposition of pointer states is destroyed when the environment can distinguish them faster than the system evolves. This occurs when the decoherence rate, $\Gamma_{\rm dephase}$, dominates over the relevant integration time, $T_{\rm int}$. The tipping point is:
\begin{equation}
  \Gamma_{\rm dephase} T_{\rm int} \approx \frac{(\Delta p_A)^2 S_\xi(0)}{2\,\hbar_{\mathrm{eff}}} \, T_{\rm int} = 1.
\end{equation}

\paragraph{4. Onset of Latching (Durable Record).} A measurement is recorded classically when the pointer is trapped in one minimum, unable to tunnel to the other. This requires the energy barrier $\Delta E_b$ to be large compared to the thermal energy of the bath. The tipping point for a stable record is:
\begin{equation}
  \frac{\Delta E_b}{k_B T_{\rm eff}} \sim \frac{C^2}{\lambda_A \, k_B T_{\rm eff}} = 1.
\end{equation}

\paragraph{Conformally Invariant Control.} These four conditions define the boundaries between measurement regimes. They can be packaged into five dimensionless ratios: $\alpha = B/K$, $\beta = \lambda_A/K$, and the three tipping parameters $\Xi_1, \Xi_2, \Xi_3$ defined by the expressions above. Because these ratios are constructed from quantities that co-vary under changes in the observational window, the regime boundaries are conformally invariant in local units, a key prediction of the framework.

\paragraph{Conformal behavior (local\,unit invariance).} In our framework, predictions are expressed in local units adapted to the coarse\,graining window: $c_s$ is fixed by $c_s^2=\kappa w_*^2$, and $\hbar_{\rm eff}\simeq E_{\rm cell}\,\tau_{\rm cell}$ is constructed to be coarse\,grain invariant. Under admissible window changes, the microscopic quantities $K$, $B$, $\lambda_A$, $M_p$, $S_\xi(0)$, $T_{\rm eff}$ co\,vary, but the \emph{dimensionless} ratios $\alpha$, $\beta$, $\Xi_1$, $\Xi_2$, $\Xi_3$ remain approximately unchanged. Hence the tipping surfaces are conformally invariant in local units. Thermodynamic convergence implies tiny residual drifts (Sec. 9 of the main paper): these shift thresholds only by higher\,order corrections, yielding at most slow, testable motions of the tipping lines without qualitative change of regimes.

\section{General Interference Suppression for $M$ Outcomes}\label{si:M-outcome-interference}
We provide a detailed derivation that any multi-state phase alignment necessarily suppresses interference, quantifying visibility in terms of apparatus overlaps and action separation.

\subsection*{Setup}
Let $\{|m\rangle\}_{m=1}^M$ be orthonormal system channels (eigenvectors, slits, bins), and let the initial system state be $|\psi_S\rangle = \sum_m c_m |m\rangle$. The apparatus starts in $|A_0\rangle$ and couples through the phase channel so that near a given outcome $m$ the apparatus pointer relaxes into a distinct phase-current minimum $|A_m\rangle$. The joint state after the measurement interaction window is
\begin{equation}
  |\Psi_{SA}\rangle \;=\; \sum_{m=1}^M c_m\, |m\rangle\otimes|A_m\rangle,\qquad |A_m\rangle \in \mathcal H_A.
\end{equation}

\subsection*{Reduced state and coherence factors}
Tracing out the apparatus gives
\begin{equation}
  \rho_S' \;=\; \mathrm{Tr}_A\,|\Psi_{SA}\rangle\langle\Psi_{SA}| \;=\; \sum_{m} |c_m|^2 |m\rangle\langle m| \; +\; \sum_{m\ne n} c_m c_n^{\!*}\, \gamma_{mn}\, |m\rangle\langle n|,
\end{equation}
where the overlap (decoherence) factors are
\begin{equation}
  \gamma_{mn} \,=\, \langle A_n|A_m\rangle,\qquad \gamma_{mm}=1,\qquad |\gamma_{mn}|\le 1.
\end{equation}
Hence off-diagonal coherences are universally suppressed by $\gamma_{mn}$. For any observable whose interference term depends on $\langle m|\rho_S'|n\rangle$ ($m\ne n$), the visibility is reduced by $|\gamma_{mn}|$.

\subsection*{Two-path intensity and visibility}
For $M=2$ and a detection amplitude at position $x$ given by $\psi(x)=\psi_1(x)|1\rangle+\psi_2(x)|2\rangle$, the intensity reads
\begin{equation}
  I(x) = |\psi_1(x)|^2 + |\psi_2(x)|^2 + 2\,\mathrm{Re}\big[\,\gamma_{12}\,\psi_1(x)\,\psi_2(x)^{\!*}\,\big].
\end{equation}
Thus the fringe visibility satisfies $\mathcal V=\mathcal V_0\,|\gamma_{12}|$, where $\mathcal V_0$ is the ideal visibility without coupling.

\subsection*{Macroscopic pointer overlaps from action separation}
In this framework, the $|A_m\rangle$ are macroscopic phase-current patterns localized in distinct minima of the apparatus potential (Sec.~\ref{si:measure-dw}). The difference between two outcomes $(m,n)$ is characterized by a current separation $\Delta p_A^{(mn)}$. Standard semiclassical estimates give
\begin{equation}
  |\langle A_n|A_m\rangle| \;\approx\; \exp\!\Big(-\frac{\mathcal A_{mn}}{\hbar_{\mathrm{eff}}}\Big),\qquad \mathcal A_{mn} \sim \pi\, \frac{\Delta p_A^{(mn)}}{\hbar_{\mathrm{eff}}},
\end{equation}
consistent with the tipping criteria in Eq.~(S\,\ref{si:thresholds}). Therefore, in the projective+latched regime one has $\mathcal A_{mn}\gg \hbar_{\mathrm{eff}}$ and $\gamma_{mn}\approx 0$ for all $m\ne n$, yielding complete suppression of interference between distinct outcomes.

\subsection*{Continuous (weak) alignment and partial decoherence}
In the weak/continuous regime the apparatus minima are shallow and the induced $|A_m\rangle$ are only partially distinguishable, producing $0<|\gamma_{mn}|<1$. The same formulas predict quantitatively how visibility degrades with coupling strength and with increasing action separation. Environmental dephasing further multiplies $\gamma_{mn}$ by a factor $e^{-\Gamma_{\rm dephase}T_{\rm int}}$ derived in Sec.~\ref{si:thresholds}, reinforcing suppression over the integration time.

\subsection*{Degeneracies and residual interference}
If multiple system channels map to the \emph{same} pointer state (degenerate outcome), their mutual overlaps satisfy $\gamma_{mn}=1$ within that degenerate class, so interference among those channels can persist. Interference is destroyed \emph{iff} the measurement establishes distinguishable apparatus states for the channels in question, i.e. $\gamma_{mn}\to 0$.

\paragraph{Conclusion.} Any multi-state phase alignment generates entanglement with macroscopically distinct apparatus states; the overlap factors $\gamma_{mn}$ universally multiply off-diagonal coherences and thus control interference visibility. In the latched regime $\gamma_{mn}\approx 0$ and interference vanishes; in continuous readout $\gamma_{mn}$ interpolates between 1 and 0, yielding partial visibility consistent with the thresholds in Sec.~\ref{si:thresholds}.

\subsection*{Partial Measurement and Intermediate Visibility}
The continuous nature of the pointer state overlap $\gamma_{mn}$ implies that measurement itself is not a binary event but a continuous spectrum. An ideal projective measurement corresponds to the limit $|\gamma_{mn}|\to 0$, while the absence of measurement corresponds to $|\gamma_{mn}|=1$. The intermediate case, $0 < |\gamma_{mn}| < 1$, describes a \emph{partial measurement}.

This regime arises when the interaction is sufficient to distinguish the system's paths partially, but not perfectly. In the context of this framework, this corresponds to shallow potential minima for the apparatus's phase current, where the pointer states $|A_m\rangle$ and $|A_n\rangle$ are distinct but not orthogonal. The fringe visibility $\mathcal V$ provides a direct measure of the degree of distinguishability. For the two-path case, the intensity is
\begin{equation}
  I(x) = |\psi_1(x)|^2 + |\psi_2(x)|^2 + 2|\gamma_{12}|\,|\psi_1(x)|\,|\psi_2(x)^{\!*}|\,\cos(\delta(x)+\phi_{12}),
\end{equation}
where $\delta(x)$ is the phase difference from the path lengths and $\phi_{12}$ is the phase of the complex overlap $\gamma_{12}$. The standard visibility is $\mathcal V = (I_{\max}-I_{\min})/(I_{\max}+I_{\min})$. For the common case $|\psi_1(x)|=|\psi_2(x)|$, this simplifies to
\begin{equation}
  \mathcal V = |\gamma_{12}|.
\end{equation}
Thus, a partial measurement yields partial which-way information and results in a partially washed-out interference pattern, with the visibility providing a quantitative measure of the pointer state overlap.

\subsection*{Application: The Quantum Eraser}
The quantum eraser is a profound demonstration of these principles, showing that which-way information can be statistically `erased' via post-selection. Our framework provides a clear, step-by-step mechanism for this phenomenon without invoking retrocausality.

\paragraph{Step 1: Entanglement with a Which-Way Detector.}
Let the system be a particle passing through two slits, $S$, initially in state $|\psi_S\rangle = c_1|1\rangle + c_2|2\rangle$. A Which-Way Detector (WWD), which is an apparatus $A$ in our formalism, is placed at the slits. After the interaction, the system and WWD become entangled. If the WWD works perfectly, it transitions to orthogonal internal states $|A_1\rangle$ and $|A_2\rangle$ depending on the slit, with $\langle A_1|A_2\rangle=0$. The joint state is:
\begin{equation}
  |\Psi_{SA}\rangle = c_1|1\rangle|A_1\rangle + c_2|2\rangle|A_2\rangle.
\end{equation}
At this stage, tracing over the WWD gives a diagonal reduced density matrix for the system, $\rho_S = |c_1|^2|1\rangle\langle 1| + |c_2|^2|2\rangle\langle 2|$, and the interference pattern at the screen is destroyed.

\paragraph{Step 2: The Eraser Measurement.}
The `eraser' is a second measurement performed not on the system $S$, but on the WWD $A$. Critically, this measurement is performed in a basis that does not distinguish between $|A_1\rangle$ and $|A_2\rangle$. Let's define the `eraser basis' as:
\begin{align}
  |+\rangle_A &= \frac{1}{\sqrt{2}}(|A_1\rangle + |A_2\rangle) \\
  |-\rangle_A &= \frac{1}{\sqrt{2}}(|A_1\rangle - |A_2\rangle)
\end{align}

\paragraph{Step 3: State Representation in the Eraser Basis.}
The core of the phenomenon is revealed by rewriting the entangled state $|\Psi_{SA}\rangle$ in this new basis. We first invert the transformation:
\begin{align}
  |A_1\rangle &= \frac{1}{\sqrt{2}}(|+\rangle_A + |-\rangle_A) \\
  |A_2\rangle &= \frac{1}{\sqrt{2}}(|+\rangle_A - |-\rangle_A)
\end{align}
Now, we substitute these expressions back into the joint state $|\Psi_{SA}\rangle$:
\begin{equation}
  |\Psi_{SA}\rangle = c_1|1\rangle \frac{1}{\sqrt{2}}(|+\rangle_A + |-\rangle_A) + c_2|2\rangle \frac{1}{\sqrt{2}}(|+\rangle_A - |-\rangle_A)
\end{equation}
Regrouping the terms by the eraser basis states, $|+\rangle_A$ and $|-\rangle_A$, yields the final, crucial form:
\begin{equation}
  |\Psi_{SA}\rangle = \frac{1}{\sqrt{2}} \bigg[ \underbrace{\left( c_1|1\rangle + c_2|2\rangle \right) \otimes |+\rangle_A}_{\text{Interference sub-ensemble}} + \underbrace{\left( c_1|1\rangle - c_2|2\rangle \right) \otimes |-\rangle_A}_{\text{Anti-interference sub-ensemble}} \bigg]
\end{equation}

\paragraph{Step 4: Post-Selection and Interference Recovery.}
The final state shows that the system's state is now correlated with the outcome of the eraser measurement.
\begin{itemize}
    \item If we measure the WWD and find it in the state $|+\rangle_A$, the system is projected into the state $|\psi_+\rangle_S \propto c_1|1\rangle + c_2|2\rangle$. The sub-ensemble of particles corresponding to this eraser outcome will form a perfect interference pattern on the screen.
    \item If we measure the WWD and find it in the state $|-\rangle_A$, the system is projected into $|\psi_-\rangle_S \propto c_1|1\rangle - c_2|2\rangle$. This sub-ensemble will form an anti-interference pattern (where the bright and dark fringes are swapped).
\end{itemize}
If we simply observe all particles hitting the screen without regard to the eraser outcome, we are summing the two patterns, which are perfectly out of phase. The total intensity is a featureless classical distribution. The interference is recovered only when we \emph{post-select} the data, using the eraser's results to sort the particles into the appropriate sub-ensemble. The which-way information is never truly destroyed; it is merely transferred into the correlation between the particle and the eraser, making it unavailable for any given sub-ensemble.

\subsection*{Visibility–Distinguishability Bound (Englert) and Mixed Pointers}
For two alternatives, define the distinguishability of the apparatus states by the trace distance $D := \tfrac12\,\lVert \rho_A^{(1)} - \rho_A^{(2)} \rVert_1$, where $\rho_A^{(m)}$ is the apparatus state conditional on alternative $m$. Let the (maximal) fringe visibility be $\mathcal V$. Then the Englert inequality holds
\begin{equation}
  \mathcal V^2 + D^2 \;\le\; 1.
\end{equation}
Sketch: For pure pointer states $|A_1\rangle,|A_2\rangle$, one has $D = \sqrt{1 - |\langle A_1|A_2\rangle|^2}$ and --- for balanced intensities --- $\mathcal V = |\langle A_1|A_2\rangle|$, saturating $\mathcal V^2+D^2=1$. For mixed $\rho_A^{(m)}$, $\mathcal V$ is upper-bounded by the Uhlmann fidelity $F(\rho_A^{(1)},\rho_A^{(2)})$ via $\mathcal V \le F$, while $D \ge \sqrt{1-F^2}$ (Fuchs–van de Graaf), which implies the inequality.

In our framework, pointer mixedness arises from environmental coupling or coarse-grain averaging. The result shows rigorously that partial which-way information ($D>0$) necessarily reduces visibility ($\mathcal V<1$); full which-way information ($D=1$) forces $\mathcal V=0$.

\paragraph{Empirical assumptions: pointer orthogonality and mixedness.}
\emph{Pointer orthogonality.} In laboratory apparatus, macroscopically distinct pointer configurations (e.g., different phase-current minima in our model) occupy effectively disjoint regions of a very large Hilbert space. In the present framework, the inner product obeys $|\langle A_n|A_m\rangle|\approx \exp(-\mathcal A_{mn}/\hbar_{\rm eff})$ with $\mathcal A_{mn}\sim \pi\,\Delta p_A^{(mn)}/\hbar_{\rm eff}$. When the coupling $K$ exceeds the bistability threshold and the barrier is appreciable (measurement regime), $\Delta p_A^{(mn)}$ is macroscopic in local units, yielding $\mathcal A_{mn}\gg 1$ and hence practical orthogonality. Weak residual overlaps, if present, show up exactly as finite $|\gamma_{mn}|$ and the predicted partial visibility.

\emph{Pointer mixedness.} Real pointers are open systems. Even if the conditional apparatus states originate as pure, uncontrolled bath modes and coarse-grain averaging render the effective pointer states $\rho_A^{(m)}$ mixed on the analysis window. Mixedness increases distinguishability through the environment, typically raising the trace distance $D$ and decreasing the visibility bound via $\mathcal V \le F(\rho_A^{(1)},\rho_A^{(2)})$. In our model, the same bath that dephases distinct minima (rate $\Gamma_{\rm dephase}$) also relaxes the apparatus toward outcome-dependent stationary states, making $\rho_A^{(m)}$ stable over the integration time and justifying the use of fixed $\rho_A^{(m)}$ in the inequality.

\emph{Operational tests.} Orthogonality/mixedness can be inferred without full tomography by: (i) measuring visibility versus controlled coupling time/strength to fit $|\gamma_{mn}|$ (or fidelity $F$); (ii) performing eraser-basis rotations and quantifying conditional visibilities (imperfect eraser analysis below); (iii) estimating action separation from independent calibration of $\Delta p_A$ and comparing to the predicted $\exp(-\mathcal A/\hbar_{\rm eff})$ scaling.

\subsection*{Explicit Two-Path Intensities and Imperfect Eraser}
Let detection amplitudes be $\psi_1(x)$ and $\psi_2(x)$ at screen coordinate $x$, with initial weights $c_1,c_2$. Conditioned on eraser outcomes $|\pm\rangle_A= (|A_1\rangle \pm |A_2\rangle)/\sqrt{2}$ (ideal case), the conditional system states are $|\psi_\pm\rangle \propto c_1|1\rangle \pm c_2|2\rangle$, yielding intensities
\begin{equation}
  I_\pm(x) \;=\; \big| c_1\,\psi_1(x) \pm c_2\,\psi_2(x) \big|^2.
\end{equation}
The unconditional pattern (summing both eraser outcomes) is
\begin{equation}
  I(x) \;=\; \tfrac12\,\big(I_+(x)+I_-(x)\big) \;=\; |c_1|^2|\psi_1(x)|^2 + |c_2|^2|\psi_2(x)|^2,
\end{equation}
demonstrating exact classical washout without retrocausality.

For an imperfect eraser with a rotated superposition basis $|e_1\rangle = \cos\theta\,|A_1\rangle + e^{i\varphi}\sin\theta\,|A_2\rangle$, $|e_2\rangle = -e^{-i\varphi}\sin\theta\,|A_1\rangle + \cos\theta\,|A_2\rangle$, the conditional intensities become
\begin{align}
  I_{e_1}(x) &= \big| \cos\theta\, c_1\,\psi_1(x) + e^{i\varphi}\sin\theta\, c_2\,\psi_2(x) \big|^2,\\
  I_{e_2}(x) &= \big| -e^{-i\varphi}\sin\theta\, c_1\,\psi_1(x) + \cos\theta\, c_2\,\psi_2(x) \big|^2.
\end{align}
Thus partial recovery obtains with visibility set by $\theta$ and the relative phase $\varphi$. The unconditional sum remains classical.

\subsection*{No-Signalling and Delayed-Choice}
Let $M_x$ be the POVM element for detection at $x$ on the system, and $E_j$ a POVM on the apparatus performed at any later time. The joint probability is $P(x,j)=\mathrm{Tr}[(M_x\otimes E_j)\,|\Psi_{SA}\rangle\langle\Psi_{SA}|]$. The marginal at the screen is
\begin{equation}
  P(x) \;=\; \sum_j P(x,j) \;=\; \mathrm{Tr}\Big[ (M_x\otimes \mathbb I)\,|\Psi_{SA}\rangle\langle\Psi_{SA}| \Big],
\end{equation}
which is independent of the later choice of $\{E_j\}$. Hence the eraser cannot signal or retroact on $P(x)$; conditional fringes emerge only within sub-ensembles selected by $j$.

\section{Matter--Kernel Coupling Lemma (Static, Linear Response)}\label{si:matter-kernel}
We show that a static energy density $\rho_m(x)$ induces a local perturbation $\delta K(x)$ of the continuum phase kernel such that $\delta K(x) \propto \rho_m(x)$ to leading order.

\paragraph{Assumptions.} Near-regular subgraphs; volume-normalized couplings; time-scale separation (phase fast, links slow); static, small perturbations of a localized soliton core; continuum limit of the Laplacian exists on the patch of interest.

\subsection*{From microscopic links to kernel perturbations}
Let a localized soliton core $\Delta$ of excess energy density $\rho_m(x)$ perturb the equilibrium link weights $J_{ij}\to J_{ij}+\delta J_{ij}$. Stationarity of the coarse functional $\Gamma[J]$ (Sec.~S1) implies, to linear order,
\begin{equation}
  \delta \Gamma \;=\; \frac{1}{2} T_{\rm eff}\,\Tr\big[L(J)^{-1}\,\delta L\big] \; -\; T\,\delta S[J] \; +\; \delta\mathcal B[J] \;=\; 0,
\end{equation}
with $\delta L$ the Laplacian variation induced by $\delta J$. Localized excess energy modifies $J$ in the vicinity of $\Delta$ by renormalising short bonds and screening long bonds. Volume normalisation ensures $\delta J$ remains local and integrable. In the continuum, $L\mapsto -\nabla\!\cdot(c_s^2\nabla)$ and small link changes map to a local perturbation of $c_s^2$:
\begin{equation}
  -\nabla\!\cdot(c_s^2\nabla) \;\longrightarrow\; -\nabla\!\cdot\big((c_s^2+\delta c_s^2)\nabla\big) \;=\; -\nabla\!\cdot(c_s^2\nabla)\; -\; \underbrace{\nabla\!\cdot(\delta c_s^2\nabla)}_{\delta K}.
\end{equation}
To leading order, $\delta c_s^2(x)$ is proportional to the local excess energy density $\rho_m(x)$: $\delta c_s^2(x)=\chi_c\,\rho_m(x)$ with $\chi_c>0$ a susceptibility fixed by microscopic parameters and the window. Hence
\begin{equation}
  \delta K(x) \;=\; -\nabla\!\cdot\big(\chi_c\,\rho_m(x)\,\nabla\big) \;\equiv\; \mathcal C[\rho_m](x),
\end{equation}
defining a local functional $\mathcal C$ of $\rho_m$.

\subsection*{Statement of the lemma}
In the static, weak-perturbation regime on near-regular patches,
\begin{equation}
  \boxed{\quad \delta K(x) \;\propto\; \rho_m(x) \quad \text{(local proportionality, to leading order)}\quad}
\end{equation}
with proportionality mediated by $\mathcal C$ above. Higher derivatives of $\rho_m$ enter at subleading order and are suppressed by the coarse scale.

\paragraph{Remarks.} (i) The sign $\chi_c>0$ is fixed by the increase of local phase stiffness around energy concentrations (cores), consistent with the pruning mechanism. (ii) Nonlocal tails of $\delta J$ renormalise the proportionality but do not change locality at leading order in the Coulombic window. (iii) Time dependence introduces retardation kernels beyond the present static scope.

\subsection*{Proposition (Discrete\,\textrightarrow\,Continuum Matter\,--\,Kernel Coupling)}
\textbf{Hypotheses.} (H1) Near\,--\,regular weighted graph with slowly varying symmetric weights and dense sampling on the region of interest; (H2) volume\,--\,normalised couplings for pair energies; (H3) static, small, localized excess energy density $\rho_m(x)$ (soliton core neighbourhood) that perturbs links only within a bounded neighbourhood; (H4) amplitude sector gapped (screening length $\ell$ finite), massless phase sector governs long\,--\,range; (H5) the standard discrete\,\textrightarrow\,continuum mapping holds for the Laplacian quadratic form.

\textbf{Statement.} Under (H1)\,--\,(H5), the variation of the continuum phase kernel induced by $\rho_m(x)$ is local to leading order and of the divergence form
\begin{equation}
  \delta K(x) \;=\; -\nabla\!\cdot\!\big(\chi_c\,\rho_m(x)\,\nabla\big)\,,\qquad \chi_c>0\,.
\end{equation}
Equivalently, writing the phase operator as $K=-\nabla\!\cdot(c_s^2\nabla)$ one has a local stiffness shift $\delta c_s^2(x)=\chi_c\,\rho_m(x)$ to leading order. The constant $\chi_c$ depends on microparameters (e.g. $\kappa, w_*, \alpha_{\rm grad}, \beta_{\rm pot}$) and the observation window, and is finite for near\,--\,regular families.

\textbf{Proof (sketch).} Start from the discrete phase quadratic form $\tfrac{\kappa}{2}\,\phi^T L(J)\,\phi$ with $L(J)$ the weighted graph Laplacian. A localized excess energy density $\rho_m$ modifies the optimal link configuration through the stationarity of the coarse functional $\Gamma[J]$ (Vacuum marginalisation, Sec.~\ref{si:vac-marg}), inducing a small, static variation $\delta J$ supported near the core. Linearising, the kernel variation is $\delta L = \sum_{ij} (\partial L/\partial J_{ij})\,\delta J_{ij}$ with $\partial L/\partial J_{ij}=E_{ii}+E_{jj}-E_{ij}-E_{ji}$. Volume normalisation and near\,--\,regularity imply (i) $\delta J$ remains local and integrable, (ii) the induced discrete quadratic form variation can be written as a weighted sum of edge\,--\,local gradients. Passing to the continuum via quadratic\,--\,form convergence (Sec.~\ref{si:disc-cont-conv}) maps $\phi^T\delta L\,\phi$ to $\int (\nabla\phi)^T\,\delta\mathbf C(x)\,\nabla\phi\,d^Dx$ with a positive\,--\,semidefinite coefficient field $\delta\mathbf C$. Isotropy to leading order (H1) reduces $\delta\mathbf C$ to a scalar multiple of the identity, $\delta\mathbf C=\delta c_s^2(x)\,\mathbb I$, yielding
\begin{equation}
  \phi^T\delta L\,\phi \;\longrightarrow\; \int \delta c_s^2(x)\,|\nabla\phi|^2\, d^Dx \;=\; -\int \phi\,\nabla\!\cdot\!\big(\delta c_s^2(x)\,\nabla\phi\big)\, d^Dx\,.
\end{equation}
Identifying $\delta K= -\nabla\!\cdot(\delta c_s^2\nabla)$ and using locality of $\delta J$ under a static, bounded perturbation, one obtains $\delta c_s^2(x)=\chi_c\,\rho_m(x)$ with $\chi_c>0$. Positivity follows because increased localized energy (core) prunes/suppresses long links and effectively increases phase stiffness locally (physically, gradients cost more), which increases the quadratic form coefficient. The proportionality (rather than convolution) is a consequence of the gapped amplitude sector (H4) and the near\,--\,regular, short\,--\,range structure of $\delta J$, which eliminates long nonlocal tails at leading order. This establishes the stated divergence form.

\textbf{Remarks.} (R1) Anisotropy at leading order replaces $\chi_c\,\rho_m\,\delta^{ij}$ by a positive\,--\,definite tensor $\chi^{ij}(x)\,\rho_m(x)$; the long\,--\,range conclusions used in the Poisson mapping remain unchanged in the isotropic far field. (R2) If $\delta J$ acquires algebraic long tails, $\delta K$ picks up weakly nonlocal corrections; within the Coulombic window those renormalise $\chi_c$ but preserve the far\,--\,field $1/r$ envelope. (R3) Time dependence introduces retardation kernels and additional kinematic weights beyond the static scope treated here.

\section*{References Prep}
\noindent Reference pointers are summarized in the main paper. Additional SI-specific citations appear inline where relevant.

% (Optional SI-only references could be listed here if they are not in the paper.)

\end{document}


