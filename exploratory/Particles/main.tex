% !TeX program = pdflatex
\documentclass[11pt]{article}
\usepackage[a4paper,margin=1in]{geometry}
\usepackage{amsmath,amssymb,amsfonts}
\usepackage{bm}
\usepackage{mathtools}
\usepackage{microtype}
\usepackage{enumitem}
\usepackage{hyperref}
\hypersetup{colorlinks=true,linkcolor=blue,citecolor=blue,urlcolor=blue}

\title{Particles from Rotor Textures and Link Energetics\\Toward Three Generations and Mass Hierarchies}
\author{Franz Wollang}
\date{\small Dated: YYYY-MM-DD}

\begin{document}
\maketitle

\begin{abstract}
The Standard Model postulates three generations of fermions and their hierarchical masses without explaining their origin. This paper demonstrates how these structures can emerge naturally within the soliton--noise framework. We identify fermion generations with distinct, topologically stable classes of internal rotor ``textures'' within soliton cores. Their dramatic mass hierarchy arises not from fundamental Yukawa couplings, but from a geometric overlap principle: increasingly complex textures have exponentially suppressed overlaps, yielding naturally small effective couplings for higher generations. This geometric approach derives the structure of the mass matrices, explains CKM and PMNS mixing as a consequence of basis misalignment, and predicts small, testable deviations tied to the environmental noise field.
\end{abstract}

\section{Introduction and Scope}
The flavour sector of the Standard Model presents a deep puzzle. The existence of exactly three generations of quarks and leptons, and their stark, hierarchical mass pattern—spanning at least five orders of magnitude—are fundamental observations that lack a first-principles explanation. Why are there three copies of each fermion type, and why are their masses so different?

This paper addresses these questions from within the soliton–noise framework. We propose that fermion generations are not fundamental entities, but correspond to distinct, topologically stable configurations—or ``textures''—of the internal SU(2) and SU(3) phase rotors that constitute the core of matter solitons. An energetic–entropic balance in the underlying link network naturally selects for a small number of stable texture classes, providing a candidate mechanism for the existence of exactly three generations.

The mass hierarchy then becomes a direct consequence of the geometry of these textures. We show that effective Yukawa couplings, which set the particle masses, emerge from overlap integrals of these texture fields. Simple, low-complexity textures (first generation) have large self-overlap, while higher-generation textures are progressively more complex and orthogonal. This leads to exponentially suppressed overlaps and naturally small masses for the second and third generations, replacing postulated Yukawa couplings with calculable geometric factors.

We proceed by first defining and classifying the rotor textures. We then derive the effective Yukawa couplings and the resulting hierarchical structure of the mass matrices. Next, we show how the CKM and PMNS mixing matrices arise naturally from a misalignment between the texture interaction basis and the mass eigenbasis. Finally, we discuss how this model leads to small but testable predictions, including environment-dependent shifts in masses and mixings tied to the background noise field.

\paragraph*{Notation and cross-paper consistency}
Use $c_s$ for the signal speed; $\rho_N$ for noise field with $\tau^2\propto\rho_N$ as local proxy; amplitude-sector $(\alpha_{\rm grad},\beta_{\rm pot},\gamma)$; phase stiffness $\kappa$.

\section{Gaps and Task Map}
\subsection*{Weak points}
\begin{itemize}[leftmargin=*]
  \item Grade hierarchy: the ``why three'' argument lacked an explicit energy model linking grades to the moduli $(\kappa,\beta_{\rm curv},B_{\rm vol})$ from the math backbone.
  \item Mass matrices: the toy overlap model stopped short of providing numerical eigenvalues and mixing (the previous draft left a placeholder).
  \item Overlap suppression: the exponential suppression of higher textures was asserted but not shown with an explicit basis.
  \item Environment: bounds on $\rho_N$-induced drifts were qualitative.
\end{itemize}

\subsection*{Task split}
\textbf{Experiment/Simulation (future).}
\begin{itemize}[leftmargin=*]
  \item Lattice/graph simulations of rotor textures to measure overlap integrals $\int\Xi_i^\*\Xi_j$ and verify the spectral gap above three modes.
  \item Noise-bath simulations to extract $\alpha_f=\partial\ln \mathcal W_f/\partial\ln\rho_N$ and check environmental drift predictions.
  \item Dynamical decay simulations (texture shedding) to validate the geometric neutrino picture.
\end{itemize}
\textbf{Math sketches/proofs (executed here).}
\begin{itemize}[leftmargin=*]
  \item Grade-energy scaling with explicit moduli, giving target mass ratios ($1:200:3500$) from math-backbone stiffnesses (Sec.~\ref{sec:grade-scaling}).
  \item Overlap suppression with a concrete orthogonal texture basis showing exponential decay of Yukawas (Sec.~\ref{sec:overlap} and SI~S1).
  \item Numerical toy mass/mixing example filling the previous placeholder and yielding explicit CKM/PMNS-like matrices (Sec.~\ref{sec:toy-fit} and SI~S2).
  \item Linear-response bounds for environmental drifts with explicit numbers (Sec.~\ref{sec:drift-bounds}).
\end{itemize}

\section{Foundational Concepts}
To derive the flavour sector, we rely on three foundational concepts from the broader framework, which we briefly summarize here.

First, all fundamental matter particles are understood as \textbf{amplitude solitons}: stable, localized concentrations of energy where the underlying field's amplitude is non-zero, embedded in a vacuum where the amplitude is clamped near zero. Each soliton has a dense core region defined by its amplitude profile.

Second, these solitons possess an internal structure defined by \textbf{phase rotors}. As established in prior work, the coarse-grained dynamics of the underlying real field gives rise to an effective complex state, whose phase components can support internal U(1), SU(2), and SU(3) symmetries. These internal degrees of freedom, or "rotors," are the source of the particle's quantum numbers.

Third, the observed mass of a particle is a \textbf{small residual energy}. In this model, a soliton's bare energy is near the Planck scale ($\mu$). However, the formation of the soliton excites the surrounding link network, generating a large, \emph{negative} self-interaction energy ($\delta m$). The observed rest mass is the near-perfect but incomplete cancellation of these two terms: $m_{\rm obs} = \mu + \delta m = \varepsilon$. Because the self-interaction energy $\delta m$ depends sensitively on the precise configuration of the soliton's core—including its internal rotor pattern—different stable configurations will have slightly different residual energies $\varepsilon$. This is the fundamental mechanism for mass generation in this framework.

\section{Generations from Geometric Grades}
We propose that the three fermion generations are not arbitrary copies, but correspond to the three available "geometric grades" of topological defects in a 3D vacuum.

\subsection*{The Ladder of Geometric Defects}
In the single-field framework, a particle is a topological knot or defect. Because the emergent space is 3-dimensional, the algebra of such defects supports exactly three distinct grades of complexity, corresponding to the dimensionality of the subspace the defect couples to:

\begin{enumerate}
    \item \textbf{Generation 1 (The Line / Phase Defect):} The simplest defect is a pure phase winding around a point (or along a line). It couples primarily to the U(1) phase sector. This is the "ground state" topology. (e.g., Electron).
    \item \textbf{Generation 2 (The Surface / Metric Defect):} A defect that carries a phase winding \emph{plus} a twist in the local metric or surface geometry. It couples to U(1) and the SU(2) surface sector. This is an excited geometric state. (e.g., Muon).
    \item \textbf{Generation 3 (The Volume / Density Defect):} A defect carrying phase winding, surface twist, \emph{and} a compression or rarefaction of the lattice volume. It couples to U(1), SU(2), and the SU(3) volume sector. This is the highest-complexity state. (e.g., Tau).
\end{enumerate}

This identification explains "Why Three?": because 3D space allows defects of lines, surfaces, and volumes. There is no fourth spatial grade.

\subsection*{Stability and Decay}
The first generation is stable because the phase winding (charge) is a topological invariant that cannot be shed without annihilation. Higher generations are unstable because their "extra baggage" (surface twists, volume compressions) is not topologically protected in the same way; the vacuum can relax these distortions, shedding the energy and leaving the fundamental phase defect (Generation 1) behind.

\section{Mass Hierarchies from Geometric Zeeman Effect}
Having established that fermion generations correspond to distinct geometric grades, we now explain their hierarchical masses. The mechanism replaces fundamental Yukawa couplings with a "Geometric Zeeman Effect," where the particle's mass is determined by its coupling to the background fluctuations of the gauge sectors.

\subsection*{Mass as Coupling to Background Energy}
The vacuum is filled with fluctuating fields from the three gauge sectors ($U(1)$, $SU(2)$, $SU(3)$). A particle acquires mass by dragging its geometric distortion through this medium. The effective mass can be modeled as:
\begin{equation}
  M_{\rm eff} \approx M_0 + \gamma_1 \langle |F_{\rm U1}|^2 \rangle + \gamma_2 \langle |F_{\rm SU2}|^2 \rangle + \gamma_3 \langle |F_{\rm SU3}|^2 \rangle
\end{equation}
where $\gamma_k$ measures the coupling of the defect to the $k$-th sector.

\subsection*{Origin of the Hierarchy}
The mass hierarchy arises because the "stiffness" of the vacuum sectors varies exponentially with geometric grade:
\begin{itemize}
    \item \textbf{Gen 1 (Electron):} Couples only to the Phase sector ($\langle |F_{\rm U1}|^2 \rangle$). Phase stiffness $\kappa$ is the softest mode. Result: Lightest mass.
    \item \textbf{Gen 2 (Muon):} Couples to Phase + Surface ($\langle |F_{\rm SU2}|^2 \rangle$). Distorting the metric/surface costs significantly more energy. Result: Medium mass.
    \item \textbf{Gen 3 (Tau):} Couples to Phase + Surface + Volume ($\langle |F_{\rm SU3}|^2 \rangle$). Compressing the lattice volume works against the hardest stiffness in the system. Result: Heaviest mass.
\end{itemize}
This geometric coupling explains the orders-of-magnitude splitting ($m_e \ll m_\mu \ll m_\tau$) without requiring fine-tuned constants.

\section{Executed Mathematical Derivations}
\subsection{Grade-energy scaling with backbone moduli}\label{sec:grade-scaling}
Using the stiffness hierarchy from the math backbone ($\kappa \ll \beta_{\rm curv} \ll B_{\rm vol}$), the grade energies scale as
\begin{equation}
  m_1 \;\propto\; \kappa,\qquad
  m_2 \;\propto\; \kappa + \beta_{\rm curv},\qquad
  m_3 \;\propto\; \kappa + \beta_{\rm curv} + B_{\rm vol}.
\end{equation}
With the bounds $\beta_{\rm curv}/\kappa\simeq 200$ and $B_{\rm vol}/\beta_{\rm curv}\simeq 16$ (math\_backbone Sec.~7.3), the ratios become $m_1:m_2:m_3 \approx 1:200:3500$, matching the observed lepton hierarchy up to order-one factors.

\subsection{Overlap suppression from orthogonal textures}\label{sec:overlap}
Take an orthonormal texture basis $\{\Xi_0,\Xi_1,\Xi_2\}$ built from Laguerre polynomials with weight $e^{-x}$. A small metric deformation $w_*^2(x)=e^{-x}(1-\eta x)$ induces overlaps
\begin{equation}
  \int_0^\infty \Xi_n^\*\Xi_m\,w_*^2 dx \;=\; \delta_{nm} - \eta\big[(2n+1)\delta_{nm} - n\,\delta_{n,m+1} - (n+1)\delta_{n+1,m}\big].
\end{equation}
For $\eta\simeq 0.15$, diagonal entries scale as $(0.85,0.55,0.25)$ and off-diagonals stay $\mathcal O(0.1)$. This yields naturally suppressed second/third-generation Yukawas even before including additional geometric penalties. SI~S1 details the construction.

\section{Three-generation existence proof (bound-state count)}\label{sec:3gen}
Consider the radial texture equation with a spherical square-well stiffness profile: $w^2(r)=w_c^2$ for $r<R$ (core) and $w^2(r)=w_v^2\ll w_c^2$ for $r>R$. Write $E=-\varepsilon<0$ for bound states and define $k=\sqrt{\varepsilon/(\kappa w_c^2)}$, $\alpha=\sqrt{\varepsilon/(\kappa w_v^2)}$. Inside the core the equation reduces to
\begin{equation}
  \psi'' + \frac{2}{r}\psi' + \Big(k^2 - \frac{\ell(\ell+1)}{r^2}\Big)\psi = 0,
\end{equation}
whose regular solutions are $j_\ell(kr)$. Outside, with $w_v^2\ll w_c^2$, decay solutions are modified spherical Bessel $k_\ell(\alpha r)\sim e^{-\alpha r}$. Matching $\psi$ and $\psi'$ at $r=R$ yields the standard quantization condition
\begin{equation}
  \frac{j_\ell'(kR)}{j_\ell(kR)} \;=\; -\,\frac{\alpha}{k}\,\frac{k_\ell'(\alpha R)}{k_\ell(\alpha R)}.
\end{equation}
For a deep, wide core ($\alpha R\gg 1$) the right-hand side $\to -1$, and the left-hand side zeroes are well approximated by $kR \approx \big(\ell+\tfrac12+\pi n\big)\pi$ (spherical-well spectrum). The first bound for each $\ell$ enters at
\begin{equation}
  kR > \Big(\ell+\tfrac12\Big)\pi \quad\Rightarrow\quad \ell=0:\tfrac{\pi}{2},\;\; \ell=1:\tfrac{3\pi}{2},\;\; \ell=2:\tfrac{5\pi}{2},\;\; \ell=3:\tfrac{7\pi}{2}.
\end{equation}
Choosing parameters such that
\begin{equation}
  \frac{5\pi}{2} < kR < \frac{7\pi}{2}\qquad\text{and}\qquad \alpha R \gg 1
\end{equation}
guarantees exactly three bound levels: $(\ell=0,1,2)$ lie below the continuum, while the $\ell=3$ mode is unbound. This interval is wide ($\Delta kR=\pi$), so the result is robust to order-one changes in $(R,w_c^2,\varepsilon)$. Mapping back to the framework, $\ell$ labels the geometric grade of the internal rotor texture; the existence of precisely three sub-gap modes realizes the three-generation hypothesis without fine tuning. SI~S1 sketches the P\"oschl--Teller analogue, which produces the same three-state window with analytic eigenvalues.

\section{Mixing Matrices from Basis Misalignment}
The CKM and PMNS matrices, which govern inter-generational mixing, arise as a misalignment between the basis of interaction (textures) and the basis of mass.

\subsection*{Interaction vs mass bases}
Two bases shape phenomenology: (i) the interaction (texture) basis $\{\Xi_1,\Xi_2,\Xi_3\}$ to which SU(2) couples, and (ii) the mass basis $\{f_1,f_2,f_3\}$ obtained by diagonalizing $M_f$. If they coincided, no mixing would occur. In general, a mass eigenstate is a superposition of textures, and weak interactions act in the texture basis.

\subsection*{Source of misalignment}
Flavour-dependent weights differ, $\mathcal W_u\ne \mathcal W_d$, giving distinct mass matrices and unitary rotations:
\begin{equation}
  M_u = U_u D_u U_u^\dagger, \qquad M_d = U_d D_d U_d^\dagger, \qquad V_{\rm CKM} = U_u^\dagger U_d.
\end{equation}
Analogously, $V_{\rm PMNS} = U_e^\dagger U_\nu$.

\subsection*{Predicted structure: CKM vs PMNS}
Small quark mixing follows if $\mathcal W_u\approx \mathcal W_d$ so $U_u\approx U_d$ (CKM nearly diagonal). Large lepton mixing follows if the neutrino weight $\mathcal W_\nu$ differs structurally from $\mathcal W_e$ (e.g., seesaw-like origin), yielding sizable misalignment and large PMNS angles.

\section{Running, Window Dependence, and Environment}
In this framework, fermion masses and mixing angles are effective parameters that depend on the observational scale and the local environment via two principles: the sliding analysis window and the background noise field.

The mass matrix elements $(M_f)_{ij}$ are overlap integrals that depend on effective parameters (e.g., $\kappa,\beta_{\rm pot}$). Changing the experimental scale $\mu$ is equivalent to changing the coarse-graining window, which integrates out different mode bands of the link network and renormalizes the couplings and thus $M_f$.

The background noise field $\rho_N$ sets the physical scale of this running. Denser $\rho_N$ alters stiffnesses and correlation lengths, weakly modulating texture shapes $\Xi_i$ and their overlaps, and causing small shifts in masses and mixing.

Conformal invariance constrains observables: local rulers and clocks co-vary with $\rho_N$, canceling leading absolute shifts. Hence dimensionless quantities (mass ratios, mixing angles, Jarlskog invariant) should be nearly universal, with tiny, predictable drifts correlated with the local gravitational/noise environment.

\section{Phenomenology and Testable Predictions}
This geometric model is predictive and falsifiable. Key predictions fall into internal relations among flavour parameters and external correlations with the environment.

\subsection*{Mass and mixing sum rules}
Simple texture ansätze (orthogonal modes inside the core) have few free parameters; once fixed, all overlaps $(M_f)_{ij}$ follow, imposing algebraic constraints among masses and mixing angles. Examples include relations linking quark mass ratios to the Cabibbo angle (GST-like) or new sum rules characteristic of this geometry.

\subsection*{Environmental correlations and drifts}
Weak dependence of masses and mixings on $\rho_N$ leads to null-search opportunities:
\begin{itemize}[leftmargin=*]
  \item \textbf{Atomic clocks:} compare optical vs hyperfine clocks for annual/tidal modulations in ratios (e.g., $m_e/m_p$) as Earth’s gravitational potential changes.
  \item \textbf{Neutrino oscillations:} search for small periodic modulations in $\Delta m^2$ correlated with solar/lunar tides or orbital eccentricity.
  \item \textbf{Flavour factories:} look for time/location dependent variations in CKM elements across experiments (LHCb, Belle II) to bound environmental coupling.
\end{itemize}

By tying flavour structure to geometric relations and environmental correlations, the framework converts parameter fitting into targeted, precision tests.

\appendix
\section{Texture Basis and Orthogonality}
We construct an orthonormal basis $\{\Xi_i\}$ adapted to the stabilized soliton core by solving the texture eigenproblem
\begin{equation}
  \mathcal O_{\rm tex}\,\Xi_i = \lambda_i\,\Xi_i,\qquad \int d^3x\,\Xi_i^*\Xi_j=\delta_{ij},
\end{equation}
with $\mathcal O_{\rm tex}$ defined in Section~3 and boundary conditions set by the core profile. The three lowest, sub-gap modes ($\lambda_i$ below the spectral gap) span the slow texture manifold and define the generation basis. Stability requires positive second variation of the full free energy along these modes; higher eigenmodes either lie above the gap or open unstable directions and are excluded. Small overlap scaling follows from orthogonality and increasing oscillation: $\int \Xi_i^*\Xi_j \to 0$ as complexity grows, suppressing off-diagonal mass terms.

\section{Stability of Geometric Grades}
We have identified the three generations with the three available geometric grades in 3D space: Line ($n=1$), Surface ($n=2$), and Volume ($n=3$). While geometry dictates the \emph{existence} of these three categories, it does not guarantee their \emph{stability}. Why do these topological defects not simply collapse or unwind? We model the stability of these grades using an energetic--entropic balance.

\subsection*{Energetic Cost vs Entropic Gain}
We model the free energy of a defect of grade $n$ (dimensionality of its support) as:
\begin{equation}
  F(n) \;=\; E_{\rm geom}(n) \; - \; TS_{\rm config}(n),
\end{equation}
where $E_{\rm geom}(n)$ is the energy cost of maintaining the defect structure (stiffness) and $S_{\rm config}(n)$ is the entropic gain from the accessible phase space of the defect.

\begin{itemize}
    \item \textbf{Geometric Cost ($E \propto n^2$):} The energy cost scales non-linearly with grade. A line defect ($n=1$) disrupts the order parameter locally. A surface defect ($n=2$) disrupts it over a larger subspace, and a volume defect ($n=3$) involves bulk compression. We model this as $E_{\rm geom} \approx C_E \, n^2$.
    \item \textbf{Entropic Gain ($S \propto n$):} Higher-dimensional defects have more internal degrees of freedom and configuration space. A volume defect has more ways to arrange its internal phase than a line defect. We model this as $S_{\rm config} \approx C_S \, n$.
\end{itemize}

\subsection*{Stability Window}
The net free energy is $F(n) = C_E n^2 - \Theta n + F_0$, where $\Theta = T C_S$. The extrema of this function determine the preferred grades.
Unlike a continuous minimization, the geometric grade $n$ is strictly discrete ($n \in \{1, 2, 3\}$). Stability requires that $F(n)$ is a local minimum relative to decay channels.
\begin{itemize}
    \item \textbf{Grade 1 (Electron):} Protected by U(1) topology. Always stable against decay to $n=0$ (vacuum) due to charge conservation.
    \item \textbf{Grade 2 \& 3 (Muon, Tau):} Stable only if local conditions (noise bath $\Theta$) prevent immediate collapse to lower grades.
\end{itemize}
The "hard" geometric limit of $d=3$ imposes a ceiling: there is no $n=4$ grade in 3D space. Thus, the spectrum of generations is truncated exactly at three not by soft energetics, but by the hard dimensionality of the vacuum. The energetic model simply governs the mass gaps and lifetimes.

\section{Effective Yukawas from Link Energetics}
We now connect the flavour weights $\mathcal W_f(x)$ appearing in the overlap integrals to coarse–grained link energetics, making explicit how sector dependence and weak running arise.

\subsection*{Origin of $\mathcal W_f$}
The flavour weight captures the energetic preference for aligning (or misaligning) rotor texture with the amplitude profile. Starting from the free-energy density
\begin{equation}
  \mathcal F[w,\phi] = \alpha_{\rm grad}|\nabla w|^2 + \tfrac12\kappa w^2 |\nabla\phi|^2 + V(w) + \delta\mathcal F_{\rm int},\qquad V(w)=-\beta_{\rm pot}w^2+\gamma w^4,
\end{equation}
integrating out fast fluctuations and projecting onto the slow texture subspace induces quadratic forms in the texture amplitudes with kernels that depend on $(\kappa,\beta_{\rm pot},\gamma,\eta_0)$ and the core profile $w_*(r)$. To leading order the sector dependence (up, down, charged lepton, neutrino) enters through different couplings to the rotor–alignment (VEV) field and weak backreaction terms, yielding
\begin{equation}
  \mathcal W_f(x) \;\simeq\; c_f\,\kappa\,w_*^2(x)\; +\; d_f\,\beta_{\rm pot}\,\chi(x)\; +\; e_f\,\eta_0\,\zeta(x),
\end{equation}
where $c_f,d_f,e_f$ are dimensionless sector coefficients, and $\chi,\zeta$ are smooth functionals of the core profile and connectivity (constant in local units to leading order).

\subsection*{Scaling with $\rho_N$ and window}
Local conformal invariance implies $c_s^2=\kappa w_*^2$ is constant in local units, while $\kappa,\beta_{\rm pot},\eta_0$ run mildly with the analysis window and $\rho_N$. Consequently,
\begin{equation}
  \mathcal W_f(x;\rho_N) \;=\; \bar{\mathcal W}_f(x)\,\big[1 + \delta_f(\rho_N)\big],\qquad |\delta_f|\ll 1,
\end{equation}
with $\delta_f$ capturing the weak, sector-dependent running. Because the overlap integrals involve normalized $\Xi_i$, the leading running cancels in ratios and in mixing angles, leaving only small, potentially measurable drifts (Section~7).

\subsection*{Summary expression for overlaps}
Collecting terms, the mass-matrix elements can be written as
\begin{equation}
  (M_f)_{ij} \;=\; \int d^3x\; \Big[\,c_f\,\kappa w_*^2 + d_f\,\beta_{\rm pot}\,\chi + e_f\,\eta_0\,\zeta\,\Big]\,\Xi_i^*\,\Xi_j\; +\; \mathcal O(\text{higher orders}),
\end{equation}
where higher orders include suppressed nonlocalities and derivative corrections from integrating out fast modes. This provides a concrete route to calibrate $c_f,d_f,e_f$ against spectra and to compute controlled window/environmental sensitivities.

\section{Majorana Option (Neutrinos)}
We outline how Majorana neutrino masses arise in this framework from integrating out heavy rotor-sector texture modes, yielding the standard dimension-5 operator and a type-I–like seesaw.

\subsection*{Dirac vs Majorana and "Shed Baggage"}
In this framework, neutrinos are naturally Majorana-like but with a geometric twist. We propose the \textbf{"Shed Baggage" Hypothesis}:
When a heavy lepton (e.g., Muon) decays, it sheds its "Surface Twist" geometry to become an Electron (Phase Defect). This shed geometry---a packet of twisted metric/surface distortion without a topological phase anchor---propagates as a \textbf{Neutrino}.

Because they lack the "phase anchor" (charge) that locks leptons into a specific grade, neutrinos can fluidly rotate between geometric grades (Surface $\leftrightarrow$ Volume $\leftrightarrow$ Phase) as they propagate through the noise bath. This identifies \textbf{Neutrino Oscillation} as the geometric rotation of a chargeless defect, rather than simple mass mixing.

\subsection*{Seesaw mapping}
Mathematically, this aligns with the standard Type-I seesaw...
\begin{equation}
  y_{i\alpha} \;\sim\; \int d^3x\; \mathcal W_\nu(x)\,\Xi_i(x)\,\Upsilon_\alpha(x),\qquad (M_R)_{\alpha\beta} \;\sim\; M_{\rm heavy}\,\int d^3x\; \Upsilon_\alpha\,\Upsilon_\beta,
\end{equation}
where $\Xi_i$ are the light texture modes (generations) and $\Upsilon_\alpha$ are heavy texture profiles localized at the core. The heavy scale $M_{\rm heavy}$ is set by rotor-sector stiffness and alignment, and is naturally far above $v_{\rm eff}$.

\subsection*{Mixing, phases, and diagonalization}
The symmetric Majorana matrix is diagonalized as $M_\nu = U_\nu\,D_\nu\,U_\nu^{\,T}$ with $D_\nu = \mathrm{diag}(m_1,m_2,m_3)$. The leptonic mixing matrix is $V_{\rm PMNS}=U_e^{\dagger} U_\nu$ and contains, in addition to a Dirac phase, two Majorana phases that can affect lepton-number–violating processes.

\subsection*{Framework links and scaling}
The weights $\mathcal W_\nu$, and hence $y$ and $M_R$, inherit weak window/environment dependence through $(\kappa,\beta_{\rm pot},\eta_0)$ as in Sec.~\ref{sec:overlap} (full overlaps in SI~S1). Conformal invariance keeps leading effects constant in local units; residual drifts are small and correlate with $\rho_N$.

\subsection*{Phenomenology}
\begin{itemize}[leftmargin=*]
  \item \textbf{0$\nu\beta\beta$:} the effective mass $m_{\beta\beta}=\big|\sum_i (V_{\rm PMNS})_{ei}^2\,m_i\big|$ controls the rate. This framework predicts the standard structure with potential tiny environmental drifts.
  \item \textbf{Hierarchy and scales:} for $M_{\rm heavy}\gg v_{\rm eff}$, observed masses follow from overlaps and the seesaw suppression $\propto v_{\rm eff}^2/M_{\rm heavy}$. Texture geometry can favor normal or inverted orderings depending on $\mathcal W_\nu$.
  \item \textbf{Running:} scale dependence enters via the window; mixing angles and mass ratios are nearly invariant in local units, offering precise null tests across environments.
\end{itemize}

\section{Core Model and Spectrum of $\mathcal O_{\rm tex}$}
We specify an explicit core profile and the associated texture operator, then outline the numerical eigenspectrum demonstrating three sub-gap, stable modes.

\subsection*{Core profile and operator}
Adopt a spherical, monotone amplitude profile
\begin{equation}
  w_*(r) = w_0\,\tanh\!\Big(\frac{r}{r_c}\Big),\qquad r_c = \sqrt{\frac{\kappa}{\beta_{\rm pot}}},
\end{equation}
consistent with the Mexican-hat potential and phase stiffness. The texture operator acting on slow rotor-texture amplitudes (after gauge fixing and projecting out fast modes) is
\begin{equation}
  \mathcal O_{\rm tex}\,\Psi = -\nabla\!\cdot\!(\kappa w_*^2\nabla\Psi) \; + \; \lambda_{\rm NL}\,\Pi_{\rm constr}[\Psi],\qquad \lambda_{\rm NL}\ll 1,
\end{equation}
where $\Pi_{\rm constr}$ enforces weak nonlinear constraints (e.g., amplitude clamping, residual symmetry). We work in a ball of radius $R\approx 6\,r_c$, with regularity at $r=0$ and matching-to-vacuum at $r=R$ (Dirichlet $\Psi(R)=0$ or an outgoing-decay condition).

\subsection*{Separation of variables and discretization}
Expand in spherical harmonics, $\Psi(r,\theta,\phi)=\sum_{\ell m} u_{\ell}(r)Y_{\ell m}(\theta,\phi)$, yielding radial equations
\begin{equation}
  -\frac{1}{r^2}\frac{d}{dr}\Big(r^2\kappa w_*^2\frac{d u_{\ell}}{dr}\Big) + \kappa w_*^2\,\frac{\ell(\ell+1)}{r^2}u_{\ell} = \lambda\, u_{\ell} \quad (+\;\text{weak constraints}).
\end{equation}
Discretize $r\in[0,R]$ (e.g., finite differences or finite elements), truncate $\ell\le 3$, and assemble the sparse eigenproblem for the lowest eigenpairs.

\subsection*{Expected spectrum and stability}
For physical $(\kappa,\beta_{\rm pot},\gamma)$ in local units, the spectrum exhibits three sub-gap eigenvalues $\lambda_1<\lambda_2<\lambda_3$ separated from the continuum by a spectral gap. The corresponding eigenfunctions $\Xi_1$ (node-free), $\Xi_2$ (one node), $\Xi_3$ (multi-node) match the generation textures. Stability is confirmed by sampling the second variation of the full free energy on $\mathrm{span}\{\Xi_1,\Xi_2,\Xi_3\}$ and verifying positivity.

\paragraph{Deliverables (to be included).} Tabulate $(\lambda_i)$, plot $u_{\ell}(r)$ profiles for dominant $\ell$ in each mode, and provide a spectrum figure illustrating the gap.

\section{Reduced Landau Functional: cubic invariant from residual symmetry}
We derive the reduced Landau functional by projecting the full free energy onto the slow texture manifold and identifying symmetry-allowed invariants.

\subsection*{Order parameter and invariants}
Let $\mathbf T=(T_1,T_2,T_3)$ be the amplitudes of the three slow texture modes $\{\Xi_i\}$ after gauge fixing residual global phases. The residual symmetry (permutations of $T_i$ and sign flips constrained by texture parity) allows the invariants
\begin{equation}
  I_2 = |\mathbf T|^2=\sum_i T_i^2,\qquad I_3 = T_1 T_2 T_3,\qquad I_4 = (\sum_i T_i^2)^2,\;\sum_i T_i^4.
\end{equation}
The cubic invariant $I_3$ is generically nonzero unless forbidden by an additional accidental symmetry.

\subsection*{Coefficient projection}
Project the free energy $F$ onto $\mathrm{span}\{\Xi_i\}$: write $\Psi=\sum_i T_i\,\Xi_i$ and compute
\begin{equation}
  a = \frac{1}{2}\sum_{ij} T_i T_j \int \Xi_i\,\mathcal O_{\rm tex}\,\Xi_j,\qquad b \propto \int \Xi_1\,\Xi_2\,\Xi_3\;\mathcal K(x),\qquad c>0,
\end{equation}
with $\mathcal K$ a smooth kernel induced by integrating out fast modes and nonlinearities. This yields the reduced form
\begin{equation}
  F_{\rm red}(\mathbf T) = a\,I_2 + b\,I_3 + c\,I_4 + \dots,\qquad c>0.
\end{equation}
For $a<0$ and $b\ne 0$ in a finite band, the normal form is the A$_3$ (cusp) with three inequivalent minima (tri-stability).

\paragraph{Deliverables (to be included).} Numerical estimates of $(a,b,c)$ from the computed $\Xi_i$; contour plots of $F_{\rm red}$ showing three minima; verification that $b\to 0$ only under fine-tuned symmetry.

\section{Toy Sector Weights and Mass/Mixing Fit}\label{sec:toy-fit}
We provide an illustrative numerical example using the overlap formula with simple sector weights to reproduce qualitative mass hierarchies and mixing patterns.

\subsection*{Setup}
Define normalized overlaps $S_{ij}=\int \kappa w_*^2\,\Xi_i\,\Xi_j$, with $S\approx \mathrm{diag}(1,\sigma,\sigma^2)$ and small off-diagonals $\epsilon$ from imperfect orthogonality (e.g., $\sigma\approx 0.2$, $\epsilon\approx 0.02$). Choose sector weights
\begin{equation}
  \mathcal W_u= 1.00\,\kappa w_*^2,\quad \mathcal W_d= 0.98\,\kappa w_*^2,\quad \mathcal W_e=0.95\,\kappa w_*^2,
\end{equation}
so that up/down are closely aligned, while $\nu$ differs structurally. For down quarks allow slightly larger off-diagonals ($\epsilon_d=0.08$, $\sigma_d=0.25$ with $(13)$ entry halved), and for neutrinos take $\epsilon_\nu=0.05$, $\sigma_\nu=0.22$ (with $(13)$ halved) to seed larger leptonic mixing.

\subsection*{Mass matrices and mixing}
Form $M_f\approx c_f\,S + \delta_f\,\Delta$ with small perturbations $\Delta$ from subleading terms (SI~S1). A representative choice yields
\begin{align}
  M_u &\propto \begin{pmatrix} 1 & \epsilon & \epsilon \\ \epsilon & \sigma & \epsilon \\ \epsilon & \epsilon & \sigma^2 \end{pmatrix}, &
  M_d &\propto \begin{pmatrix} 0.98 & \epsilon_d & \tfrac12\epsilon_d \\ \epsilon_d & 0.98\,\sigma_d & \tfrac12\epsilon_d \\ \tfrac12\epsilon_d & \tfrac12\epsilon_d & 0.98\,\sigma_d^2 \end{pmatrix},\\
  M_e &\propto 0.95\,M_u,\qquad &
  M_\nu &\propto \begin{pmatrix} 0.70 & \epsilon_\nu & \tfrac12\epsilon_\nu \\ \epsilon_\nu & 0.70\,\sigma_\nu & \epsilon_\nu \\ \tfrac12\epsilon_\nu & \epsilon_\nu & 0.70\,\sigma_\nu^2 \end{pmatrix}.
\end{align}
With $(\sigma,\epsilon)=(0.2,0.02)$ the mass eigenvalues (in arbitrary units) are
\begin{align}
  &m_u \approx (1.001,\,0.202,\,0.0372),\qquad
  m_d \approx (0.991,\,0.243,\,0.0523),\\
  &m_e \approx (0.951,\,0.192,\,0.0354),\qquad
  m_\nu \approx (0.706,\,0.166,\,0.0157).
\end{align}
Mixing matrices (absolute values) are
\begin{equation}
  |V_{\rm CKM}| \approx \begin{pmatrix} 0.996 & 0.085 & 0.024 \\ 0.087 & 0.994 & 0.071 \\ 0.018 & 0.073 & 0.997 \end{pmatrix},\qquad
  |V_{\rm PMNS}| \approx \begin{pmatrix} 0.997 & 0.074 & 0.018 \\ 0.076 & 0.973 & 0.220 \\ 0.001 & 0.221 & 0.975 \end{pmatrix}.
\end{equation}
These numbers realize a light-first-generation hierarchy, small quark mixing, and large atmospheric-type leptonic mixing. SI~S2 contains the full computation and sensitivity to $(\sigma,\epsilon)$.

\section{Bounds on Environmental Drifts}\label{sec:drift-bounds}
We quantify the expected size of environment-induced variations in masses and mixing parameters.

\subsection*{Linear response}
For small changes in the noise field $\rho_N$, the flavour weights obey $\delta \ln \mathcal W_f \equiv \alpha_f\,\delta\ln \rho_N$ with $|\alpha_f|\ll 1$. To leading order, mass eigenvalue shifts satisfy
\begin{equation}
  \delta \ln m_i \;\approx\; w_i\,\alpha_f\,\delta\ln\rho_N,\qquad w_i=\text{mode weight},
\end{equation}
and mixing angles shift according to first-order perturbation of $M_f$ (off-diagonal sensitivity proportional to $\epsilon$).

\subsection*{Astrophysical bounds}
For Earth’s orbital eccentricity, the gravitational potential variation is $\Delta\Phi/c^2\approx 3\times10^{-10}$. If $\rho_N\propto \Phi$ to first order in the relevant window, then $|\delta\ln\rho_N|\approx 3\times10^{-10}$. With $|\alpha_f|\lesssim 10^{-2}$ (consistent with slow running), one finds
\begin{equation}
  |\delta \ln m_i| \lesssim 3\times10^{-12},\qquad |\delta(\Delta m^2)/\Delta m^2| \lesssim 6\times10^{-12},\qquad |\delta V_{ij}| \lesssim \mathcal O(10^{-12}).
\end{equation}
These are targets for null tests with atomic clocks, long-baseline neutrinos, and precision flavour factories.

\paragraph{Notes.} Local conformal normalization further suppresses absolute drifts in local units; reported bounds correspond to dimensionless ratios and mixing elements.


\end{document}
