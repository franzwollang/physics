% !TeX program = pdflatex
\documentclass[11pt]{article}
\usepackage[a4paper,margin=1in]{geometry}
\usepackage{amsmath,amssymb,amsfonts}
\usepackage{bm}
\usepackage{mathtools}
\usepackage{microtype}
\usepackage{enumitem}
\usepackage{hyperref}
\hypersetup{colorlinks=true,linkcolor=blue,citecolor=blue,urlcolor=blue}

\title{Particles from Rotor Textures and Link Energetics\\Toward Three Generations and Mass Hierarchies}
\author{Franz Wollang}
\date{\small Dated: YYYY-MM-DD}

\begin{document}
\maketitle

\begin{abstract}
The Standard Model postulates three generations of fermions and their hierarchical masses without explaining their origin. This paper demonstrates how these structures can emerge naturally within the soliton--noise framework. We identify fermion generations with distinct geometric grades of topological defects in a 3D vacuum (Line, Surface, Volume), and with corresponding classes of stable internal rotor textures within matter solitons. The dramatic mass hierarchy arises not from fundamental Yukawa couplings but from a \emph{moduli hierarchy} in the vacuum free energy: different grades predominantly stress the phase stiffness, amplitude-gradient, and amplitude-potential sectors, which carry parametrically different costs. Within and between grades, texture overlap integrals then determine the detailed structure of effective mass matrices, while CKM and PMNS mixing arise from basis misalignment. The framework predicts small, testable environmental drifts tied to the background noise field.
\end{abstract}

\section{Introduction and Scope}
The flavour sector of the Standard Model presents a deep puzzle. The existence of exactly three generations of quarks and leptons, and their stark, hierarchical mass pattern—spanning at least five orders of magnitude—are fundamental observations that lack a first-principles explanation. Why are there three copies of each fermion type, and why are their masses so different?

This paper addresses these questions from within the soliton–noise framework. We propose that fermion generations are not fundamental entities, but correspond to distinct, topologically stable configurations—or ``textures''—of the internal SU(2) and SU(3) phase rotors that constitute the core of matter solitons. An energetic–entropic balance in the underlying link network naturally selects for a small number of stable texture classes, providing a candidate mechanism for the existence of exactly three generations.

The mass hierarchy then becomes a direct consequence of how different defect grades load different stiffness sectors (``vacuum moduli'') of the coarse free energy. This sets the \emph{dominant} scale separation between generations. Texture overlap integrals (within the stabilized soliton core) supply the next layer: the detailed structure of effective mass matrices and the pattern of mixings once the grade-dependent stiffness scales are fixed.

We proceed by first defining and classifying the rotor textures. We then derive the effective Yukawa couplings and the resulting hierarchical structure of the mass matrices. Next, we show how the CKM and PMNS mixing matrices arise naturally from a misalignment between the texture interaction basis and the mass eigenbasis. Finally, we discuss how this model leads to small but testable predictions, including environment-dependent shifts in masses and mixings tied to the background noise field.

\paragraph*{Notation and cross-paper consistency}
Use $c_s$ for the signal speed; $\rho_N$ for noise field with $\tau^2\propto\rho_N$ as local proxy; amplitude-sector $(\alpha_{\rm grad},\beta_{\rm pot},\gamma)$; phase stiffness $\kappa$.

\section{Foundational Concepts}
To derive the flavour sector, we rely on three foundational concepts from the broader framework, which we briefly summarize here.

First, all fundamental matter particles are understood as \textbf{amplitude solitons}: stable, localized departures of the amplitude field from its homogeneous vacuum value $w\approx w_*$. Each soliton has a core region where $w$ is displaced and gradients/curvatures concentrate, embedded in a vacuum where $w$ is clamped near $w_*$.

Second, these solitons possess an internal structure defined by \textbf{phase rotors}. As established in prior work, the coarse-grained dynamics of the underlying real field gives rise to an effective complex state, whose phase components can support internal U(1), SU(2), and SU(3) symmetries. These internal degrees of freedom, or "rotors," are the source of the particle's quantum numbers.

Third, the observed mass of a particle is a \textbf{small residual energy}. In this model, a soliton's bare energy is near the Planck scale ($\mu$). However, the formation of the soliton excites the surrounding link network, generating a large, \emph{negative} self-interaction energy ($\delta m$). The observed rest mass is the near-perfect but incomplete cancellation of these two terms: $m_{\rm obs} = \mu + \delta m = \varepsilon$. Because the self-interaction energy $\delta m$ depends sensitively on the precise configuration of the soliton's core—including its internal rotor pattern—different stable configurations will have slightly different residual energies $\varepsilon$. This is the fundamental mechanism for mass generation in this framework.

\section{The Topological Imperative: Spontaneous Axis Formation}
A fundamental question for any soliton model is why a spherically symmetric "lump" of energy would acquire a preferred axis (spin) or distinct poles (charge). We find that this asymmetry is not an imposed property but a topological necessity of the soliton's interactions with its environment.

\subsection{The Shell Theorem Constraint}
While the soliton's core may be isotropic, it interacts with the fluctuating background noise field ($\rho_N$) primarily through its boundary layer or "shell"---the interface where the stiff amplitude core transitions to the vacuum. The environment imposes a continuous tangential "wind" or phase gradient on this shell.
The \textbf{Poincar\'e--Hopf (Hairy Ball) Theorem} dictates that no non-vanishing continuous tangent vector field can exist on a 2-sphere. Any such field must possess zeros (defects) whose indices sum to the Euler characteristic, $\chi(S^2) = 2$.
Consequently, the isotropic shell cannot respond isotropically. To accommodate the environmental interaction, it is topologically forced to develop singular points where the phase order parameter vanishes or unwinds.

\subsection{Monopole Instability vs. Dipole Stability}
The topological charge of 2 can be satisfied in two generic ways, which explain the absence of magnetic monopoles and the ubiquity of dipoles:
\begin{itemize}
    \item \textbf{The Monopole Solution (Unstable):} A single defect with index $+2$. This concentrates high gradient energy ($\propto n^2 = 4$) at a single point. It is thermodynamically unstable against splitting.
    \item \textbf{The Dipole Solution (Stable):} Two defects with index $+1$ (a "North" and "South" pole). This reduces the gradient cost ($\propto 1^2 + 1^2 = 2$) and allows the defects to repel each other to opposite poles, minimizing interaction energy.
\end{itemize}
Thus, a "featureless" sphere spontaneously breaks symmetry to form a \textbf{Dipole Axis}. This axis connects the two topological defects on the shell. The soliton is not a rigid sphere but a \textbf{oriented entity} defined by this spontaneous axis.

\subsection{Mechanical Spin and Quantum Uncertainty}
This geometric axis provides a local realist mechanism for spin and its apparent uncertainty:
\begin{enumerate}
    \item \textbf{Deterministic Axis:} At any instant, the particle possesses a definite geometric axis defined by its shell defects.
    \item \textbf{Thermodynamic Diffusion:} In isolation, this axis is buffeted by the vacuum noise bath ($\tau$). It undergoes rotational diffusion, creating an isotropic probability distribution over long times, recovering the spherically symmetric $s$-orbital of quantum mechanics.
    \item \textbf{Gyroscopic Latching:} During a measurement (e.g., Stern-Gerlach), the external field breaks the degeneracy. The macroscopic phase current circulation around the defects grants the axis \textbf{Phase Inertia} (angular momentum). The particle acts as a microscopic gyroscope, "latching" into alignment and maintaining it against noise for exponentially long timescales (the "forever" of standard QM), provided the environmental noise temperature $T_{\rm eff}$ remains low.
\end{enumerate}
This replaces intrinsic quantum indeterminacy with dynamic stability: the state is definite but diffuses due to the irreducible noise of the vacuum.

\section{Generations from Geometric Grades}
We propose that the three fermion generations are not arbitrary copies, but correspond to the three available "geometric grades" of topological defects in a 3D vacuum.

\subsection*{The Ladder of Geometric Defects}
In the single-field framework, a particle is a topological knot or defect. Because the emergent space is 3-dimensional, the algebra of such defects supports exactly three distinct grades of complexity, corresponding to the dimensionality of the subspace the defect couples to:

\begin{enumerate}
    \item \textbf{Generation 1 (The Line / Phase Defect):} The simplest defect is a pure phase winding around a point (or along a line). It couples primarily to the U(1) phase sector. This is the "ground state" topology. (e.g., Electron).
    \item \textbf{Generation 2 (The Surface / Metric Defect):} A defect that carries a phase winding \emph{plus} a twist in the local metric or surface geometry. It couples to U(1) and the SU(2) surface sector. This is an excited geometric state. (e.g., Muon).
    \item \textbf{Generation 3 (The Volume / Density Defect):} A defect carrying phase winding, surface twist, \emph{and} a compression or rarefaction of the lattice volume. It couples to U(1), SU(2), and the SU(3) volume sector. This is the highest-complexity state. (e.g., Tau).
\end{enumerate}

This identification explains "Why Three?": because 3D space allows defects of lines, surfaces, and volumes. There is no fourth spatial grade.

\subsection*{Stability and Decay}
The first generation is stable because the phase winding (charge) is a topological invariant that cannot be shed without annihilation. Higher generations are unstable because their "extra baggage" (surface twists, volume compressions) is not topologically protected in the same way; the vacuum can relax these distortions, shedding the energy and leaving the fundamental phase defect (Generation 1) behind.

\section{Mass Hierarchies from Vacuum Moduli}
Having established that fermion generations correspond to distinct geometric grades, we now explain their hierarchical masses. The mechanism replaces fundamental Yukawa couplings with a "Moduli Hierarchy," where the particle's mass is determined by which stiffness term of the vacuum free energy it stresses.

\subsection{The Moduli Hierarchy: Why $m_\tau \gg m_e$}
The mass of a defect is the energy cost of imposing its topology on the vacuum. Different geometric grades stress different "moduli" (stiffness parameters) of the free energy, which have vastly different magnitudes.

Recall the free energy density:
\begin{equation}
  \mathcal F[w,\phi] = \underbrace{\frac{\kappa}{2}\,w^2\,|\nabla\phi|^2}_{\text{Phase Stiffness (Soft)}} \;+\; \underbrace{\alpha_{\rm grad}\,|\nabla w|^2}_{\text{Surface Tension}} \;+\; \underbrace{V(w)}_{\text{Amplitude Potential (Hard)}}
\end{equation}

\paragraph{Generation 1 (Line Defect / Electron): The Soft Mode.}
A Grade 1 defect is a topological winding of the phase $\phi$ (a vortex). Outside a microscopic core, the amplitude remains relaxed at its vacuum value $w \approx w_*$. The energy cost is dominated by the \textbf{Phase Stiffness} term. Since the phase is the Goldstone mode of the broken symmetry, it is massless (long-range). The cost of a phase twist is gradient-limited and "soft," yielding the lightest mass scale.

\paragraph{Generation 3 (Volume Defect / Tau): The Hard Mode.}
A Grade 3 defect is a topological compression of the lattice volume. In the infinite-clique framework, the local node density maps to the field intensity $w^2$. Therefore, a volume defect forces a bulk displacement of the amplitude away from equilibrium ($w \neq w_*$) to sustain the compression. This stresses the \textbf{Amplitude Potential}. To stabilize the vacuum against collapse, the restoring force $V''(w_*)$ (the "Higgs" mass of the vacuum) must be extremely stiff. Consequently, maintaining a volume defect costs an energy proportional to the heavy amplitude gap.

\paragraph{Generation 2 (Surface Defect / Muon): The Intermediate Mode.}
A Grade 2 defect is a twist in the local metric structure (a domain wall or surface curvature). It implies gradients in the amplitude profile $\nabla w$ without necessarily forcing a bulk volume shift. Its energy is dominated by the \textbf{Amplitude Gradient} term $\alpha_{\rm grad}|\nabla w|^2$. This sits intermediate between the soft phase stiffness and the hard potential stiffness.

Thus, the mass hierarchy reflects the spectral hierarchy of the vacuum itself: $m_e : m_\mu : m_\tau \sim \text{Phase Gradient} : \text{Amplitude Gradient} : \text{Amplitude Potential}$.

\section{Bulk Resistance and the Quark/Lepton Distinction}
While geometric grades explain the generational ladder ($e,\mu,\tau$), they do not immediately explain the fundamental division between leptons (integer charge, unconfined) and quarks (fractional charge, confined). We propose that this distinction arises from a \textbf{Window-Local Chiral Bias} in the vacuum geometry.

\subsection{The Chiral Vacuum: a pseudoscalar $\theta$-condensate}
We formalize the ``threaded vacuum'' as a \emph{Lorentz-invariant} chiral order parameter: a pseudoscalar field $\theta(x)$ that enters the effective rotor-sector action through a CP-odd topological density. While the global net chirality over the infinite fractal hierarchy vanishes (see Chiral Bias paper), spontaneous symmetry breaking and annealing within our observational window result in a macroscopic region dominated by a single sign (our observed ``handed'' vacuum).

For an internal rotor connection with curvature $F_{\mu\nu}$ (SU(2)/SU(3) in the coarse-grained theory), the standard invariant is
\begin{equation}
  \mathcal L_{\rm bias} \;\propto\; \theta(x)\,\mathrm{Tr}\!\big(F_{\mu\nu}\tilde F^{\mu\nu}\big),\qquad
  \tilde F^{\mu\nu} := \tfrac12 \epsilon^{\mu\nu\rho\sigma}F_{\rho\sigma}.
\end{equation}
\noindent\textbf{Why this is the natural form.} $\theta$ is a (pseudo)scalar rather than a vector, so it does \emph{not} introduce a preferred spatial direction (no ``ether wind''). When $\theta$ is approximately constant, this term is a total derivative and does not modify the local propagation of gauge waves at leading (classical) order, but it \emph{does} bias the weights/energetics of topological sectors and defect configurations that carry nontrivial $\mathrm{Tr}(F\wedge F)$ support. In condensed-matter language, it plays the role of an \emph{effective helicity preference (helicity modulus)} for the rotor sector: configurations with one sign of integrated helicity/topological density are energetically preferred over the opposite sign.

\paragraph{Domain formation and fast alignment.}
The ``global chiral bias'' can arise dynamically by a standard symmetry-breaking domain process: random fluctuations nucleate patches with $\theta\approx+\theta_0$ and $\theta\approx-\theta_0$ (degenerate minima of an effective potential $U(\theta)$). As the amplitude sector stiffens and the universe cools, domain walls become energetically costly and rapidly annihilate, leaving a macroscopically large region dominated by a single sign (our observed ``handed'' vacuum). Residual small-scale fluctuations correspond to local, transient $\theta(x)$ variations inside the coarse-graining window.

\paragraph{How $\theta$ biases defects (``with'' vs.\ ``against'' the grain).}
Within this picture, opposite windings are no longer exactly degenerate once the defect excites rotor curvature in its core:
\begin{itemize}
    \item \textbf{Bias-matched winding (``with the grain'').} A defect whose internal twist/helicity produces $\mathrm{Tr}(F_{\mu\nu}\tilde F^{\mu\nu})$ of the sign favoured by $\theta$ has a lower effective self-energy. In the simplest case, an anchored line-like defect (electron-like) can be stable as a low-complexity object.
    \item \textbf{Bias-mismatched winding (``against the grain'').} The opposite winding is biased upward in energy; stabilizing it requires additional geometric structure that can redistribute the induced curvature/helicity through the full 3D rotor manifold. This motivates the ``bulk resistance'' scaffold as the minimal way to support the disfavoured charge without being unwound by the vacuum.
\end{itemize}
In this view, the CP-violating phases observed in flavour physics can be interpreted as small, residual misalignments/tilts of effective $\theta$-like parameters acting on unstable excitations and mixing, rather than an independent microscopic ingredient.

\subsection{The Proton as a "Bulk Anchor"}
To stabilize a charge against the grain (i.e., to create a stable Proton), one cannot use the mismatched 1D thread. Instead, one must build a rigid 3D scaffold that "pushes back" against the vacuum geometry in all three spatial dimensions simultaneously.
\begin{itemize}
    \item \textbf{The Tripod Mechanism:} The minimal stable structure is a "tripod" that anchors to all three independent gradient modes ($\partial_x, \partial_y, \partial_z$) of the 3D volume.
    \item \textbf{Quarks as Struts:} We identify the components of this tripod as \textbf{Quarks}. Each quark anchors one gradient direction.
    \item \textbf{Color as Direction:} The "Color" charge is simply the label for which spatial gradient mode ($\text{Red}\leftrightarrow x, \text{Green}\leftrightarrow y, \text{Blue}\leftrightarrow z$) the quark occupies.
\end{itemize}
   Thus, a Proton is stable not because it matches the thread, but because it is a "Volume Lock" (Grade 3 composite) that mechanically resists the bulk bias by spanning the full tangent space.

   \subsection{The Spectrum of Baryon Stability}
   The stability of any baryon depends on whether its internal geometry can resist the vacuum's chiral pressure. We model the quarks as structural elements based on their alignment with the vacuum grain:
   \begin{itemize}
       \item \textbf{Up Quark ($u$, $+\frac{2}{3}$):} Winds \emph{against} the grain. Acts as a rigid \textbf{Strut}.
       \item \textbf{Down Quark ($d$, $-\frac{1}{3}$):} Winds \emph{with} the grain. Acts as a flexible \textbf{Tie} (prone to relaxing into a line).
   \end{itemize}
   Stability requires at least two struts to lock the tripod open. This yields a clear stability spectrum:
   \begin{enumerate}
       \item \textbf{Proton ($uud$, $+1$):} \textbf{Two Struts.} The two Up quarks form a rigid structural arch. The single Down tie connects them but does not compromise integrity. \textbf{Result: Stable Ground State.}
       \item \textbf{Neutron ($udd$, $0$):} \textbf{One Strut.} A single Up quark tries to hold the tripod open against two pulling Down ties. The structure possesses \textbf{soft phase modes}: the phase field in the directions transverse to the strut is not rigidly pinned and can undergo large-amplitude fluctuations. These low-energy geometric wobbles make the neutron volumetrically inefficient ($m_n > m_p$) and prone to ``snapping'' into a decay configuration (beta decay). Inside a nucleus, neighboring protons provide rigid boundary conditions that suppress these soft modes, stabilizing the particle. \textbf{Result: Metastable (Free) / Stable (Bound).}
       \item \textbf{Delta-Minus / Negative Proton ($ddd$, $-1$):} \textbf{Zero Struts.} Three flexible ties with no rigid support. The vacuum pressure instantly collapses the tripod. The three flux tubes merge into a single bundle winding with the grain. \textbf{Result: Unstable $\to$ Collapse to Electron ($e^-$).}
       \item \textbf{Delta-Plus-Plus ($uuu$, $+2$):} \textbf{Three Struts.} A hyper-rigid structure. While geometrically stable, forcing three struts against the grain costs excessive energy. It rapidly decays via strong interaction to the lower-energy 2-strut state ($p^+$). \textbf{Result: Short-lived Resonance.}
   \end{enumerate}

   \subsection{Nuclear Structure: Phase Continuity and the Valley of Stability}
   The stability of the nucleus is not merely a packing problem, but a global optimization of the phase field topology. The balance between Protons ($Z$) and Neutrons ($N$) emerges from the competition between \textbf{Coulomb Repulsion} and \textbf{Phase Gradient Minimization} (the Strong Force).

   \begin{itemize}
       \item \textbf{Protons as Rigid Boundary Conditions:} The Proton ($uud$) is a rigid 2-strut structure. Its phase field is locked into a high-gradient configuration that cannot relax. While mechanically stable, it carries a heavy Coulomb penalty.
       \item \textbf{Neutrons as Phase Buffers:} The Neutron ($udd$) is neutral but contains soft phase modes. By itself, it is unstable. However, when placed adjacent to a Proton, the rigid gradients of the Proton act as a \textbf{Boundary Condition} for the Neutron. The requirement that the phase field be continuous across the interface ($\nabla \phi_1 = \nabla \phi_2$) pins the Neutron's soft modes, preventing the fluctuations that lead to decay.
       \item \textbf{The Strong Force as Continuity Energy:} The "Strong Force" in this framework is simply the energy saved by sharing a continuous phase field. Two adjacent baryons that match phase gradients at their boundary have significantly lower total energy than two isolated baryons with mismatched, abrupt gradients. The system binds to minimize the integral of $|\nabla \phi|^2$ over the total volume.
       \item \textbf{The Valley of Stability:}
       \begin{enumerate}
           \item \textbf{Small Nuclei ($N \approx Z$):} The stabilization of Neutrons via boundary conditions dominates. Every Neutron needs a Proton neighbor to pin its soft modes. This favors a 1:1 ratio.
           \item \textbf{Heavy Nuclei ($N > Z$):} As the nucleus grows, the long-range Coulomb repulsion ($Z^2$) grows faster than the short-range surface binding ($\propto A$). To prevent Coulomb explosion, the nucleus must dilute its charge density. It packs in extra Neutrons to increase the volume (pushing protons apart) while still maintaining the phase continuity network.
           \item \textbf{Drip Lines:} If $N \gg Z$, there are not enough rigid boundaries to pin all the soft modes; excess neutrons are unconstrained and "drip" off or decay. If $Z \gg N$, the Coulomb penalty overcomes the gradient energy savings, blowing the nucleus apart.
       \end{enumerate}
   \end{itemize}

   \subsection{Confinement as Geometric Instability}
This picture naturally yields confinement. An isolated quark is a "monopod"---it anchors only one spatial dimension (e.g., $x$). Against a 3D bulk bias, it is mechanically unstable; the unanchored directions ($y, z$) allow the bulk twist to rotate or dissolve it. Stability is only achieved when all three legs of the tripod are present (Baryon) or when a quark-antiquark pair locks (Meson). Confinement is the vacuum's enforcement of the "Bulk Resistance" condition.

\subsection{The Top vs. Tau Split: Active vs. Passive Volume}
This geometric distinction also resolves the puzzle of why the Top Quark ($t$) is $\sim 100\times$ heavier than the Tau Lepton ($\tau$), despite both being "Volume" (Grade 3) defects.
\begin{itemize}
    \item \textbf{Tau (Passive Compliance):} As a lepton, the $\tau$ winds \emph{with} the grain. It is a "Volume" defect only in the sense that it occupies space (like an inflated balloon). It does not fight the vacuum twist; it simply displaces amplitude. Its mass comes from the passive volume cost: $E_\tau \propto \text{Vol} \times V_{\rm pot}$.
    \item \textbf{Top (Active Resistance):} As a quark, the $t$ winds \emph{against} the grain. It is a "Volume" defect in the sense of a rigid strut holding itself open against the crushing chiral pressure of the bulk. Its mass includes a massive "Shear Stress" term required to maintain the artificial left-handed cavity. $E_t \propto \text{Vol} \times (V_{\rm pot} + \text{Shear Stress})$.
\end{itemize}
The factor of $\sim 100$ reflects the ratio of the chiral shear modulus to the simple amplitude potential stiffness.

\subsection{The Twisted Ribbon Visualization: Why Fractional Charge?}
How can a topological winding be fractional? A loop must normally wind an integer number of times ($2\pi n$) to be single-valued. The "Bulk Resistance" tripod resolves this via a \textbf{Twisted Ribbon} geometry.

Imagine the vacuum phase field as a flat ribbon.
\begin{enumerate}
    \item \textbf{Integer Charge (Lepton):} You take the ribbon, twist it $360^\circ$ ($2\pi$), and glue the ends. This is a stable, independent loop.
    \item \textbf{Fractional Charge (Quark):} You twist the ribbon by a fraction of a full turn (illustratively $2\pi/3$, the minimal volumetric closure).
    \begin{itemize}
        \item \textbf{Problem:} You cannot glue the ends to make a loop; the phase doesn't match.
        \item \textbf{Solution:} You need \textbf{three} such ribbons to meet at a central axis (the "flux tube"). Ribbon A ($0 \to 120^\circ$) + Ribbon B ($120 \to 240^\circ$) + Ribbon C ($240 \to 360^\circ$).
    \end{itemize}
\end{enumerate}
Together, they form a complete $360^\circ$ winding around the common axis. This explains:
\begin{itemize}
    \item \textbf{Confinement:} A single $120^\circ$ twist cannot exist alone because its phase boundary would extend to infinity as a "discontinuity sheet" (infinite energy). It \emph{must} terminate on other partial twists to close the geometry.
    \item \textbf{Asymptotic Freedom:} Close to the center, the ribbons are just 3 separate defects. Far away, they look like one integer defect (the Proton).
    \item \textbf{The "Volume Defect":} The junction where these three twisted sheets meet creates a unique Trivector (Volume) distortion in the lattice, distinguishing quarks (Volume/Grade-3) from leptons (Line/Surface).
\end{itemize}

\section{Geometric Mechanisms of Interaction}
Standard Model vertices (decays) are not arbitrary points but geometric reconfiguration events governed by two conservation laws: Topological Conservation (Charge winding) and Geometric Shedding (Momentum/Spin conservation).

\subsection{Lepton Decay: Anchor Splitting}
Consider the decay $\mu^- \to e^- + \bar{\nu}_e + \nu_\mu$.
\begin{enumerate}
    \item \textbf{Initial State:} A Muon ($\mu^-$) is a \textbf{Phase Anchor} (Charge) coupled to a \textbf{Surface Twist} (Grade-2 Baggage).
    \item \textbf{Splitting:} Vacuum pressure acts on the unstable Twist, pinching it off from the Anchor. The Twist detaches as a free-floating, chargeless bivector packet ($\nu_\mu$).
    \item \textbf{Excitation:} The Anchor is left in a highly excited, "shaken" state (Virtual $W^-$ boson). It still has charge -1 but has lost its defining muon geometry.
    \item \textbf{Relaxation:} The excited Anchor snaps into the ground state (Electron). To conserve the angular momentum/spin of the transition, it must emit a counter-balancing geometric ripple ($\bar{\nu}_e$).
    \item \textbf{Final State:} Stable Electron + two geometric packets ($\nu_\mu, \bar{\nu}_e$).
\end{enumerate}

\subsection{Beta Decay: Structural Failure and Ejection}
Beta Decay ($n \to p^+ + e^- + \bar{\nu}_e$) is the vacuum-induced failure of the Neutron's metastable tripod.
\begin{enumerate}
    \item \textbf{The Failure:} The Neutron ($udd$) relies on a single Up strut to hold its volume open. Thermal fluctuations eventually cause this single strut to buckle. The structure collapses.
    \item \textbf{The Ejection:} To save the nucleon, the system ejects the source of the structural weakness: the negative winding of a Down quark. The collapsing $d$-leg (charge $-1/3$) effectively rips a full unit of negative flux ($-1$) out of the vacuum to form a stable, independent defect.
    \item \textbf{The Repair:} The ejection of $-1$ charge leaves the tripod with a net gain of $+1$. The failed $d$ leg ($-1/3$) is replaced by a $u$ leg ($+2/3$). This locks the tripod into the stable, two-strut Proton configuration ($uud$).
    \item \textbf{The Debris:} The ejected negative flux relaxes instantly to its topological ground state: the \textbf{Electron} (Line Defect). The violent twist required to rip the leg off sheds geometric shrapnel as the Antineutrino ($\bar{\nu}_e$).
\end{enumerate}
Thus, the Electron is not created ``from nothing''; it is the \textbf{ground state of the failed structural element} of the baryon.
The Weak Force is thus identified as the mechanics of \textbf{Phase Anchor Splitting/Reconfiguration}.

\section{Mixing Matrices from Basis Misalignment}
The CKM and PMNS matrices, which govern inter-generational mixing, arise as a misalignment between the basis of interaction (textures) and the basis of mass.

\subsection*{Interaction vs mass bases}
Two bases shape phenomenology: (i) the interaction (texture) basis $\{\Xi_1,\Xi_2,\Xi_3\}$ to which SU(2) couples, and (ii) the mass basis $\{f_1,f_2,f_3\}$ obtained by diagonalizing $M_f$. If they coincided, no mixing would occur. In general, a mass eigenstate is a superposition of textures, and weak interactions act in the texture basis.

\subsection*{Source of misalignment}
Flavour-dependent weights differ, $\mathcal W_u\ne \mathcal W_d$, giving distinct mass matrices and unitary rotations:
\begin{equation}
  M_u = U_u D_u U_u^\dagger, \qquad M_d = U_d D_d U_d^\dagger, \qquad V_{\rm CKM} = U_u^\dagger U_d.
\end{equation}
Analogously, $V_{\rm PMNS} = U_e^\dagger U_\nu$.

\subsection*{Predicted structure: CKM vs PMNS}
Small quark mixing follows if $\mathcal W_u\approx \mathcal W_d$ so $U_u\approx U_d$ (CKM nearly diagonal). Large lepton mixing follows if the neutrino weight $\mathcal W_\nu$ differs structurally from $\mathcal W_e$ (e.g., seesaw-like origin), yielding sizable misalignment and large PMNS angles.

\section{Running, Window Dependence, and Environment}
In this framework, fermion masses and mixing angles are effective parameters that depend on the observational scale and the local environment via two principles: the sliding analysis window and the background noise field.

The mass matrix elements $(M_f)_{ij}$ are overlap integrals that depend on effective parameters (e.g., $\kappa,\beta_{\rm pot}$). Changing the experimental scale $\mu$ is equivalent to changing the coarse-graining window, which integrates out different mode bands of the link network and renormalizes the couplings and thus $M_f$.

The background noise field $\rho_N$ sets the physical scale of this running. Denser $\rho_N$ alters stiffnesses and correlation lengths, weakly modulating texture shapes $\Xi_i$ and their overlaps, and causing small shifts in masses and mixing.

Conformal invariance constrains observables: local rulers and clocks co-vary with $\rho_N$, canceling leading absolute shifts. Hence dimensionless quantities (mass ratios, mixing angles, Jarlskog invariant) should be nearly universal, with tiny, predictable drifts correlated with the local gravitational/noise environment.

\section{Phenomenology and Testable Predictions}
This geometric model is predictive and falsifiable. Key predictions fall into internal relations among flavour parameters and external correlations with the environment.

\subsection*{Mass and mixing sum rules}
Simple texture ansätze (orthogonal modes inside the core) have few free parameters; once fixed, all overlaps $(M_f)_{ij}$ follow, imposing algebraic constraints among masses and mixing angles. Examples include relations linking quark mass ratios to the Cabibbo angle (GST-like) or new sum rules characteristic of this geometry.

\subsection*{Environmental correlations and drifts}
Weak dependence of masses and mixings on $\rho_N$ leads to null-search opportunities:
\begin{itemize}[leftmargin=*]
  \item \textbf{Atomic clocks:} compare optical vs hyperfine clocks for annual/tidal modulations in ratios (e.g., $m_e/m_p$) as Earth’s gravitational potential changes.
  \item \textbf{Neutrino oscillations:} search for small periodic modulations in $\Delta m^2$ correlated with solar/lunar tides or orbital eccentricity.
  \item \textbf{Flavour factories:} look for time/location dependent variations in CKM elements across experiments (LHCb, Belle II) to bound environmental coupling.
\end{itemize}

By tying flavour structure to geometric relations and environmental correlations, the framework converts parameter fitting into targeted, precision tests.

\appendix
\section{Texture Basis and Orthogonality}
We construct an orthonormal basis $\{\Xi_i\}$ adapted to the stabilized soliton core by solving the texture eigenproblem
\begin{equation}
  \mathcal O_{\rm tex}\,\Xi_i = \lambda_i\,\Xi_i,\qquad \int d^3x\,\Xi_i^*\Xi_j=\delta_{ij},
\end{equation}
with $\mathcal O_{\rm tex}$ defined in Section~3 and boundary conditions set by the core profile. The three lowest, sub-gap modes ($\lambda_i$ below the spectral gap) span the slow texture manifold and define the generation basis. Stability requires positive second variation of the full free energy along these modes; higher eigenmodes either lie above the gap or open unstable directions and are excluded. Small overlap scaling follows from orthogonality and increasing oscillation: $\int \Xi_i^*\Xi_j \to 0$ as complexity grows, suppressing off-diagonal mass terms.

\section{Stability of Geometric Grades}
We have identified the three generations with the three available geometric grades in 3D space: Line ($n=1$), Surface ($n=2$), and Volume ($n=3$). While geometry dictates the \emph{existence} of these three categories, it does not guarantee their \emph{stability}. Why do these topological defects not simply collapse or unwind? We model the stability of these grades using an energetic--entropic balance.

\subsection*{Energetic Cost vs Entropic Gain}
We model the free energy of a defect of grade $n$ (dimensionality of its support) as:
\begin{equation}
  F(n) \;=\; E_{\rm geom}(n) \; - \; TS_{\rm config}(n),
\end{equation}
where $E_{\rm geom}(n)$ is the energy cost of maintaining the defect structure (stiffness) and $S_{\rm config}(n)$ is the entropic gain from the accessible phase space of the defect.

\begin{itemize}
    \item \textbf{Geometric Cost ($E \propto n^2$):} The energy cost scales non-linearly with grade. A line defect ($n=1$) disrupts the order parameter locally. A surface defect ($n=2$) disrupts it over a larger subspace, and a volume defect ($n=3$) involves bulk compression. We model this as $E_{\rm geom} \approx C_E \, n^2$.
    \item \textbf{Entropic Gain ($S \propto n$):} Higher-dimensional defects have more internal degrees of freedom and configuration space. A volume defect has more ways to arrange its internal phase than a line defect. We model this as $S_{\rm config} \approx C_S \, n$.
\end{itemize}

\subsection*{Stability Window}
The net free energy is $F(n) = C_E n^2 - \Theta n + F_0$, where $\Theta = T C_S$. The extrema of this function determine the preferred grades.
Unlike a continuous minimization, the geometric grade $n$ is strictly discrete ($n \in \{1, 2, 3\}$). Stability requires that $F(n)$ is a local minimum relative to decay channels.
\begin{itemize}
    \item \textbf{Grade 1 (Electron):} Protected by U(1) topology. Always stable against decay to $n=0$ (vacuum) due to charge conservation.
    \item \textbf{Grade 2 \& 3 (Muon, Tau):} Stable only if local conditions (noise bath $\Theta$) prevent immediate collapse to lower grades.
\end{itemize}
The "hard" geometric limit of $d=3$ imposes a ceiling: there is no $n=4$ grade in 3D space. Thus, the spectrum of generations is truncated exactly at three not by soft energetics, but by the hard dimensionality of the vacuum. The energetic model simply governs the mass gaps and lifetimes.

\section{Effective Yukawas from Link Energetics}
We now connect the flavour weights $\mathcal W_f(x)$ appearing in the overlap integrals to coarse–grained link energetics, making explicit how sector dependence and weak running arise.

\subsection*{Origin of $\mathcal W_f$}
The flavour weight captures the energetic preference for aligning (or misaligning) rotor texture with the amplitude profile. Starting from the free-energy density
\begin{equation}
  \mathcal F[w,\phi] = \alpha_{\rm grad}|\nabla w|^2 + \tfrac12\kappa w^2 |\nabla\phi|^2 + V(w) + \delta\mathcal F_{\rm int},\qquad V(w)=-\beta_{\rm pot}w^2+\gamma w^4,
\end{equation}
integrating out fast fluctuations and projecting onto the slow texture subspace induces quadratic forms in the texture amplitudes with kernels that depend on $(\kappa,\beta_{\rm pot},\gamma,\eta_0)$ and the core profile $w_*(r)$. To leading order the sector dependence (up, down, charged lepton, neutrino) enters through different couplings to the rotor–alignment (VEV) field and weak backreaction terms, yielding
\begin{equation}
  \mathcal W_f(x) \;\simeq\; c_f\,\kappa\,w_*^2(x)\; +\; d_f\,\beta_{\rm pot}\,\chi(x)\; +\; e_f\,\eta_0\,\zeta(x),
\end{equation}
where $c_f,d_f,e_f$ are dimensionless sector coefficients, and $\chi,\zeta$ are smooth functionals of the core profile and connectivity (constant in local units to leading order).

\subsection*{Scaling with $\rho_N$ and window}
Local conformal invariance implies $c_s^2=\kappa w_*^2$ is constant in local units, while $\kappa,\beta_{\rm pot},\eta_0$ run mildly with the analysis window and $\rho_N$. Consequently,
\begin{equation}
  \mathcal W_f(x;\rho_N) \;=\; \bar{\mathcal W}_f(x)\,\big[1 + \delta_f(\rho_N)\big],\qquad |\delta_f|\ll 1,
\end{equation}
with $\delta_f$ capturing the weak, sector-dependent running. Because the overlap integrals involve normalized $\Xi_i$, the leading running cancels in ratios and in mixing angles, leaving only small, potentially measurable drifts (Section~7).

\subsection*{Summary expression for overlaps}
Collecting terms, the mass-matrix elements can be written as
\begin{equation}
  (M_f)_{ij} \;=\; \int d^3x\; \Big[\,c_f\,\kappa w_*^2 + d_f\,\beta_{\rm pot}\,\chi + e_f\,\eta_0\,\zeta\,\Big]\,\Xi_i^*\,\Xi_j\; +\; \mathcal O(\text{higher orders}),
\end{equation}
where higher orders include suppressed nonlocalities and derivative corrections from integrating out fast modes. This provides a concrete route to calibrate $c_f,d_f,e_f$ against spectra and to compute controlled window/environmental sensitivities.

\section{Majorana Option (Neutrinos): The "Flying Geometry"}
We outline how Majorana neutrino masses arise in this framework from integrating out heavy rotor-sector texture modes, yielding the standard dimension-5 operator and a type-I–like seesaw.

\subsection*{Dirac vs Majorana and "Shed Baggage"}
In this framework, neutrinos are naturally Majorana-like but with a geometric twist. We propose the \textbf{"Shed Baggage" Hypothesis}:
When a heavy lepton (e.g., Muon) decays, it sheds its "Surface Twist" geometry to become an Electron (Phase Defect). This shed geometry---a packet of twisted metric/surface distortion without a topological phase anchor---propagates as a \textbf{Neutrino}.

\paragraph{Step 1: U(1) anchoring $\Rightarrow$ superselection.}
Charged leptons are U(1)-anchored phase defects: an integer phase winding is pinned to an amplitude core and cannot change continuously without an amplitude zero (unwinding) or defect reconnection. This makes the ``grade'' of a charged excitation effectively superselected in the low-energy window: the vacuum can relax non-protected distortions, but it cannot freely convert a charged, anchored defect between inequivalent topological classes while preserving the winding.

\paragraph{Step 2: Neutral packets $\Rightarrow$ generic rotor-bath evolution (oscillations).}
The shed object carries no U(1) anchor (no Vector current), so its internal SU(2)/SU(3) orientation is not pinned by a conserved winding. It is a "flying geometry" (pure Bivector/Trivector packet). As it propagates through the noise bath, it is governed by the same generic rotor-bath Hamiltonian structure used in the framework's oscillation mapping: an internal state vector precesses under a small vacuum splitting plus a medium/noise potential,
\begin{equation}
  i\,\partial_t \Psi \;=\; \Big(H_0 \;+\; V_{\rm eff}(\rho_N)\Big)\Psi,
\end{equation}
with $V_{\rm eff}$ (and dephasing) induced by coupling to $\rho_N$ within the observational window. In this view, neutrino flavour oscillations are not an extra postulate but a generic consequence of unanchored internal-rotor dynamics in a fluctuating medium: without an anchor the orientation can rotate (mix) and accumulate relative phase along the path. While the shed twist lacks the massive amplitude core of a charged lepton, it retains a residual geometric curvature energy---the cost of the metric/surface distortion itself ($E \sim \alpha_{\rm grad}\int |\nabla \Xi|^2$). This irreducible curvature energy constitutes the non-zero neutrino mass floor.

\subsection*{Seesaw mapping}
Mathematically, this aligns with the standard Type-I seesaw...
\begin{equation}
  y_{i\alpha} \;\sim\; \int d^3x\; \mathcal W_\nu(x)\,\Xi_i(x)\,\Upsilon_\alpha(x),\qquad (M_R)_{\alpha\beta} \;\sim\; M_{\rm heavy}\,\int d^3x\; \Upsilon_\alpha\,\Upsilon_\beta,
\end{equation}
where $\Xi_i$ are the light texture modes (generations) and $\Upsilon_\alpha$ are heavy texture profiles localized at the core. The heavy scale $M_{\rm heavy}$ is set by rotor-sector stiffness and alignment, and is naturally far above $v_{\rm eff}$.

\subsection*{Mixing, phases, and diagonalization}
The symmetric Majorana matrix is diagonalized as $M_\nu = U_\nu\,D_\nu\,U_\nu^{\,T}$ with $D_\nu = \mathrm{diag}(m_1,m_2,m_3)$. The leptonic mixing matrix is $V_{\rm PMNS}=U_e^{\dagger} U_\nu$ and contains, in addition to a Dirac phase, two Majorana phases that can affect lepton-number–violating processes.

\subsection*{Framework links and scaling}
The weights $\mathcal W_\nu$, and hence $y$ and $M_R$, inherit weak window/environment dependence through $(\kappa,\beta_{\rm pot},\eta_0)$ as in Appendix~C. Conformal invariance keeps leading effects constant in local units; residual drifts are small and correlate with $\rho_N$.

\subsection*{Phenomenology}
\begin{itemize}[leftmargin=*]
  \item \textbf{0$\nu\beta\beta$:} the effective mass $m_{\beta\beta}=\big|\sum_i (V_{\rm PMNS})_{ei}^2\,m_i\big|$ controls the rate. This framework predicts the standard structure with potential tiny environmental drifts.
  \item \textbf{Hierarchy and scales:} for $M_{\rm heavy}\gg v_{\rm eff}$, observed masses follow from overlaps and the seesaw suppression $\propto v_{\rm eff}^2/M_{\rm heavy}$. Texture geometry can favor normal or inverted orderings depending on $\mathcal W_\nu$.
  \item \textbf{Running:} scale dependence enters via the window; mixing angles and mass ratios are nearly invariant in local units, offering precise null tests across environments.
\end{itemize}

\section{Core Model and Spectrum of $\mathcal O_{\rm tex}$}
We specify an explicit core profile and the associated texture operator, then outline the numerical eigenspectrum demonstrating three sub-gap, stable modes.

\subsection*{Core profile and operator}
Adopt a spherical, monotone amplitude profile
\begin{equation}
  w_*(r) = w_0\,\tanh\!\Big(\frac{r}{r_c}\Big),\qquad r_c = \sqrt{\frac{\kappa}{\beta_{\rm pot}}},
\end{equation}
consistent with the Mexican-hat potential and phase stiffness. The texture operator acting on slow rotor-texture amplitudes (after gauge fixing and projecting out fast modes) is
\begin{equation}
  \mathcal O_{\rm tex}\,\Psi = -\nabla\!\cdot\!(\kappa w_*^2\nabla\Psi) \; + \; \lambda_{\rm NL}\,\Pi_{\rm constr}[\Psi],\qquad \lambda_{\rm NL}\ll 1,
\end{equation}
where $\Pi_{\rm constr}$ enforces weak nonlinear constraints (e.g., amplitude clamping, residual symmetry). We work in a ball of radius $R\approx 6\,r_c$, with regularity at $r=0$ and matching-to-vacuum at $r=R$ (Dirichlet $\Psi(R)=0$ or an outgoing-decay condition).

\subsection*{Separation of variables and discretization}
Expand in spherical harmonics, $\Psi(r,\theta,\phi)=\sum_{\ell m} u_{\ell}(r)Y_{\ell m}(\theta,\phi)$, yielding radial equations
\begin{equation}
  -\frac{1}{r^2}\frac{d}{dr}\Big(r^2\kappa w_*^2\frac{d u_{\ell}}{dr}\Big) + \kappa w_*^2\,\frac{\ell(\ell+1)}{r^2}u_{\ell} = \lambda\, u_{\ell} \quad (+\;\text{weak constraints}).
\end{equation}
Discretize $r\in[0,R]$ (e.g., finite differences or finite elements), truncate $\ell\le 3$, and assemble the sparse eigenproblem for the lowest eigenpairs.

\subsection*{Expected spectrum and stability}
For physical $(\kappa,\beta_{\rm pot},\gamma)$ in local units, the spectrum exhibits three sub-gap eigenvalues $\lambda_1<\lambda_2<\lambda_3$ separated from the continuum by a spectral gap. The corresponding eigenfunctions $\Xi_1$ (node-free), $\Xi_2$ (one node), $\Xi_3$ (multi-node) match the generation textures. Stability is confirmed by sampling the second variation of the full free energy on $\mathrm{span}\{\Xi_1,\Xi_2,\Xi_3\}$ and verifying positivity.

\paragraph{Deliverables (to be included).} Tabulate $(\lambda_i)$, plot $u_{\ell}(r)$ profiles for dominant $\ell$ in each mode, and provide a spectrum figure illustrating the gap.

\section{Reduced Landau Functional: cubic invariant from residual symmetry}
We derive the reduced Landau functional by projecting the full free energy onto the slow texture manifold and identifying symmetry-allowed invariants.

\subsection*{Order parameter and invariants}
Let $\mathbf T=(T_1,T_2,T_3)$ be the amplitudes of the three slow texture modes $\{\Xi_i\}$ after gauge fixing residual global phases. The residual symmetry (permutations of $T_i$ and sign flips constrained by texture parity) allows the invariants
\begin{equation}
  I_2 = |\mathbf T|^2=\sum_i T_i^2,\qquad I_3 = T_1 T_2 T_3,\qquad I_4 = (\sum_i T_i^2)^2,\;\sum_i T_i^4.
\end{equation}
The cubic invariant $I_3$ is generically nonzero unless forbidden by an additional accidental symmetry.

\subsection*{Coefficient projection}
Project the free energy $F$ onto $\mathrm{span}\{\Xi_i\}$: write $\Psi=\sum_i T_i\,\Xi_i$ and compute
\begin{equation}
  a = \frac{1}{2}\sum_{ij} T_i T_j \int \Xi_i\,\mathcal O_{\rm tex}\,\Xi_j,\qquad b \propto \int \Xi_1\,\Xi_2\,\Xi_3\;\mathcal K(x),\qquad c>0,
\end{equation}
with $\mathcal K$ a smooth kernel induced by integrating out fast modes and nonlinearities. This yields the reduced form
\begin{equation}
  F_{\rm red}(\mathbf T) = a\,I_2 + b\,I_3 + c\,I_4 + \dots,\qquad c>0.
\end{equation}
For $a<0$ and $b\ne 0$ in a finite band, the normal form is the A$_3$ (cusp) with three inequivalent minima (tri-stability).

\paragraph{Deliverables (to be included).} Numerical estimates of $(a,b,c)$ from the computed $\Xi_i$; contour plots of $F_{\rm red}$ showing three minima; verification that $b\to 0$ only under fine-tuned symmetry.

\section{Toy Sector Weights and Mass/Mixing Fit}
We provide an illustrative numerical example using the overlap formula with simple sector weights to reproduce qualitative mass hierarchies and mixing patterns.

\subsection*{Setup}
Define normalized overlaps $S_{ij}=\int \kappa w_*^2\,\Xi_i\,\Xi_j$, with $S\approx \mathrm{diag}(1,\sigma,\sigma^2)$ and small off-diagonals $\epsilon$ from imperfect orthogonality (e.g., $\sigma\approx 0.2$, $\epsilon\approx 0.02$). Choose sector weights
\begin{equation}
  \mathcal W_u= 1.00\,\kappa w_*^2,\quad \mathcal W_d= 0.98\,\kappa w_*^2,\quad \mathcal W_e=0.95\,\kappa w_*^2,\quad \mathcal W_\nu=0.70\,\kappa w_*^2,
\end{equation}
so that up/down are closely aligned, while $\nu$ differs structurally.

\subsection*{Mass matrices and mixing}
Form $M_f\approx c_f\,S + \delta_f\,\Delta$ with small perturbations $\Delta$ from subleading terms (Appendix~C). A representative choice yields
\begin{equation}
  M_u \propto \begin{pmatrix} 1 & \epsilon & \epsilon \\ \epsilon & \sigma & \epsilon \\ \epsilon & \epsilon & \sigma^2 \end{pmatrix},\quad M_d \propto \begin{pmatrix} 0.98 & \epsilon & \epsilon \\ \epsilon & 0.98\,\sigma & \epsilon \\ \epsilon & \epsilon & 0.98\,\sigma^2 \end{pmatrix},\quad M_e \propto \begin{pmatrix} 0.95 & \epsilon & \epsilon \\ \epsilon & 0.95\,\sigma & \epsilon \\ \epsilon & \epsilon & 0.95\,\sigma^2 \end{pmatrix}.
\end{equation}
Diagonalization gives hierarchical eigenvalues ($\propto 1,\sigma,\sigma^2$) and nearly aligned $U_u\approx U_d$ (CKM close to identity), while $M_\nu\propto S$ with distinct normalization can generate large leptonic mixing when combined with $M_e$.

\paragraph{Deliverables (to be included).} A concrete numeric instance (e.g., $\sigma=0.2$, $\epsilon=0.02$) with extracted approximate mass ratios and mixing matrices; sensitivity analysis for $(c_f)$.

\section{Bounds on Environmental Drifts}
We quantify the expected size of environment-induced variations in masses and mixing parameters.

\subsection*{Linear response}
For small changes in the noise field $\rho_N$, the flavour weights obey $\delta \ln \mathcal W_f \equiv \alpha_f\,\delta\ln \rho_N$ with $|\alpha_f|\ll 1$. To leading order, mass eigenvalue shifts satisfy
\begin{equation}
  \delta \ln m_i \;\approx\; w_i\,\alpha_f\,\delta\ln\rho_N,\qquad w_i=\text{mode weight},
\end{equation}
and mixing angles shift according to first-order perturbation of $M_f$ (off-diagonal sensitivity proportional to $\epsilon$).

\subsection*{Astrophysical bounds}
For Earth’s orbital eccentricity, the gravitational potential variation is $\Delta\Phi/c^2\approx 3\times10^{-10}$. If $\rho_N\propto \Phi$ to first order in the relevant window, then $|\delta\ln\rho_N|\approx 3\times10^{-10}$. With $|\alpha_f|\lesssim 10^{-2}$ (consistent with slow running), one finds
\begin{equation}
  |\delta \ln m_i| \lesssim 3\times10^{-12},\qquad |\delta(\Delta m^2)/\Delta m^2| \lesssim 6\times10^{-12},\qquad |\delta V_{ij}| \lesssim \mathcal O(10^{-12}).
\end{equation}
These are targets for null tests with atomic clocks, long-baseline neutrinos, and precision flavour factories.

\paragraph{Notes.} Local conformal normalization further suppresses absolute drifts in local units; reported bounds correspond to dimensionless ratios and mixing elements.


\end{document}
